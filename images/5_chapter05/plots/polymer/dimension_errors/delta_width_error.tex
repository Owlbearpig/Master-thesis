\documentclass{standalone}
\usepackage{tikz}
\usepackage{tikz-3dplot}
\usepackage{pgfplots}
\usepackage{pgfplotstable}
\usepgfplotslibrary{fillbetween}
\usepackage{siunitx}
\pgfplotsset{compat=1.5}

\begin{document}

\pgfplotsset{
    table/search path={plot_data/5_chapter05/polymer/dimension_errors},
}

\begin{tikzpicture}
	\begin{axis}[
	    name=ax1,
	    %scale only axis,
        ymin=0.29, ymax=0.58,
        xmin=71, xmax=114,
        %axis y line*=left,
        no markers,
        every axis plot/.append style={semithick},
        axis line style = very thick,
        every tick/.style={black, very thick},
        %ytick={1.25, 1.5708, 1.80},
        %yticklabels={1.25, $\frac{\pi}{2}$, 1.80},
        xlabel=Frequency (\si{\giga \hertz}),
        ylabel=Phase shift $\frac{\delta}{\pi}$ (rad),
        legend entries={{Calculated}, {Measured}},
        legend style={legend columns=1, at={(0.1,0.4)}, anchor=north west,
                      nodes={scale=0.7, transform shape}}
        ]
        
        \addplot[mark=none, black, samples=4, dotted, domain=0:130] {0.5*1.03};
    	\addlegendentry{$\pm\SI{3}{\percent}$}
    	\addplot[mark=none, black, samples=4, dotted, domain=0:130,forget plot] {0.5*0.97};
        
        %\foreach \m in {2,3,...,11}{
        %	\addplot[red] table [
        %	x expr=\thisrowno{1}*10^-9,
        %	y expr=\thisrowno{\m},
        %	col sep=comma] 
        %	{5percent_width_error.csv};
        %	%\addlegendentry{Calculated}
        %}
        \addplot[red] table [
        	x expr=\thisrowno{1}*10^-9,
        	y expr=\thisrowno{2}/3.14159,
        	col sep=comma] 
        	{delta_measured_6degOffset.csv};
        	\addlegendentry{Measured}
        
    	\addplot[blue] table [
            x expr=\thisrowno{1}*10^-9,
            y expr=\thisrowno{11}/3.14159,
            col sep=comma] 
            {5percent_width_error.csv};
            \addlegendentry{Curve number 10}
    	
    	\addplot [green, dashed, thick] table [
    	    x expr=\thisrowno{1}*10^-9,
    	    y expr=\thisrowno{52}/3.14159,
    	    col sep=comma] 
    	    {5percent_width_error.csv};
    	    \addlegendentry{Minimum average}
        \addplot [blue, dashed, thick] table[
    	    x expr=\thisrowno{1}*10^-9,
    	    y expr=\thisrowno{53}/3.14159,
    	    col sep=comma] 
    	    {5percent_width_error.csv};
            \addlegendentry{Maximum average}
    	
	\end{axis}
\end{tikzpicture}

\begin{tikzpicture}
	\begin{axis}[
	    name=ax1,
	    %scale only axis,
        ymin=0.29, ymax=0.58,
        xmin=71, xmax=114,
        %axis y line*=left,
        no markers,
        every axis plot/.append style={semithick},
        axis line style = very thick,
        every tick/.style={black, very thick},
        %ytick={1.25, 1.5708, 1.80},
        %yticklabels={1.25, $\frac{\pi}{2}$, 1.80},
        xlabel=Frequency (\si{\giga \hertz}),
        ylabel=Phase shift $\frac{\delta}{\pi}$ (rad),
        legend entries={{Calculated}, {Measured}},
        legend style={legend columns=1, at={(0.5,0.95)}, anchor=north west,
                      nodes={scale=0.7, transform shape}}
        ]
        
        \addplot[mark=none, black, samples=4, dotted, domain=0:130] {0.5*1.03};
    	\addlegendentry{$\pm\SI{3}{\percent}$}
    	\addplot[mark=none, black, samples=4, dotted, domain=0:130,forget plot] {0.5*0.97};
        
        %\foreach \m in {2,3,...,11}{
        %	\addplot[red] table [
        %	x expr=\thisrowno{1}*10^-9,
        %	y expr=\thisrowno{\m},
        %	col sep=comma] 
        %	{5percent_width_error.csv};
        %	%\addlegendentry{Calculated}
        %}
        \addplot[red] table [
        	x expr=\thisrowno{1}*10^-9,
        	y expr=\thisrowno{2}/3.14159,
        	col sep=comma] 
        	{delta_measured_6degOffset.csv};
        	\addlegendentry{Measured}
        
    	\addplot[blue] table [
            x expr=\thisrowno{1}*10^-9,
            y expr=\thisrowno{2}/3.14159,
            col sep=comma] 
            {5percent_shrink_error.csv};
            \addlegendentry{\SI{5}{\percent} shrinkage}
    	
	\end{axis}
\end{tikzpicture}
\end{document}
