The frequency range from around \SI{100}{\giga \hertz} to \SI{24}{\tera \hertz} located midway between microwaves and infrared light is known as the terahertz (THz) frequency range \cite{Jepsen2011}. Electromagnetic (EM) waves in this range have numerous applications not only for fundamental research but also for the application in optoelectronic THz systems. The energy of THz photons range from \SI{1}{\milli \electronvolt} to \SI{100}{\milli \electronvolt} which allows probing of physical phenomena such as low-energy excitations in 2D materials \cite{Winnerl2018} and rotational as well as vibrational modes via time domain spectroscopy \cite{Walther2002, Giesen2005, Baxter2011, Mittleman2003}. In addition, compared to other frequency ranges THz photons are not energetic enough to be ionizing which means that spectroscopy utilizing THz radiation preserves the sample of interest and is non hazardous \cite{Jepsen2011}.

Due to the increasing number of applications and research based on THz radiation, characterization and control of the field polarization also gained interest especially with the development of waveplates amongst other polarization related components \cite{Castro-Camus2012}. Waveplates and other polarization converters usually depend on the wavelength. This poses a problem for example in combination with time domain spectroscopy due to the usage of broadband radiation and it can be a limitation for polarimetry in general because of this. However, designs based on the combination of multiple monochromatic waveplates at different azimuths have been used to realize achromatic waveplates in the past \cite{Wu2020, Masson2006, Ivanov2012, Herrera-Fernandez2015}. In this work we propose and show the experimental demonstration of a similar design which is however based on a single 3D printed structure. With this method production cost and complexity can in theory be reduced. In addition, it has the advantage of reducing angle misalignment errors since this technique completely circumvents manual assembly.

To that extend we introduce THz time domain spectroscopy specifically generation and detection of THz radiation in the following chapter. In the third chapter the three experimental spectroscopy setups which have been used for the measurements in this work are presented. Following, in the fourth chapter the theoretical fundamentals are described which includes different representations of the polarization state, the underlying optical properties of anisotropic media which are utilized to transform one polarization state into another via waveplates and finally how the Jones calculus can be used to describe the achromatic waveplates. The characterization of the fabricated waveplate design and a discussion thereof is presented in the fifth chapter. A conclusion as well as an outlook is given in the final chapter.

%For example magnetic fields in the interstellar medium induce polarization due to the Faraday effect which can be characterized using waveplates \cite{Harper2018}.

%- Motivation (maybe just do this at presentation)
%Polarization control is an important aspect of optics. etc https://en.wikipedia.org/wiki/BB84
%    - usages of waveplates
%    - Folder with examples from litterature

% - Structure