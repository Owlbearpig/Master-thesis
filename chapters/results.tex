Using the mathematical description and the optimization approach presented in chapter \ref{ch:wp_design} we present two different $\lambda/4$ AWP design types. The first design type aims at utilizing 3D printing of ceramic \ce{Al2O3} (alumina) for the fabrication of AWPs. The birefringence of 3D printed alumina is related to the printing process \cite{Ornik2021}. We rely on this reported birefringence and try to take advantage of this property to reduce the angular misalignment between individual plates forming the AWP. In the first section of this chapter results of two design optimizations for two different frequency ranges are presented. Furthermore, the two theoretical results are compared with the design reported by Masson et. al. \cite{Masson2006} and different parameters which can be used to describe the properties of the waveplates are discussed. Since the individual fabricated waveplates did not exhibit the expected birefringent values, the AWP was not assembled and experimentally characterized. The characterization of refractive index and absorption coefficient values of individual plates are shown in the Appendix section \ref{sec:ceramic_characterization}.

In the second section of this chapter the second design type is presented. This third result is based on form birefringence (see subsection \ref{sec:form_birefringence}). The goal of this part was to fabricate a stratified structure using a conventional low-cost polymer based 3D printer. Furthermore, an additional goal was to utilize 3D printing to avoid excessive angular misalignment of individual plates forming the AWP. However, the printing resolution of such low-cost 3D printers is limited and therefore this AWP was designed to work in the low THz or sub-THz range (\SIrange{75}{110}{\giga \hertz}). In this section first the design and expected AWP properties are presented. Then the experimental characterization of the AWP is shown and compared to the theoretical expectations. Finally the sources for the deviation between the theory and experiment are identified and discussed.

\section{Composite 3D printed ceramic AWP}
In a publication by Ornik et al. it was shown that \ce{Al2O3} as a 3D printed ceramic exhibited birefringent properties. A birefringence of approximately $0.05$ was reported for the range from \SIrange{0.25}{2.5}{\tera \hertz}. This value is fairly low compared to the birefringence of sapphire which is the crystalline form of \ce{Al2O3} with a birefringence of approximately $0.32$ \cite{Wu2020}. It was further shown that the direction of the slow axis was parallel to the printing direction. The sample is printed in two steps. In the first step, layer by layer is exposed to UV light which causes the polymerization in the slurry. This way the so-called green body is formed, which is then placed in an electric furnace exposed to \SI{1700}{\celsius} for the debinding and sintering processes. During these processes the sample shrinks for about \SI{21}{\volpercent} mainly due to the polymer burn-out. 

Two cylindric samples were printed and characterized and the corresponding data are shown in figure \ref{fig:ri_abs} \cite{Ornik2021}. (a) shows the measured refractive index along the slow and fast direction for the two samples and the birefringence as a function of the frequency, while (b) shows the measured absorption coefficient for the two samples and directions as a function of the frequency. We see that within the error range the birefringence of the two samples is equal. Therefore, either set of sample parameters works for setting up the objective function, we choose to use the measurement result of sample 1. 

\begin{figure}[ht]
    \begin{subfigure}[b]{.5\linewidth}
    \caption{}\label{}
    \centering\includegraphics[scale=0.73]{images/results/plots/ceramic/ri_bf_a.pdf}
    \end{subfigure}%
    \hspace{3em}
    \begin{subfigure}[b]{.5\linewidth}
    \caption{}\label{}
    \centering\includegraphics[scale=0.73]{images/results/plots/ceramic/ri_bf_b.pdf}
    \end{subfigure}
    \caption{(a) The refractive index and birefringence as a function of frequency for the two alumina samples; sample 1 and 2. For both samples the refractive index was measured for the perpendicular slow and fast axes. Considering the error margins both samples have an equal birefringence. (b) The absorption coefficient, again for the slow and fast axis of the samples. Data source: \cite{Ornik2021}}\label{fig:ri_abs}
\end{figure}

Stacking a number of these ceramic discs or plates we can in theory construct an AWP similar to the TAQ by Masson. This technique has the advantages that since the plates are 3D printed we can ideally control the thickness of the individual plates and also simplify the angular alignment. To do so, we considered printing cylindrical shaped samples with a flat side for easier orientation when stacking the individual plates. However, since the samples were printed standing directly on the platform without any support, the side of the sample in contact with the platform had to be also flat to ensure proper contact with the building platform. In figure \ref{fig:ceramic_stack} three plates of the described shape are shown and the idea of stacking the single plates at an orientation according to the optimized design.
%This ceramic AWP can then serve as a comparison to the TAQ and therefore also partly as a verification of the objective function at least for the $\lambda/4$ AWP type. In this case the objective functions $L_{\lambda/4}$ and $L_{\lambda/2}$ only depend on the set of angles and thicknesses. In other words the birefringence does not change at each iteration and is the same for each individual plate. A schematic of an AWP of this type consisting of three ceramic discs is shown in figure \ref{fig:ceramic_stack}. Each disc has two flat sides on the edge where one is used for alignment and the other is the printing base. The alignment flat is placed keeping in mind that the slow axis is along the printing direction. In reality the discs are stacked so that ideally there should be no air gaps in between each waveplate.

\begin{figure}[ht]
    \centering
    \includegraphics[scale=3.00]{images/results/tikz_wpstack_ceramic.pdf}
    \caption{Exploded view of a ceramic AWP with $n=3$. In general each waveplate has a thickness $d_i$ and an orientation angle or azimuth $\alpha_i$. The individual plates are printed with two flat sides; one is used to align the discs and the other serves as support during the printing process. The red arrow points to the supportive base which is therefore the printing direction and hence the slow direction.}
    \label{fig:ceramic_stack}
\end{figure}

%In order to reproduce the TAQ the birefringence is required which can obtained from the reported values in \cite{DGrischkowsky1990}, this is also the source used in \cite{Masson2006}. To this extend the Sellmeier equation allows us to interpolate the values in the reported data range which is from \SIrange{0.2}{2.0}{\tera \hertz}, details of the procedure can be found in the appendix section \ref{sec:sellmeier}. Using the thicknesses and angles of the TAQ given in table \ref{tab:masson_result} as well as the birefringence of quartz we can then calculate $L_{\lambda/4}(\nu)$ for the TAQ to use it for comparison. Furthermore, we optimized $L_{\lambda/4}$ for $n=6$ in the range from \SIrange{0.25}{1.50}{\tera \hertz}. The design parameters obtained from this optimization are shown in table \ref{tab:res_cl4} (Result 1). Additionally, the individual terms of $L_{\lambda/4}(\nu)$ of this result as well as those for the TAQ are plotted in figure \ref{fig:loss_function_cl4} as a function of frequency. We see that the magnitude of $L_{\lambda/4}$ for these two results is similar. Assuming that $L_{M}$ is a correct measure of the frequency dependent performance of the waveplate this means that $L_{\lambda/4}$ is also a suitable measure of the quality of the result. At around \SI{1.5}{\tera \hertz} there is a clear cut-off for Result 1 after which $L_{\lambda/4}$ increases steeply. This steep increase starts at around \SI{1.6}{\tera \hertz} for the TAQ. Defining the bandwidth as the ratio between the upper and lower frequency we get a value of  $\frac{\SI{1.5}{\tera \hertz}}{\SI{0.3}{\tera \hertz}}=5$. 

We optimized two designs which we call Result 1 and 2 for the frequency ranges \SIrange[range-phrase=-, range-units=single]{0.25}{1.50}{\tera \hertz} and \SIrange[range-phrase=-, range-units=single]{0.50}{2.25}{\tera \hertz}, respectively. In the work by Masson et al. the bandwidth of an AWP is defined as the ratio of the highest and lowest frequency in the range \cite{Masson2006}. Adopting this definition we have a bandwidth of around 5 and 6 for Result 1 and 2, respectively. The design parameters obtained from the optimization of the two results are listed in table \ref{tab:res_cl4}. 

\begin{table}[ht]
    \centering
    \includegraphics[scale=1.0]{images/results/ceramic_result_table.pdf}
    \caption{Design parameters for Result 1 and 2. Both results are obtained through the optimization of $L_{\lambda/4}$ for $n=6$. In the case of Result 1 the frequency range for the optimization was limited to \SIrange[range-phrase=-, range-units=single]{0.25}{1.50}{\tera \hertz} while for Result 2 the range was set to \SIrange[range-phrase=-, range-units=single]{0.50}{2.25}{\tera \hertz}.}
    \label{tab:res_cl4}
\end{table}

%\begin{figure}[H]
%    \centering
%    \includegraphics[scale=0.78]{images/results/plots/ceramic/loss_function.pdf}
%    \caption{The left subfigure shows the value of $L_{\lambda/4}$ as a function of frequency for the two results compared to the TAQ published in \cite{Masson2006}. We see that in the range in which the optimization took place the magnitude of $L_{\lambda/4}$ for the three results is fairly similar. The right subfigure shows the values of $L_M$ and $L_{\lambda/4}$ for the TAQ design with respect to the frequency. The largest difference between the two is around \SI{0.5}{\tera \hertz} and \SI{0.7}{\tera \hertz}, otherwise the trend of the two objective functions is relatively similar.}
%    \label{fig:loss_function_cl4}
%\end{figure}

The most notable difference we see between the two results is their total thickness. The total thickness of Result 1 is around \SI{32}{\milli \meter} while for Result 2 it is only around \SI{26}{\milli \meter}. However both results are fairly thick compared to the sapphire waveplate by Wu et al. \cite{Wu2020}. This is likely because similar to quartz, the birefringence of the ceramic plates is relatively low. This in turn means that for the waveplates to have an effect in the lower end of the spectrum the individual plates must be fairly thick, due to the proportionality between the phase shift and the frequency. Additionally, since the lower end of the optimized frequency range is slightly higher for Result 2 compared to Result 1 its total thickness is also slightly lower.
 
 We can calculate the thickness required to cause a quarter wave shift at the minimum of the frequency range for a linearly polarized input with an azimuth of \SI{45}{\degree}. In other words the thickness of a ceramic zero order quarter waveplate at the minimum frequency, this is given by equation $\ref{eq:thickness_quarter_waveplate}$. In this case the minimum frequency is \SI{0.25}{\tera \hertz}, at this frequency the birefringence is $\Delta n = 0.047$. The required thickness is therefore approximately $d_{\lambda/4}=\frac{\SI{1200}{\micro \meter}}{4\cdot0.047}=\SI{6.38}{\milli \meter}$. Comparing this value to the total thicknesses of the two results we see that it is several times smaller and almost \SI{4}{\milli \meter} thinner than the thickest plate. If we keep in mind that a probabilistic algorithm is part of the optimization method, then this could indicate that the obtained results are not the thinnest possible configurations for the given frequency ranges and material. On the other hand however, we have to keep in mind that $d_{\lambda/4}$ is for an angle of \SI{45}{\degree} while the six plates of the AWP are in general at different angles reducing the phase shift caused by each plate. It is therefore questionable whether we can actually directly compare $d_{\lambda/4}$ to the total thickness.

Furthermore, if we consider two results where the total thickness of the latter is larger, then the magnitude or intensity of the corresponding output state will in general be lower compared to the output of the first result. However, $L_{\lambda/4}(\nu)$ is insensitive to isotropic changes in the absorption. So if the ellipticity and orientation of the polarization ellipses corresponding to the two results are equal for all frequencies then $L_{\lambda/4}$ will attain the same value in both cases. In other words, $L_{\lambda/4}(\nu)$ is the same for two states $\bm{\mathcal{E}}$ and $p\bm{\mathcal{E}}$, $0<p<1$ for any frequency. The intensity coefficients $p_1^2$ and $p_2^2$ for the equivalent elliptic diattenuator Jones matrix describing the result with a larger total thickness will both be scaled by $p^2$. Since in the definition of $L_{\lambda/4}$ the elements of the Jones matrix enter as a ratio, the factor $p^2$ then subsequently cancels out and both results will attain the same value of $L_{\lambda/4}$. This in turn means that there is no direct penalty for increasing the thicknesses other than increasing the effect of dichroism.

To gain an estimate of the performance of the objective function and the optimization routine we compare the values of the objective function $L_{\lambda/4}$ with respect to frequency proposed in this work to the values of the one used by Masson et al.; $L_{M}$ given in equation \ref{eq:mass_loss} \cite{Masson2006}. We do this using the design parameters of the TAQ given in table \ref{tab:masson_result} as well as the optical parameters of quartz published in \cite{DGrischkowsky1990} which is also the source used by Masson et al. \cite{Masson2006}. However, since we do not have direct access to the measured data points we use the Sellmeier equation to interpolate the optical parameters of quartz in the reported data range which is from \SIrange{0.2}{2.0}{\tera \hertz}, details of the procedure can be found in the appendix section \ref{sec:sellmeier}. It is worth emphasizing that in contrast to $L_{\lambda/4}$ the dichroism is unaccounted for in the calculation of $L_{M}$. The comparison is shown in figure \ref{fig:loss_function_cl4_b}, i.e. the calculated values of the two objective functions $L_{\lambda/4}$ and $L_{M}$ for the design parameters of the TAQ given in table \ref{tab:masson_result}. We see that the shape of the two functions match fairly well except for some lower frequencies and especially at around \SI{0.55}{\tera \hertz} we see a larger deviation. 

\begin{figure}[H]
    \centering
    \includegraphics[scale=0.78]{images/results/plots/ceramic/loss_function_b.pdf}
    \caption{Values of $L_M$ and $L_{\lambda/4}$ for the TAQ design with respect to the frequency. The largest difference between the two is around \SI{0.5}{\tera \hertz} and \SI{0.7}{\tera \hertz}, otherwise the trend of the two objective functions is relatively similar.}
    \label{fig:loss_function_cl4_b}
\end{figure}

To address the question whether Result 1 is the global minimum or if there is a better suited one we plotted the minimum value of $\sum_{\nu}L_{\lambda/4}(\nu)$ as function of the current iteration step in figure  \ref{fig:cl4_convergence}. We see that the algorithm converges fairly fast and we only noticed major changes for approximately the first 25 iterations. This indicates that finding a solution with $n=6$ which performs significantly better in the range of \SIrange{0.25}{1.50}{\tera \hertz} is unlikely using the current algorithm and objective function.

\begin{figure}[H]
    \centering
    \includegraphics[scale=0.7]{images/results/plots/ceramic/convergence.pdf}
    \caption{Minimum value of $L_{\lambda/4}$ as a function of the total iteration count for the frequency range \SIrange{0.25}{1.5}{\tera \hertz} and a plate count of six. This shows that for the chosen settings the algorithm convergences rather fast.}
    \label{fig:cl4_convergence}
\end{figure}

This means that we can conclude $L_{\lambda/4}$ is a fairly good estimate of the performance of the AWP if we assume $L_{M}$ is as well. Additionally, the optimization routine converges to a local minimum of the objective function sufficiently fast. Therefore, the design procedure proposed in this work performs similarly to the one described in the work by Masson et al. with the addition that we consider dichroism as well. 

In the following we compare Result 1 and 2 based on the design parameters given in table \ref{tab:res_cl4} and 3D printed alumina to the quartz TAQ for which the parameters are listed in table \ref{tab:masson_result}. Both materials have a similar birefringence but the absorption and dichroism of alumina is significantly higher. We can therefore expect a similar performance in terms of polarization conversion between these two types but we also expect that the alumina AWPs cause higher losses. For this purpose the individual terms of $L_{\lambda/4}$ for the three designs are shown in figure \ref{fig:loss_function_cl4_a}, i.e. $L_{\lambda/4}(\nu)$ for each frequency, where the frequency range is bound by the available material data. We see that the magnitude of the values is similar in the optimized frequency range.

%The design parameters of Result 2 are given in the lower half of table \ref{tab:res_cl4} and the dotted line in the left subfigure of figure \ref{fig:loss_function_cl4} shows the individual terms of $L_{\lambda/4}$ for each frequency bound by the available material data range. The right subfigure shows a comparison between $L_M(\nu)$ and $L_{\lambda/4}(\nu)$ for the TAQ. We see that the shape of the two functions match fairly well except for the lower frequencies and especially at around \SI{0.55}{\tera \hertz} we see a larger deviation. 

\begin{figure}[H]
    \centering
    \includegraphics[scale=0.78]{images/results/plots/ceramic/loss_function_a.pdf}
    \caption{The value of $L_{\lambda/4}$ as a function of frequency for the two results compared to the TAQ published in \cite{Masson2006}. We see that in the range in which the optimization took place the magnitude of $L_{\lambda/4}$ for the three results is fairly similar.}
    \label{fig:loss_function_cl4_a}
\end{figure}

%Although, if we shift the frequency range of the optimization to \SIrange[range-phrase=-, range-units=single]{0.5}{2.25}{\tera \hertz} then we obtain a second result (Result 2) for which the bandwidth is almost five as well $\left(\frac{\SI{2.25}{\tera \hertz}}{\SI{0.5}{\tera \hertz}}=4.5\right)$. With that the total thickness of the AWP can be reduced to \SI{26}{\milli \meter} compared to the \SI{32}{\milli \meter} of Result 1. 

%The design parameters of Result 2 are given in the lower half of table \ref{tab:res_cl4} and the dotted line in the left subfigure of figure \ref{fig:loss_function_cl4} shows the individual terms of $L_{\lambda/4}$ for each frequency bound by the available material data range. The right subfigure shows a comparison between $L_M(\nu)$ and $L_{\lambda/4}(\nu)$ for the TAQ. We see that the shape of the two functions match fairly well except for the lower frequencies and especially at around \SI{0.55}{\tera \hertz} we see a larger deviation. 

In addition to considering the performance of the optimization procedure and objective function we can also compare the theoretical performance of the designs in terms of their polarization conversion effectiveness. For this purpose we discuss different properties of the final polarization state as a function of the frequency, where we assume initially HLP THz radiation if not stated otherwise.

Figure \ref{fig:cl4_alpha} shows $\alpha$ for the output state for Result 1 and the TAQ. Since $\alpha$ is a measure of the ellipticity of the output state it shows the difference between a circular shape and the shape of the output. For a circle $\alpha$ is $\arctan(1)=\SI{45}{\degree}$ which is therefore the target value. If we compare $\alpha$ for the TAQ and at the frequency \SI{0.55}{\tera \hertz} to $L_{\lambda/4}$ at the same frequency, we see that the large deviation from the optimum is correctly indicated by $L_{\lambda/4}$ while $L_M$ shows a minimum and thereby fails to recognize the deviation. We see the same situation at around \SI{0.75}{\tera \hertz}. 

\begin{figure}[H]
    \centering
    \includegraphics[scale=0.73]{images/results/plots/ceramic/alpha.pdf}
    \caption{$\alpha$ of the output state as a function of the frequency given a HLP input for the TAQ and Result 1. Ideally $\alpha$ should be equal to \SI{45}{\degree} independent of frequency. We see that the deviation of Result 1 from the optimum is distributed more evenly on the frequency range, while for the TAQ the deviation is higher at lower frequencies.}
    \label{fig:cl4_alpha}
\end{figure}

Comparing the values of $\alpha$ to figure \ref{fig:loss_function_cl4_b} and considering that the values of $L_M$ and $L_{\lambda/4}$ shown in figure \ref{fig:loss_function_cl4_b} are calculated from the same Jones matrix, we can also conclude that at least for some frequencies $L_{\lambda/4}$ is better at identifying deviations of the output from the optimal CP state. This could be due to the fact that the value of $L_{\lambda/4}$ is directly a measure of the difference between the state of the output and the RCP state.

%Considering that $L_M$ and $L_{\lambda/4}$ are calculated from the same Jones matrix, which is constructed from the parameters of the TAQ and crystalline quartz, figure \ref{fig:cl4_alpha} combined with figure \ref{fig:loss_function_cl4_b} shows that at least for some frequencies $L_{\lambda/4}$ is better at identifying deviations of the output from the optimal CP state. This could be due to the fact that the value of $L_{\lambda/4}$ is directly a measure of the difference between the state of the output and the RCP state.

\begin{figure}[H]
    \begin{subfigure}[b]{.5\linewidth}
    \caption{}\label{}
    \centering\includegraphics[scale=0.7]{images/results/plots/ceramic/pe_cl4_lp_a.pdf}
    \end{subfigure}%
    \begin{subfigure}[b]{.5\linewidth}
    \caption{}\label{}
    \centering\includegraphics[scale=0.7]{images/results/plots/ceramic/pe_cl4_lp_b.pdf}
    \end{subfigure}
    \caption{Polarization ellipse representation of the output states for a HLP input at seven different frequencies. (a) The output of the TAQ. (b) The output of the ceramic AWP with the parameters from Result 1.}
    \label{fig:cl4_pe_lp}
\end{figure}

Figure \ref{fig:cl4_pe_lp} shows the polarization ellipses of the normalized states at seven different frequencies. The ellipses for the output of the TAQ is shown in (a) while (b) shows those for Result 1. The arrows indicate the handedness of the polarization. We see that as expected we get LCP states in the frequency range in which the optimization took place. Outside of that range the states are significantly more elliptical. We also see a slightly better result for the intermediate frequencies compared to the lowest at \SI{0.3}{\tera \hertz}. 

% TODO note starts to work as lambda half at higher freq. -> 90deg azimuth (nah)

\begin{figure}[H]
    \begin{subfigure}[b]{.5\linewidth}
    \caption{}\label{}
    \centering\includegraphics[scale=0.7]{images/results/plots/ceramic/pe_cl4_cp_a.pdf}
    \end{subfigure}%
    \begin{subfigure}[b]{.5\linewidth}
    \caption{}\label{}
    \centering\includegraphics[scale=0.7]{images/results/plots/ceramic/pe_cl4_cp_b.pdf}
    \end{subfigure}
    \caption{Polarization ellipse representation of the output states for a LCP input at seven different frequencies. (a) The output of the TAQ. (b) The ceramic AWP with the parameters from Result 1.}
    \label{fig:cl4_pe_cp}
\end{figure}

Figure \ref{fig:cl4_pe_cp} is similar to figure \ref{fig:cl4_pe_lp} except that the input in this case is LCP. In theory a quarter waveplate should produce a linearly polarized output for a circularly polarized input. We see that this is the case for the TAQ up to at least \SI{1.50}{\tera \hertz}, after which the state increasingly deviates from the linear ellipse shape. Interestingly the ellipses of Result 1 have a lower ellipticity compared to those of the TAQ, which indicates that the TAQ is better suited for turning a circularly polarized input into a linear polarized output. The reason for this could be that we only consider the circularity of the output state in the objective function used to obtain Result 1, while for the TAQ the resulting phase shift was directly optimized. This could therefore indicate that it is not sufficient to only consider the circularity of the output state in the objective function and a measure of the linearity or another appropriate property should be included as well. Although, on the other hand the relatively large deviation that can be seen at \SI{1.50}{\tera \hertz} can also be recognized in the plot of the objective function in figure \ref{fig:loss_function_cl4_a}. Which means that the deviation is recognized by the objective function but a minimum was not found that minimizes $L_{\lambda/4}$ at this frequency. This could therefore indicate that it is more difficult to minimize $L_{\lambda/4}$ compared to minimizing $L_M$.

The circular and linear polarization degrees $V_c$ and $V_l$ for a linearly polarized input state is shown in figure \ref{fig:cl4_pol_deg} as a function of frequency for Result 1 and the TAQ. Subfigures (a) and (c) show the full frequency range of $V_c$ and $V_l$, respectively, while subfigures (b) and (d) show the first \SI{1.6}{\tera \hertz}. For clarity $V_c$ is offset by one in (b) so that values of $V_c$ closer to zero are in fact closer to negative one. We see that as expected $V_c$ is around $-1$ in the optimization range for both the TAQ and Result 1 which indicates that the light is almost purely LCP. Interestingly, outside of the optimization frequency range the light switches to be primarily RCP. 

\begin{figure}[H]
\centering
\subcaptionbox{\label{fig:cl4_pol_deg_a}}
    {\hspace*{-2em}\includegraphics[width=0.41\linewidth]{images/results/plots/ceramic/polDeg/degCirc.pdf}}
\qquad
\subcaptionbox{\label{fig:cl4_pol_deg_b}}
    {\hspace*{-2em}\includegraphics[width=0.45\linewidth]{images/results/plots/ceramic/polDeg/degCircZoom.pdf}}
\subcaptionbox{\label{fig:cl4_pol_deg_c}}
    {\hspace*{-2em}\includegraphics[width=0.41\linewidth]{images/results/plots/ceramic/polDeg/degLin.pdf}}
\qquad
\subcaptionbox{\label{fig:cl4_pol_deg_d}}
    {\hspace*{-2em}\includegraphics[width=0.44\linewidth]{images/results/plots/ceramic/polDeg/degLinZoom.pdf}}
\caption{Subfigures (a) and (c) in the left column show the circular and linear polarization degrees, respectively, as a function of the frequency for Result 1 and the TAQ. Subfigures (b) and (d) in the right column also show the polarization degrees but for a smaller frequency range. The circular polarization degree in (b) is offset by one.}
\label{fig:cl4_pol_deg}
\end{figure}

Furthermore, subfigure (b) shows that $V_c$ for Result 1 varies less for frequencies below \SI{0.6}{\tera \hertz} and more for higher frequencies compared to the TAQ and varies less overall. This could be due to the difference between the refractive indices used for quartz during optimization and the values used for the calculation of $V_c$, since these deviations are fairly small. Subfigure (c) shows that the linear polarization degree increases to one and decreases again while the light switches to RCP. Subfigure (d) shows a similar trend as (b) in the sense that Result 1 is closer to zero at lower frequencies compared to the TAQ. We also see that $V_l$ is not zero, the light is therefore not perfectly CP but also partly LP, which is difficult to see in the polarization ellipse representation. 

Using the values for the absorption coefficient of quartz reported in \cite{DGrischkowsky1990} we can calculate the intensity of the light after passing through the \SI{31.45}{\milli \meter} thick TAQ. We can carry out the same calculation for the ceramic AWP designs; Result 1 and the slightly thinner Result 2 which have a total thickness of \SI{31.94}{\milli \meter} and \SI{26.24}{\milli \meter}, respectively. The transmission as a function of frequency for these three designs is shown in figure \ref{fig:cl4_intensity}. We clearly see that even though the three designs have an almost equal total thickness the insertion loss of the ceramic AWPs is several orders of magnitude higher compared to the quartz TAQ, which limits the practicality of ceramic AWPs.
A possibility to reduce the significance of this could be to decrease the bandwidth of the ceramic AWP and thereby also decrease the required total thickness. Another possibility to partly circumvent this problem could be to simply only use the lower frequency range where the absorption is lower. Additionally, the intensity of the lower frequency components is higher, at least in the case of the THz-TDS systems used in this work. Since apparently the birefringence of 3D printed alumina depends on the printing process it might be possible to enhance the birefringence by modifying the printing process, which in turn means that the total thickness can be reduced, this is still work in progress.

\begin{figure}[ht]
    \centering
    \includegraphics[scale=0.75]{images/results/plots/ceramic/intensity.pdf}
    \caption{Transmission as a function of frequency for Result 1 and 2 compared to the quartz TAQ. The insertion losses of Result 1 and 2 are significantly higher compared to the TAQ, which is due to the higher absorption of the ceramic plates compared to the quartz plates.}
    \label{fig:cl4_intensity}
\end{figure}

Figure \ref{fig:cl4_params} shows the two sets of parameterization angles ($\alpha$, $\delta$) and ($\psi$, $\chi$) for (a) the TAQ and (b) Result 1 as a function of frequency. We see that the values of $\alpha$ and $\delta$ are around \SI{45}{\degree} and \SI{270}{\degree}, respectively, for both Result 1 and the TAQ indicating LCP. Again, similar to what the polarization degree showed we see a transition to RCP outside of the optimized frequency range, where in the case of Result 1 the transition is sharper. For both the TAQ and Result 1 the transition happens via a decrease of $\delta$. Since the phase shift is proportional to the birefringence times the path length and increases for increasing frequency it could indicate that the birefringence is negative. Furthermore, since $\delta = \delta_y - \delta_x$ and $\delta$ decreases for frequencies above approximately \SI{1.6}{\tera \hertz}, it shows that the phase of the x-component advances faster which in turn could indicate that the fast axis is partly along the x-axis, where the x-axis is defined as the axis along which the incident light is polarized. Additionally, we see a bump in $\alpha$ around \SI{1.75}{\tera \hertz}. This is likely due to the same reason that causes the sharper decrease of $\delta$ for Result 1 compared to the TAQ, since $\delta$ and $\alpha$ are related via $\sin(2\chi)=\sin(2\alpha)\sin(\delta)$ and the trend of the ellipticity $\chi$ is similar in both cases. We further see that the azimuth $\psi$ changes rapidly with frequency. This means that if we were to use two identical AWPs of this type in series to create a $\lambda/2$ AWP it likely would not have the desired effect, since the output state is highly dependent on the azimuth of the input. As a reminder, the azimuth of the input state for all frequencies and in all cases shown so far is zero (HLP).

\begin{figure}[ht]
    \begin{subfigure}[b]{.5\linewidth}
    \caption{}\label{}
    \centering\includegraphics[scale=0.73]{images/results/plots/ceramic/params_dotted_a.pdf}
    \end{subfigure}%
    \begin{subfigure}[b]{.5\linewidth}
    \caption{}\label{}
    \centering\includegraphics[scale=0.73]{images/results/plots/ceramic/params_dotted_b.pdf}
    \end{subfigure}
    \caption{The two sets of parameterization angles ($\alpha$, $\delta$) and ($\psi$, $\chi$) as a function of frequency. Subfigures (a) and (b) show the angles for the TAQ and Result 1, respectively. The values of $\alpha$ and $\delta$ are as expected for LCP as well as after the transition to RCP for both designs. The variation of the azimuth $\psi$ makes it impossible to place two of these designs in series to form a $\lambda/2$ AWP. As expected, for LCP and RCP frequency components $\chi$ attains a value of plus or minus \SI{45}{\degree}, respectively.}
    \label{fig:cl4_params}
\end{figure}

Finally, to finish the theoretical predictions of the performance we will discuss the eigenstates and eigenvalues of Result 1 and use the TAQ for comparison. The calculated retardance as well as the diattenuation for both the TAQ and Result 1 is shown in subfigures \ref{fig:cl4_ret_diat} (a) and \ref{fig:cl4_ret_diat} (b), respectively. The retardance and the diattenuation of an optical element is obtained directly from the Jones matrix describing the element and is therefore independent of the input state. Additionally, as a reminder the retardance and diattenuation are, respectively, defined as the phase and relative intensity transmittance difference between the eigenpolarizations after passing the waveplate. It can be shown that this phase shift and transmittance difference are the maximum possible of all states \cite{Lu1994}. 

We see that the diattenuation is increasing with frequency and correlating with the dichroism. Since the dichroism of quartz is lower compared to alumina we also find that the diattenuation of the TAQ is smaller compared to the ceramic waveplate. Even though the diattenuation is several orders of magnitude higher for Result 1 we still theoretically see a similar performance compared to the TAQ. This means that either the influence of dichroism on the performance is less significant or the model accounts for it correctly. Additionally, the spikes around \SI{0.5}{\tera \hertz} and \SI{0.9}{\tera \hertz} are the frequencies where the absorption is direction independent, which we also recognize in figure \ref{fig:ri_abs}. 

Furthermore, any state that is not an eigenstate will experience a smaller phase shift and compared to the retardance at a given frequency. This means that the retardance is an indicator of the order of the waveplate since it shows the maximum possible number or fractions of shifted waves, both results can therefore be considered as zero order waveplates since the retardance is less than $2\pi$ \cite{Samoylov2004}. 

\begin{figure}[H]
    \begin{subfigure}[b]{.5\linewidth}
    \caption{}\label{}
    \centering\includegraphics[scale=0.59]{images/results/plots/ceramic/ret_and_diat_a.pdf}
    \end{subfigure}%
    \begin{subfigure}[b]{.5\linewidth}
    \caption{}\label{}
    \centering\includegraphics[scale=0.59]{images/results/plots/ceramic/ret_and_diat_b.pdf}
    \end{subfigure}
    \caption{(a) Calculated retardance and (b) diattenuation for the TAQ and Result 1. The retardance stays below $2\pi$ indicating that both the TAQ and the waveplate from Result 1 are zero order waveplates. The diattenuation is higher for Result 1 due to the higher dichroism of alumina compared to quartz.}
    \label{fig:cl4_ret_diat}
\end{figure}

Another interesting property are the principal axes of the waveplate. In case of a single birefringent linear retarder the principal axes are simply the fast and slow axis of the birefringent material; in other words linear polarization states for which the polarization planes align with the principal axes of the material. However in general these axes are not always linear since they can be regarded as the fastest and slowest propagating states of the optical element. For any given optical element the effective principal axes are therefore simply the eigenstates of the respective Jones matrix. As mentioned earlier these are also the states that are retarded the most or in other words most effectively phase shifted relative to each other. 

We calculated the eccentricity and azimuth of the eigenstates for the TAQ and Result 1 which is shown in figures \ref{fig:cl4_eigenstate_eccentricity} and \ref{fig:cl4_eigenstate_azimuths}, respectively. We expect that the eccentricity is one and zero for linearly and circularly polarized states, respectively. It shows that the principle axes are elliptically polarized in the optimization frequency range for both the TAQ and Result 1, which are therefore both elliptical waveplates. This also shows that these composite achromatic waveplates in general are not linear waveplates even though for an individual chromatic waveplate this is the case.

\begin{figure}[H]
    \begin{subfigure}[b]{.5\linewidth}
    \caption{}\label{}
    \centering\includegraphics[scale=0.75]{images/results/plots/ceramic/eigenstate_eccentricity_a.pdf}
    \end{subfigure}%
    \begin{subfigure}[b]{.5\linewidth}
    \caption{}\label{}
    \centering\includegraphics[scale=0.75]{images/results/plots/ceramic/eigenstate_eccentricity_b.pdf}
    \end{subfigure}
    \caption{Eccentricity of the two eigenstates for both (a) the TAQ and (b) Result 1. It shows that both AWPs are examples of elliptical waveplates since the eccentricities in both cases are below one but above zero in the range in which the optimization took place.}
    \label{fig:cl4_eigenstate_eccentricity}
\end{figure}

% TODO Fix phasewraps in plots (done)
% TODO Fix fig 5.5 (done)
We see that the azimuth shown in figure \ref{fig:cl4_eigenstate_azimuths} of the two eigenstates and thereby the angle of the effective principle axes for the TAQ and Result 1 is slightly dispersive. Furthermore, we see that the axes do not align with any of the axes of the individual plates. This behavior of the principle axes can be a problem for certain applications such as in the measurement of the polarization of the light from stars. Since then the axial dispersion results in changes in the orientation of the reference frame used to describe the determined Stokes parameters according to the frequency range \cite{Clarke2004, Bailey2019}. Although, since the plates in this work are supposed to be used for conversation of a certain state into another it is not as detrimental. Even though the axes are not independent of wavelength it is worth noting that the axes are at least pairwise orthogonal.

\begin{figure}[H]
    \begin{subfigure}[b]{.5\linewidth}
    \caption{}\label{}
    \centering\includegraphics[scale=0.75]{images/results/plots/ceramic/eigenstate_azimuths_a.pdf}
    \end{subfigure}%
    \begin{subfigure}[b]{.5\linewidth}
    \caption{}\label{}
    \centering\includegraphics[scale=0.75]{images/results/plots/ceramic/eigenstate_azimuths_b.pdf}
    \end{subfigure}
    \caption{Azimuthal angle of the eigenstates for (a) the TAQ and (b) Result 1, which shows the effective orientation of the principle axes. We see that the axes in both cases are slightly dispersive.}
    \label{fig:cl4_eigenstate_azimuths}
\end{figure}

Since the input in the context of this work is linearly polarized we would ideally want the eigenstates to be linear. Furthermore, we get an interesting result in case of negligible dichroism. The Jones matrix $\hat{T}$ representing the composite waveplate can then be written as given in equation \ref{eq:equiv_matrix} or specifically as follows:
\begin{equation}
    \hat{T} = 
    \begin{pmatrix} 
    A & B \\
    -B^* & A^*
    \end{pmatrix}.
\end{equation}
If we then calculate the eigenstates we get the condition that they are linear if $\operatorname{Re}A=0$. In this case the retardation dephasing $\Delta_M$ given by equation \ref{eq:retardation_dephasing} coincides with the retardation defined as the absolute phase or angle difference between the two complex eigenvalues. Furthermore, in case of a half waveplate $A$ is purely imaginary \cite{McIntyre1968}. This means that in order to obtain a design for a half waveplate it is sufficient to find design parameters which minimize $\operatorname{Re}A$.

To verify the design concept experimentally a set consisting of five ceramic discs similar to the three sketched in figure \ref{fig:ceramic_stack} as well as five additional test samples were produced by a third party group. We characterized the newly printed samples but unfortunately they showed almost no birefringence, which therefore made them useless as waveplates. This rather unexpected result at first can be attributed to the fact that these samples were printed with the same printer type as those reported by Ornik et al. \cite{Ornik2021}, however by a different group and operator and using different settings. It seems that the lower sintering temperature could be the reason for this unexpected outcome. However, this needs to be properly verified in the future. The description of the samples and the results of the measurement are shown in the appendix section \ref{sec:ceramic_characterization}. A design concept we could verify experimentally is the one based on form birefringent waveplates which will be discussed in the following section.

\section{3D-printed, form-birefringent composite AWP}
The concept of form birefringent polymer waveplates is almost identical to the form birefringent silica waveplates described at the end of chapter \ref{ch:theory}. Specifically, instead of using SLE to create the form birefringent stratified structures we used a commercial 3D printer, in this case the Prusa i3 MK3S+. A 3D printer simply liquifies a polymer filament which is then deposited layer by layer and line by line at certain positions to create the structure given by the compiled G-code file. 3D printing is therefore an additive process while SLE is subtractive.

As a first step in the design process we chose a polymer or rather a polymer filament which would be used for the printing of the structures. Based on the measurements published in \cite{Busch2014} of seven different 3D printed polymer samples possibly suited for THz optics we selected high impact polystyrene (HIPS) and polypropylene (PP) for further characterization. These two polymers were shown to have a relatively low absorption coefficient at \SI{500}{\giga \hertz} which is a common property of most non-polar polymers \cite{Jordens2010, Castro-Camus2020}, low absorption is important since the final composite waveplate is rather thick. A \SI{2}{\milli \meter} thick rectangular sample of each material were printed and subsequently characterized using the GHz setup described in \ref{sec:GHz_setup}. The refractive index and absorption coefficient obtained from the measurement is shown in subfigures \ref{fig:HIPS_PP_ri} (a) and (b), respectively.
 
We see that the refractive index does not change much over the frequency range for both materials. The absorption coefficient is slightly lower for PP compared to HIPS, which would make it better suited. Unfortunately, using PP we were only able to print simple squares due to its high flexibility and thermal expansion. All subsequent structures and samples were therefore printed using the HIPS filament.

\begin{figure}[H]
    \begin{subfigure}[b]{.5\linewidth}
    \caption{}\label{}
    \centering\includegraphics[scale=0.7]{images/results/plots/polymer/HIPS_PP_ri_a.pdf}
    \end{subfigure}%
    \begin{subfigure}[b]{.5\linewidth}
    \caption{}\label{}
    \centering\includegraphics[scale=0.7]{images/results/plots/polymer/HIPS_PP_ri_b.pdf}
    \end{subfigure}
    \caption{(a) Measured refractive index and (b) absorption coefficient of the \SI{2}{\milli \meter} HIPS and PP samples. The tails at the frequency boundaries are due to system artifacts.}
    \label{fig:HIPS_PP_ri}
\end{figure}

The optimization was performed similar to the ceramic design. Although, since the individual waveplates of this design are based on form birefringent waveplates we also optimized the stratification dimensions $a$ and $b$. To reduce the number of dimensions of the objective function we used the same values of $a$ and $b$ for all waveplates in the design; therefore all individual plates had the same birefringence. However, in the context of this work the main drawback of 3D printing is that the minimum feature size roughly depends on the nozzle diameter of the 3D printer. In our case we were working with a nozzle diameter of \SI{400}{\micro \meter}. However, we could only achieve a minimum feature size of around \SI{700}{\micro \meter} with a minimum air gap width of roughly \SI{300}{\micro \meter}. This means that effectively optimizing the stratification dimensions had no influence on the performance of the final design. Additionally, the variation between the stratification dimensions of the actual print and the design was found to be insignificant. 
Therefore, with a spatial period of around \SI{1}{\milli \meter} and a wavelength of \SI{2.725}{\milli \meter} at \SI{110}{\giga \hertz} we could still obtain almost a $1:3$ ratio considering the condition given in equation \ref{eq:rytov_cond1} that the wavelength should at least be larger than the stratification period for the approximation of the form birefringence to be reasonable. Interestingly, in another work it was shown by comparing the values of the form birefringence of a periodically stratified medium to the result of a more accurate numerical theory that the approximation of the birefringence used in this work is valid even up to a $3:4$ wavelength-periodicity ratio \cite{Busch2016}.

For this design we used the full frequency range since the bandwidth of the GHz setup was fairly narrow, specifically $\frac{\SI{110}{\giga \hertz}}{\SI{75}{\giga \hertz}} \approx 1.5$. Furthermore, to reduce the runtime of the optimization we reduced the frequency resolution to one tenth of the GHz setup resolution since the number of 2x2 matrix multiplications for each objective function evaluation is given by $\frac{B(n-1)}{\Delta \nu}$, where $B$, $n$ and $\Delta \nu$ is the bandwidth, number of individual waveplates and frequency resolution, respectively. The objective function itself was similar to $L_{\lambda/4}$, although in equation \ref{eq:l4_loss_function} we changed $r$ to $r^{-1}$. This modification was added in order to investigate if as expected the handedness of the output would then also be inverted compared to the output of the other results based on the original objective function. We subsequently optimized the modified objective function multiple times for $n=1, ..., 5$ and found a design consisting of four waveplates to be more than sufficient considering the reduced frequency range. 

\begin{table}[ht]
    \centering
    \includegraphics[scale=1.0]{images/results/polymer_result_table.pdf}
    \caption{Design parameters for Result 3 obtained through the optimization of $L_{\lambda/4}$ for $n=4$. We used the full bandwidth of the GHz setup for the optimization.}
    \label{tab:res_hips_l4}
\end{table}

The design with the lowest objective function value was chosen as the result of the optimization runs. The design parameters of this design are shown in table \ref{tab:res_hips_l4} and photos of the resulting printed structure are shown in figure \ref{fig:sent2david}.

\begin{figure}[H]
    \centering
    \subcaptionbox{\label{fig:sent2david_1}}
        {\includegraphics[width=0.35\linewidth]{images/results/polymer/sent2david_1.jpg}}
    \subcaptionbox{\label{fig:sent2david_2}}
        {\includegraphics[width=0.35\linewidth]{images/results/polymer/sent2david_2.jpg}}
    \caption{3D printed composite AWP according to the design dimensions given in table \ref{tab:res_hips_l4}. The waveplate is printed in one piece leaving no gaps between the individual waveplates. (a) shows the backside and (b) the front where "In" and "Out" marks the side of the input and output, respectively.}
    \label{fig:sent2david}
\end{figure}

Based on the result of the optimization we can again calculate a number of predicted properties of the output and the waveplate. First of all the expected effective refractive indices, the absorption coefficients parallel ($n_p$) and perpendicular ($n_s$) to the stratification and the birefringence are shown in figure \ref{fig:effective_ri_bf_hips}. The values are calculated using $a=\SI{630}{\micro \meter}$, $b=\SI{520}{\micro \meter}$ and the measured refractive index of HIPS. We see that the expected birefringence is around $0.086$ with a relatively low dispersion although with a slight increase at lower frequencies. Since the effective refractive index is approximately an average of the refractive indices of HIPS and air we see an even lower absorption coefficient compared to the sample consisting of HIPS only. Additionally, the calculated dichroism of the stratified structure is decreasing with increasing frequency. Again, data at the frequency boundaries is not reliable due to system artifacts.

\begin{figure}[H]
    \begin{subfigure}[b]{.5\linewidth}
    \caption{}\label{}
    \centering\includegraphics[scale=0.7]{images/results/plots/polymer/effective_ri_bf_a.pdf}
    \end{subfigure}%
    \hspace{2em}
    \begin{subfigure}[b]{.5\linewidth}
    \caption{}\label{}
    \centering\includegraphics[scale=0.7]{images/results/plots/polymer/effective_ri_bf_b.pdf}
    \end{subfigure}
    \caption{(a) Calculated effective refractive indices and (b) absorption coefficient of stratified HIPS parallel ($n_p$) and perpendicular ($n_s$) to the stratification given $a=\SI{630}{\micro \meter}$, $b=\SI{520}{\micro \meter}$.}
    \label{fig:effective_ri_bf_hips}
\end{figure}

Figure \ref{fig:polymer_pe_lp} shows the polarization ellipse representation of the normalized output for a (a) HLP and (b) LCP input. We see that the output should be almost perfectly RCP as expected in case of a linear input. In case of a LCP input the output is linearly polarized and again at different azimuths depending on the frequency, similarly to the ceramic designs. 

\begin{figure}[H]
    \begin{subfigure}[b]{.5\linewidth}
    \caption{}\label{}
    \centering\includegraphics[scale=0.7]{images/results/plots/polymer/pe_lp_a.pdf}
    \end{subfigure}%
    \begin{subfigure}[b]{.5\linewidth}
    \caption{}\label{}
    \centering\includegraphics[scale=0.7]{images/results/plots/polymer/pe_lp_b.pdf}
    \end{subfigure}
    \caption{Polarization ellipses of the normalized output at different frequencies for a (a) HLP and (b) LCP input, respectively.}
    \label{fig:polymer_pe_lp}
\end{figure}

HIPS is almost non-absorbing compared to alumina which further means that the transmission losses shown in figure \ref{fig:polymer_intensity} are relatively low considering the fairly thick device thickness of almost \SI{3}{\centi \meter}. The biggest losses are around \SI{80}{\giga \hertz} where the transmission is around \SI{1.8}{\decibel}.

\begin{figure}[H]
    \centering
    \includegraphics[scale=0.7]{images/results/plots/polymer/intensity.pdf}
    \caption{Transmitted intensity as a function of frequency calculated based on the design given in table \ref{tab:res_hips_l4} and the measured absorption coefficient of HIPS.}
    \label{fig:polymer_intensity}
\end{figure}

The circular and linear polarization degrees $1-V_c$ and $V_l$ are shown in subfigures \ref{fig:polymer_pol_deg} (a) and (b), respectively. $V_c$ is approximately one as expected for RCP light and $V_l$ is in the order of a few percent. We also see that at the edges of the frequency range the waveplate should perform slightly worse compared to other frequencies.

\begin{figure}[H]
    \begin{subfigure}[b]{.5\linewidth}
    \caption{}\label{}
    \centering\includegraphics[scale=0.6]{images/results/plots/polymer/polDeg_a.pdf}
    \end{subfigure}%
    \begin{subfigure}[b]{.5\linewidth}
    \caption{}\label{}
    \centering\includegraphics[scale=0.6]{images/results/plots/polymer/polDeg_b.pdf}
    \end{subfigure}
    \caption{Calculated (a) circular and (b) linear polarization degree $V_c$ and $V_l$ as a function of frequency. $V_c$ is almost one which indicates RCP light and $V_l$ is close to zero.}
    \label{fig:polymer_pol_deg}
\end{figure}

The calculated retardance and diattenuation of the waveplate from Result 3 is shown in subfigures \ref{fig:polymer_ret_and_diat} (a) and (b), respectively. We again see that the waveplate can be considered to be zero order since the retardance is below $2\pi$. Additionally, we also see that the diattenuation again correlates with the dichroism. Furthermore, considering the inhomogeneity of the Jones matrix and the deviation of the two eigenstates from being mutually orthogonal which is shown in the two subfigures in figure \ref{fig:inhomogeneity_orthogonality} (a) and (b), respectively. We see that there is a direct correlation between all three properties; the dichroism, inhomogeneity and the deviation from orthogonality. The eigenstates are therefore orthogonal and the Jones matrix is homogeneous when the dichroism is zero otherwise there is a slight deviation of up to \SI{1.6}{\degree}. 

\begin{figure}[H]
    \begin{subfigure}[b]{.5\linewidth}
    \caption{}\label{}
    \centering\includegraphics[scale=0.55]{images/results/plots/polymer/ret_and_diat_a.pdf}
    \end{subfigure}%
    \begin{subfigure}[b]{.5\linewidth}
    \caption{}\label{}
    \centering\includegraphics[scale=0.55]{images/results/plots/polymer/ret_and_diat_b.pdf}
    \end{subfigure}
    \caption{(a) Calculated retardance and (b) diattenuation as a function of frequency based on the design dimensions from Result 3. Result 3 represents a zero order waveplate since the retardance is lower than $2\pi$ for all frequencies. Again we see a correlation between the diattenuation and the dichroism of the structure.}
    \label{fig:polymer_ret_and_diat}
\end{figure}

\begin{figure}[H]
    \begin{subfigure}[b]{.5\linewidth}
    \caption{}\label{}
    \centering\includegraphics[scale=0.7]{images/results/plots/polymer/inhomogeneity_orthogonality_a.pdf}
    \end{subfigure}%
    \begin{subfigure}[b]{.5\linewidth}
    \caption{}\label{}
    \centering\includegraphics[scale=0.7]{images/results/plots/polymer/inhomogeneity_orthogonality_b.pdf}
    \end{subfigure}
    \caption{(a) Calculated inhomogeneity $\mu$ of the Jones matrix representing the AWP of Result 3. (b) The deviation from mutual orthogonality of its eigenstates as a function of frequency.}
    \label{fig:inhomogeneity_orthogonality}
\end{figure}

Figure \ref{fig:polymer_eigenstate_params} shows the (a) azimuth and (b) eccentricity of the eigenstates based on Result 3. Similar to the ceramic design and the TAQ we see that the eigenstates are again slightly elliptical and their azimuths show a small frequency dependency.

\begin{figure}[H]
    \begin{subfigure}[b]{.5\linewidth}
    \caption{}\label{}
    \centering\includegraphics[scale=0.6]{images/results/plots/polymer/eigenstate_params_a.pdf}
    \end{subfigure}%
    \begin{subfigure}[b]{.5\linewidth}
    \caption{}\label{}
    \centering\includegraphics[scale=0.6]{images/results/plots/polymer/eigenstate_params_b.pdf}
    \end{subfigure}
    \caption{(a) Calculated azimuths of the eigenstates as well as their eccentricity where the latter is shown in (b)}
    \label{fig:polymer_eigenstate_params}
\end{figure}

We calculated the $(\alpha, \delta)$ and $(\psi, \chi)$ parameterizations of the polarization ellipses representing the output of Result 3 given a HLP input, the individual parameters are shown figure in \ref{fig:polymer_params_dotted}. $\alpha$ and $\delta$ are almost frequency independent as expected for an AWP since $\alpha$ is a measure of how much the ellipse resembles a perfect circle and $\delta$ is the phase shift between the field components which is expected to be equal to $\frac{\pi}{2}$ for RCP. We see that the ellipticity angle $\chi$ is almost constant as well and is approximately equal to the expected $\frac{\pi}{4}$. As mentioned earlier, in theory the azimuth is undefined for a perfectly circularly polarized state since it determines the orientation of the polarization ellipse. It should therefore not be relevant for the performance.

\begin{figure}[H]
    \centering
    \includegraphics[scale=0.65]{images/results/plots/polymer/params_dotted.pdf}
    \caption{Two parameterizations $(\alpha, \delta)$ and $(\psi, \chi)$ of the output polarization ellipse for a linearly horizontally polarized input. $\alpha$ and $\delta$ are around $\frac{\pi}{4}$ and $\frac{\pi}{2}$ independent of frequency as expected for RCP. The azimuth $\psi$ shows a frequency dependency while the ellipticity angle $\chi$ is frequency independent with a value of approximately $\frac{\pi}{4}$.}
    \label{fig:polymer_params_dotted}
\end{figure}

Before we discuss the characterization of the waveplate it is worth considering the Poincaré representation of the output of each individual waveplate. The two Poincaré spheres for the frequencies \SI{90}{\giga \hertz} and \SI{105}{\giga \hertz} given an initial HLP input are shown in subfigures \ref{fig:res3_poincare} (a) and \ref{fig:res3_poincare} (b), respectively. The lines connecting each state via the shortest path are supposed to represent the transformation of the state caused by the respective individual waveplate. We see that the incident linearly polarized state is gradually transformed into the RCP state. Interestingly, in \cite{Masson2006} they note that due to the retardance varying with frequency the path along the sphere changes but as expected for an AWP the end state remains the same. Although, in the present case we do not see this happening; the path varies little with frequency. 

\begin{figure}[H]
    \centering
    \subcaptionbox{\label{fig:res3_poincare_90ghz}}
        {\includegraphics[width=0.49\linewidth]{images/results/plots/polymer/poincare/Poincare_Res3_90GHz.png}}
    \subcaptionbox{\label{fig:res3_poincare_105ghz}}
        {\includegraphics[width=0.49\linewidth]{images/results/plots/polymer/poincare/Poincare_Res3_105GHz.png}}
    \caption{Poincaré sphere representation of the normalized input and output state of each individual waveplate of Result 3. at (a) \SI{90}{\giga \hertz} and (b) \SI{105}{\giga \hertz}. We see that the initial linear input $(1,1,0,0)$ is gradually transformed to the final circular output $(1,0,0,1)$ for both frequencies shown here.}
    \label{fig:res3_poincare}
\end{figure}

%\begin{itemize}
%    \item 1:3 ratio seems to be enough in other works too (done)
%    \item Add calculated eff. ref. ind. + abs. coeff. -> bf (done)
%    \item retardance to show zero order (done)
%    \item ellipse of output (done)
%    \item eigenstate azimuth + eccentricity (done)
%    \item pol degrees (done)
%    \item transmitted intensity (done)
%    \item parameters (done)
%    \item inhomogenity correlates with dichroism/diattenuation since it removes reversal property. Also correlates with deviation from orthogonality of eigenstates. (done)
%\end{itemize}

\subsection{Characterization of the fabricated AWP}

To characterize the waveplate the measurement technique described in \cite{Masson2006} was applied to the GHz setup. The measurement was performed using a single polarizer which was inserted into the beam path between the waveplate and detector. Originally in \cite{Masson2006} two polarizers were inserted although due to the strongly linearly polarized antennas one polarizer was considered to be sufficient for this setup. The measurement itself is performed by measuring the amplitude $S(\phi)$ of the transmitted signal at different angles $\phi$ of the inserted polarizer with respect to the reference frame. It is worth noting that for this measurement the phase of the signal is in fact not required. In general the Jones vector representing the state after the waveplate can be written as follows:
\begin{equation}
    \bm{\mathcal{E}}_{wp}= 
    \begin{pmatrix}
    a \\
    be^{i\delta}
    \end{pmatrix},
\end{equation}
where $a,b$ are the major and minor ellipse axes and $\delta$ is the phase difference between the field components. The final state $\bm{\mathcal{E}}_{f}$ representing the measured signal is then given by $\bm{\mathcal{E}}_{wp}$ followed by the Jones matrices representing the rotated and fixed polarizer. This means that we can calculate $\bm{\mathcal{E}}_{f}$ as follows:
\begin{equation}
    \bm{\mathcal{E}}_{f}= 
    \begin{pmatrix}
        1 & 0\\
        0 & 0
    \end{pmatrix}
    \begin{pmatrix}
        c_{\phi} & -s_{\phi}\\
        s_{\phi} & c_{\phi}
    \end{pmatrix}
    \begin{pmatrix}
        1 & 0\\
        0 & 0
    \end{pmatrix}
    \begin{pmatrix}
        c_{\phi} & s_{\phi}\\
        -s_{\phi} & c_{\phi}
    \end{pmatrix}
    \bm{\mathcal{E}}_{wp}.
\end{equation}
The measured amplitude or the square root of the measured intensity is then given by the following expression:
\begin{equation}
    S(\phi)=\sqrt{I} = \sqrt{\bm{\mathcal{E}}_{f} \bm{\mathcal{E}}_{f}^{\dagger}} = |c_{\phi}|\left[(ac_{\phi}+bs_{\phi}c_{\delta})^2 + (bs_{\phi}s_{\delta})^2\right]^{1/2}.
    \label{eq:ampl_fit_func}
\end{equation}
The measurement is performed in steps of \SI{10}{\degree} of $\phi$. This leads to 36 individual measurements which are subsequently normalized with respect to the measurement where the polarizer and polarization plane of the detector are parallel. For a given frequency we fit equation \ref{eq:ampl_fit_func} or specifically the three parameters $a,b$ and $\delta$ to the measured amplitude as a function of $\phi$. The result of the evaluation is shown in figure \ref{fig:meas_result}, where subfigure (a) shows the expected phase shift calculated based on the measured optical parameters of HIPS and the design dimensions given in table \ref{tab:res_hips_l4} compared to the phase shift evaluated from the obtained data. 

\begin{figure}[H]
    \begin{subfigure}[b]{.5\linewidth}
    \caption{}\label{}
    \centering\includegraphics[scale=0.65]{images/results/plots/polymer/measurement_result_a.pdf}
    \end{subfigure}%
    \begin{subfigure}[b]{.5\linewidth}
    \caption{}\label{}
    \centering\includegraphics[scale=0.65]{images/results/plots/polymer/measurement_result_b.pdf}
    \end{subfigure}
    \caption{(a) Expected and measured phase shift $\delta$. (b) Ratio of the major and minor axes $a$ and $b$, respectively. For an achromatic $\lambda/4$ waveplate we expect a phase shift of $\frac{\pi}{2}$ and a ratio of 1 of the ellipse axes independent of frequency. The ideal values are in both plots shown as a black dotted horizontal line and the dashed line represents a monochromatic waveplate with a birefringence of $0.08$ and a \SI{92.5}{\giga \hertz} design frequency.}
    \label{fig:meas_result}
\end{figure}

The subfigure (b) of figure \ref{fig:meas_result} shows the measured ratio of the major and minor axes $a$ and $b$, respectively, which again are compared to the expected theoretical ratio. Considering the results shown in figure \ref{fig:meas_result} we see large deviations between the expected and measured values. A major part of the subsequent work therefore consists of finding the cause of this deviation. We first consider that there could be an error in the evaluation of fitting equation \ref{eq:ampl_fit_func} to the measured amplitudes. Therefore, we first directly plot the measured signals at several different positions $\phi$ of the polarizer which is shown in figure \ref{fig:measured_amplitude}. We see a similar slope of the curves compared to the measured phase shift shown in \ref{fig:meas_result} (a), where the slope is highest for angles around $\SI{45}{\degree}+m\SI{90}{\degree}$ with integer $m$. Therefore, since already the measured signal shows a similar trend without performing any post-processing it is unlikely that the deviation is due to an error in the evaluation.

\begin{figure}[H]
    \centering
    \includegraphics[scale=0.63]{images/results/plots/polymer/measured_amplitude.pdf}
    \caption{Normalized measured amplitude for 10 different angles of the polarizer.}
    \label{fig:measured_amplitude}
\end{figure}

We further considered the impact of differences between the dimensions of the actual printed waveplate and the design by theoretically evaluating the effect of random variation of waveplate thickness on the phase shift. The total thickness of the waveplate was measured to be \SI{27}{\milli \meter}, which is less than \SI{1}{\percent} deviation from the designed thickness. To be on the safe side, we decided to add a $\pm \SI{5}{\percent}$ random error to the thickness of each individual waveplate. Then we repeated this random thickness procedure 50 times and for each repetition evaluated the phase difference. Performing this procedure we produced 50 designs with random thickness variations. The result of this is shown in figure \ref{fig:delta_width_err} where the blue curve shows the phase shift of one of the randomly picked designs and the red curve shows the phase shift obtained from the measurement. The green and blue dashed curves show the phase shift of the designs with the minimum and maximum phase shift for all frequencies, respectively, i.e. the outliers. The horizontal dashed lines represent a $\pm \SI{3}{\percent}$ deviation from $\frac{\pi}{2}$. We see that thickness variations are unlikely to have caused the large deviation between the expected and measured result.

\begin{figure}[H]
    \centering
    \includegraphics[scale=0.63]{images/results/plots/polymer/dimension_errors/delta_width_error_a.pdf}
    \caption{Random thickness error of maximum \SI{\pm 5}{\percent} added to each individual waveplate. Green and blue dashed curves show the designs with the minimum and maximum phase shift of all designs, respectively. A lower resolution is used to speed up the calculation. The dashed lines show a $\pm \SI{3}{\percent}$ deviation from $\frac{\pi}{2}$ of the phase shift.}
    \label{fig:delta_width_err}
\end{figure}

%\begin{figure}[H]
%    \begin{subfigure}[b]{.5\linewidth}
%    \centering\includegraphics[scale=0.63]{images/results/plots/polymer/dimension_errors/delta_width_error_a.pdf}
%    \caption{}\label{}
%    \end{subfigure}%
%    \begin{subfigure}[b]{.5\linewidth}
%    \centering\includegraphics[scale=0.63]{images/results/plots/polymer/dimension_errors/delta_width_error_b.pdf}
%    \caption{}\label{}
%    \end{subfigure}
%    \caption{Random thickness error of maximum \SI{\pm 5}{\percent} added to each individual waveplate. (b) \SI{5}{\percent} reduction of the thickness of each waveplate. A lower resolution is used to speed up the calculation.}
%    \label{fig:delta_width_err}
%\end{figure}

Another possibility we considered could be that the deviation is caused by angle variations between the design and the produced waveplate. Again, to examine this hypothesis we produced 50 designs with angle variations in a similar fashion as it was done for the designs with thickness errors. The result is shown in figure \ref{fig:delta_angle_err} where subfigure (a) shows the phase shift for the designs with a maximum of \SI{\pm2}{\degree} angle deviation added to each waveplate. Subfigure (b) shows a maximum \SI{\pm5}{\degree} misalignment error, in other words the same angle was added to all four waveplates. 

\begin{figure}[H]
    \begin{subfigure}[b]{.5\linewidth}
    \caption{}\label{}
    \centering\includegraphics[scale=0.65]{images/results/plots/polymer/dimension_errors/delta_angle_error_a.pdf}
    \end{subfigure}%
    \begin{subfigure}[b]{.5\linewidth}
    \caption{}\label{}
    \centering\includegraphics[scale=0.65]{images/results/plots/polymer/dimension_errors/delta_angle_error_b.pdf}
    \end{subfigure}
    \caption{(a) Random angle deviation of maximum \SI{\pm2}{\degree} added to each waveplate. (b) 50 different designs where we added the same but maximum of \SI{5}{\degree} deviation to every waveplate. Green and blue dashed curves show the designs with the minimum and maximum value, respectively. The horizontal dashed lines represent a $\pm \SI{3}{\percent}$ deviation from $\frac{\pi}{2}$.}
    \label{fig:delta_angle_err}
\end{figure}

We see that with an increased random angular deviation, the phase shift deviation becomes greater, however in a random fashion without following a particular trend, except for greater deviation for higher frequencies. Therefore it is unlikely to have caused the deviation. Additionally, angle errors are unlikely in the first place since the waveplate is printed in one piece. Furthermore, comparing to the errors caused by thickness variations we see that it is important to ensure a correct alignment in case of individually stacked waveplates. 

The last structural parameter is the periodicity of the stratification. We took 10 photos using a microscope of the waveplate from the top and bottom at different positions and measured $a$ and $b$. Two of the photos are shown in figure \ref{fig:hips_wp_photos}; photo (a) is taken from the bottom side and photo (b) is taken from the top side.

\begin{figure}[H]
\centering
    \subcaptionbox{\label{fig:p2bot4}}
        {\hspace*{-2em}\includegraphics[width=0.4\linewidth]{images/results/plots/polymer/dimension_errors/p2_bot_4.png}}
    \qquad
    \subcaptionbox{\label{fig:p2top2}}
        {\hspace*{-2em}\includegraphics[width=0.4\linewidth]{images/results/plots/polymer/dimension_errors/p2_top_2.png}}
    \caption{Microscope image of the 3D printed waveplate. (a) Bottom surface, (b) Top surface.}
\label{fig:hips_wp_photos}
\end{figure}

We calculated the average values of $a$ and $b$ for the 10 different positions and obtained the following: $a=\SI[separate-uncertainty = true]{734.55(4560)}{\micro \meter}$, $b=\SI[separate-uncertainty = true]{392.95(3820)}{\micro \meter}$ and $d=\SI[separate-uncertainty = true]{1139.00(3360)}{\micro \meter}$ while the expected values were $a=\SI{600}{\micro \meter}$, $b=\SI{500}{\micro \meter}$ and $d=\SI{1100}{\micro \meter}$. Using the measured values of $a$ and $b$ together with the measured optical parameters of HIPS we calculated the new adjusted form birefringence based on the average stratification dimensions. For the design and measured values of $a$ and $b$ the calculated refractive indices parallel ($n_p$) and perpendicular ($n_s$) to the stratification as well as the birefringence is shown in figure \ref{fig:ri_stripe_err}. We see that the refractive indices in both directions are approximately $0.05$ smaller for the design values of $a$ and $b$ compared to the measured values. However, the birefringence differs by less than $0.01$.

\begin{figure}[H]
    \begin{subfigure}[b]{.5\linewidth}
    \caption{}\label{}
    \centering\includegraphics[scale=0.65]{images/results/plots/polymer/dimension_errors/ri_stripe_error_a.pdf}
    \end{subfigure}%
    \begin{subfigure}[b]{.5\linewidth}
    \caption{}\label{}
    \centering\includegraphics[scale=0.65]{images/results/plots/polymer/dimension_errors/ri_stripe_error_b.pdf}
    \end{subfigure}
    \caption{(a) Calculated refractive indices $n_p$ and $n_s$ using the measured material parameters, once for the average of the measured values ($a$, $b$) and once for the design values given in table \ref{tab:res_hips_l4}. (b) The calculated birefringence for the measured and the design values ($a$, $b$).}
    \label{fig:ri_stripe_err}
\end{figure}

With these values of the birefringence for the measured values of $a$ and $b$ we again calculated the phase shift which is shown in figure \ref{fig:delta_stripe_err}. Even though the difference between the values of the actual printed structure and the design values are quite large we see that it has little impact on the expected phase difference. We can therefore with fairly high certainty rule out that the deviation of the stratification dimensions is the cause of the measured phase shift deviation.

\begin{figure}[H]
    \centering
    \includegraphics[scale=.7]{images/results/plots/polymer/dimension_errors/delta_stripe_error.pdf}
    \caption{The resulting phase shift using the calculated birefringence for the design values of $a$ and $b$ as well as the measured values of $a$ and $b$.}
    \label{fig:delta_stripe_err}
\end{figure}

It is worth mentioning that we also considered an error in the characterization of the optical parameters of the \SI{2}{\milli \meter} HIPS sample which are used to calculate the form birefringence. However, we can give an estimate of how large this deviation would have to be to obtain the measured phase shift. This can be accomplished by gradually increasing the refractive index of HIPS at each frequency and calculating the expected phase shift until we obtain the measured value. Assuming that the absorption is negligible we found that we would have to add at least $0.15$ to the originally measured refractive index of HIPS in order to obtain the measurement result. Additionally, a subsequent characterization of the original \SI{2}{\milli \meter} HIPS sample using the two THz-TDS setups showed a deviation smaller than $0.01$ compared to the refractive index measurement performed using the GHz setup.

Considering the different parameters used in the calculation of the phase shift this left us only with a problem determining the birefringence as being the cause of the deviation. We therefore compared the calculated birefringence to a CST simulation\footnote{CST Studio Suite is a commercial software for analyzing electromagnetic components. A few additional remarks about the simulation are given in the appendix section \ref{sec:CST simulation}}. Figure \ref{fig:CSTvsFormBF} shows the birefringence calculated using the Rytov approximation which is the model described in section \ref{sec:form_birefringence} about form birefringence shown as the red curve in the plot and the blue curve represents the birefringence obtained via a CST simulation. In both cases the calculations are based on the measured optical parameters of HIPS as well as the stratification dimensions $a=\SI[separate-uncertainty = true]{734.55(4560)}{\micro \meter}$ and $b=\SI[separate-uncertainty = true]{392.95(3820)}{\micro \meter}$ obtained from the microscope measurement.

\begin{figure}[H]
    \centering
    \includegraphics[scale=.7]{images/results/plots/polymer/CSTvsFormBF.pdf}
    \caption{The birefringence of a single waveplate with the measured values of $a$ and $b$ calculated using the Rytov approximation and compared to a CST simulation.}
    \label{fig:CSTvsFormBF}
\end{figure}

If we assume the simulated form birefringence is correct and consider that the deviation between the calculated and simulated form birefringence is smaller than what we saw for the birefringence difference between the measured and expected stratification dimensions $a$ and $b$ which were shown to be insignificant, then this comparison is a strong indication that the difference between simulated and calculated form birefringence is insignificant. A simple explanation is that there is an additional effect which changes the total or actual birefringence of the waveplate, in other words a hypothesized additional intrinsic birefringence besides the expected form birefringence. 

To test this we printed a single \SI{4}{\milli \meter} thick grating (i.e., a plate with the corresponding stratified structure) for which we again measured the average width of the material bars and air gaps using a camera mounted to a microscope and found that $a=\SI[separate-uncertainty = true]{631.37(2970)}{\micro \meter}$ and $b=\SI[separate-uncertainty = true]{460.25(4231)}{\micro \meter}$. We then proceeded to measure the birefringence using the bow-tie setup (section \ref{sec:bow_tie_setup}). Again, similar to previous measurements of the birefringence we simply measured the refractive index for two perpendicular directions; in this case parallel and perpendicular to the stratification of the sample. The result of the measurement is shown in figure \ref{fig:ri_slim_grating} and is compared to the expected form birefringence which is calculated using $a=\SI{631.37}{\micro \meter}$, $b=\SI{460.25}{\micro \meter}$ as well as the refractive index of HIPS based on the characterization of the \SI{2}{\milli \meter} thick square sample (figure \ref{fig:HIPS_PP_ri}). 

We see that the measured refractive indices both parallel and perpendicular to the stratification are larger than expected, the same is true for the birefringence which is around $0.02$ larger than expected. Therefore, the result of this measurement indeed confirms that the birefringence is significantly deviating from the expectation and that there is an additional effect which changes the measured birefringence besides the form birefringence. 

%\subcaptionbox{A cat\label{cat}}
%[.4\linewidth]{\includegraphics{cat}}%
%\subcaptionbox{An elephant\label{elephant}}
%[.4\linewidth]{\includegraphics{elephant}}

\begin{figure}[H]
    \begin{subfigure}[b]{.5\linewidth}
    \caption{}
    \centering\includegraphics[scale=0.6]{images/results/plots/polymer/ri_slim_grating_a.pdf}
    \end{subfigure}%
    \begin{subfigure}[b]{.5\linewidth}
    \caption{}
    \centering\includegraphics[scale=0.6]{images/results/plots/polymer/ri_slim_grating_b.pdf}
    \end{subfigure}
    \caption{(a) Measurement of the refractive index of the \SI{4}{\milli \meter} thick grating at \SI{0}{\degree} and \SI{90}{\degree} as well as the calculated refractive indices using $a=\SI{631.37}{\micro \meter}$, $b=\SI{460.25}{\micro\meter}$. At \SI{0}{\degree} the bars of the grating are aligned horizontally ($n_p$). (b) The birefringence of the calculation and measurement.}
    \label{fig:ri_slim_grating}
\end{figure}

\newpage

Since there apparently is an additional effect which increases the birefringence we decide to calculate the expected phase shift considering the measured birefringence shown in figure \ref{fig:ri_slim_grating} instead of the theoretical form birefringence. The result of this is shown in figure \ref{fig:delta_gratings_bf}. Even though this does not quite reproduce the measured result it is still the most plausible explanation for the deviation seen in the result, if we consider that this additional birefringence might vary between structures then it could fully explain the deviation seen in the result of the measurement.

\begin{figure}[H]
    \centering
    \includegraphics[scale=.55]{images/results/plots/polymer/delta_gratings_bf.pdf}
    \caption{The red curve shows the phase shift determined from the measurement of the waveplate while the blue curve shows the calculated phase shift of the output for the design values using the birefringence determined from the \SI{4}{\milli \meter} thick stratified sample which was measured using the bow-tie setup (section \ref{sec:bow_tie_setup}).}
    \label{fig:delta_gratings_bf}
\end{figure}

We assume that the additional intrinsic birefringence stems from the printing procedure.
Specifically, the structure is printed layer by layer and each layer consists of individually extruded polymer lines. Additionally, it has been shown that heating and subsequent extruding or pulling of a polymer can lead to alignment or reduction of the random orientation of the molecules in the polymer, i.e. a permanent birefringence is induced by the printing process \cite{Solr-urn:nbn:de:hebis:04-z2017-0786}. However, for the \SI{2}{\milli \meter} thick square test sample this effect would not have shown in the characterization with the GHz system when extracting the material refractive index and the absorption coefficient. This was the case, since by default in case of the square samples the 3D printer prints them so that the lines of each layer cross the lines of the previous layer creating a crossed pattern. Therefore due to this pattern a mesoscopic intrinsic birefringence cancels out on a macroscopic scale. On the contrary, in case of the gratings the printer by default traces out the lines of the functional area or the center lattice of the grating in the same direction due to the stratification. However if the lines of the square samples were traced out in the same direction then we should in theory be able to measure this intrinsic birefringence. We therefore printed two additional square samples where each line in each layer were drawn parallel to each other by the printer and characterized them using the bow-tie setup (section \ref{sec:bow_tie_setup}). One sample had a thickness of \SI{8}{\milli \meter} and the other a thickness of \SI{2}{\milli \meter}. The measured refractive indices along and perpendicular to the printing direction of these two samples are shown in figure \ref{fig:ri_fullplates} as well as those of the original \SI{2}{\milli \meter} thick sample with crossing lines which is included for comparison.
%However, since the lines in each layer are crossed or perpendicular relative to the lines in the following layer we do not see this additional birefringence in the original \SI{2}{\milli \meter} thick test sample which was characterized using the GHz setup. Although, if the lines of each layer are all printed in the same direction this effect becomes measurable on a macroscopic scale. Therefore, in order to investigate this assumption we printed two additional square samples where each line in each layer were drawn parallel to each other by the printer. One sample had a thickness of \SI{8}{\milli \meter} and the other a thickness of \SI{2}{\milli \meter}. 

\begin{figure}[H]
    \begin{subfigure}[b]{.33\linewidth}
    \caption{}\label{}
    \centering\includegraphics[scale=0.6]{images/results/plots/polymer/IntrinsicBF/ri_fullplates_b.pdf}
    \end{subfigure}%
    \begin{subfigure}[b]{.33\linewidth}
    \caption{}\label{}
    \centering\includegraphics[scale=0.6]{images/results/plots/polymer/IntrinsicBF/ri_fullplates_c.pdf}
    \end{subfigure}
    \begin{subfigure}[b]{.33\linewidth}
    \caption{}\label{}
    \centering\includegraphics[scale=0.6]{images/results/plots/polymer/IntrinsicBF/ri_fullplates_a.pdf}
    \end{subfigure}
    \caption{Refractive indices measured along two perpendicular directions of the sample using the bow-tie setup (section \ref{sec:bow_tie_setup}). Two of the samples (a) and (b) are printed so that the lines are all traced out in the same direction, while the third sample (c) is the original \SI{2}{\milli \meter} thick sample.}
    \label{fig:ri_fullplates}
\end{figure}

Figure \ref{fig:FullPlates_bf} shows the birefringence of the two samples printed in the same direction (a) and (b) as well as the sample with the default crossed printing pattern (c), the birefringence is calculated based on the measured refractive indices of the samples shown in figure \ref{fig:ri_fullplates}. As expected we see that the plates where the lines are all printed in one direction are indeed birefringent and the original \SI{2}{\milli \meter} thick sample does not appear to be birefringent. Additionally, the birefringence seems to be decreasing with increasing thickness of the sample, a similar trend is described in \cite{Solr-urn:nbn:de:hebis:04-z2017-0786}. It is worth mentioning that besides the thickness this type of birefringence also depends on the cooling temperature and speed of the extruded material \cite{Solr-urn:nbn:de:hebis:04-z2017-0786}.

Moreover, we see that the refractive indices of the two samples printed in the same direction are slightly higher compared to the original $\SI{2}{\milli \meter}$ sample with the crossed pattern. This means that in case of the gratings we can also expect the contribution from the form birefringence to be slightly higher since the form birefringence increases if we increase the difference of the refractive indices of the two materials (HIPS and air)\footnote{This can be seen by increasing $\epsilon$ in for example equation \ref{eq:inequality_proof} (Appendix section \ref{sec:bf_proof}) while keeping the other variables constant.}.

\begin{figure}[H]
    \centering
    \includegraphics[scale=.56]{images/results/plots/polymer/IntrinsicBF/FullPlates.pdf}
    \caption{Birefringence based on the result of the refractive index measurement shown in figure \ref{fig:ri_fullplates}.}
    \label{fig:FullPlates_bf}
\end{figure}

We can calculate the additional amount of birefringence necessary to have caused the phase shift seen in the original measurement (\ref{fig:meas_result}), the result of this calculation is shown in figure \ref{fig:FullPlatesWcalc} as the dashed black line. 

The calculation of the form birefringence is based on the measured values of $a$ and $b$ for the waveplate as well as the refractive index of the birefringent \SI{2}{\milli \meter} thick sample printed in the same direction and measured at \SI{0}{\degree}. We see that the birefringence difference which we attribute to the intrinsic birefringence is of the same order as the birefringence of the two birefringent samples (figure \ref{fig:FullPlates_bf}). 

\begin{figure}[H]
    \centering
    \includegraphics[scale=0.7]{images/results/plots/polymer/IntrinsicBF/FullPlatesWcalc_a.pdf}
    \caption{Measured birefringence of the three square samples compared to the calculated intrinsic birefringence based on the characterization of the waveplate. }
    \label{fig:FullPlatesWcalc}
\end{figure}
%\begin{figure}[H]
    %\begin{subfigure}[b]{.5\linewidth}
    %\centering\includegraphics[scale=0.7]{images/results/plots/polymer/IntrinsicBF/FullPlatesWcalc_a.pdf}
    %\caption{}\label{}
    %\end{subfigure}%
    %\begin{subfigure}[b]{.5\linewidth}
    %\centering\includegraphics[scale=0.7]{images/results/plots/polymer/IntrinsicBF/FullPlatesWcalc_b.pdf}
    %\caption{}\label{}
    %\end{subfigure}
    %\caption{(a) Measured birefringence of the three square samples compared to the calculated intrinsic birefringence based on the characterization of the waveplate. (b) Measured phase shift from the characterization of the waveplate and calculated phase shift based on the combination of form birefringence and the calculated intrinsic birefringence.}
%    \label{fig:FullPlatesWcalc}
%\end{figure}

We can therefore conclude that the waveplates in addition to the form birefringence obtain an additional intrinsic birefringence from the printing process since the material bars or elongated elements making up the grating are all printed in the same direction. The added birefringence was unaccounted for during the designing process of the waveplate and therefore results in a deviation of the phase shift from the expectation. Additionally, we know from the calculations that the intrinsic birefringence is strong enough to be able to cause the deviation. The other effects mentioned earlier can of course also be present but are causing smaller deviations which therefore makes them difficult to distinguish in the measurement. The additional birefringence is not necessarily a disadvantage if it is predictable and can be accounted for correctly, since then the total thickness of the waveplate can possibly be reduced. Assuming the intrinsic birefringence, which explains the measurement result, is the only effect which is unaccounted for in the model and the design then we with a fairly high certainty conclude that the concept of a form birefringence-based composite AWP does indeed work. 
