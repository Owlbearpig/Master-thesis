The frequency range from around \SI{100}{\giga \hertz} to \SI{10}{\tera \hertz} located midway between microwaves and infrared light is known as the terahertz (THz) frequency range \cite{Jepsen2011}. Electromagnetic (EM) waves in this range have numerous applications not only for fundamental research but also for the application in optoelectronic THz systems. The energy of THz photons loosely ranges from \SIrange[range-units=single]{1}{100}{\milli \electronvolt} which allows probing of physical phenomena such as low-energy excitations in 2D materials \cite{Winnerl2018} and rotational as well as vibrational modes via time domain spectroscopy \cite{Walther2002, Giesen2005, Baxter2011, Mittleman2003}. In addition, THz photons are not energetic enough to be ionizing which means that spectroscopy utilizing THz radiation preserves the sample of interest and is non hazardous \cite{Jepsen2011}.

The development of the technology for this frequency range underwent tremendous progress in the recent decades which resulted in development of more efficient, compacter, mobile and cost-efficient systems \cite{Wilk08, Jordens2008, Wilk11, Dietz11, Probst14, Probst15, Abdulmunem17, Kohlhaas17, Merghem2017}. This development made THz technology attractive for different industrial applications \cite{Naftaly2019}. Furthermore, the system development was accompanied by the development of different optical components such as couplers \cite{Huang2019, Nielsen10, Wu2016, Ying17}, beam splitter \cite{Ung12, Lenets21}, wave guides \cite{Ma2016, Gui2015}, and waveplates \cite{Nouman2016, Scherger2011, Wang2015}.

%Due to the increasing number of applications and research based on THz radiation, characterization and control of the field polarization also gained interest especially with the development of waveplates amongst other polarization related components \cite{Castro-Camus2012}. Waveplates and other polarization converters usually depend on the wavelength. This poses a problem for example in combination with time domain spectroscopy due to the usage of broadband radiation and it can be a limitation for polarimetry in general because of this. However, designs based on the combination of multiple monochromatic waveplates at different azimuths have been used to realize achromatic waveplates in the past \cite{Wu2020, Masson2006, Ivanov2012, Herrera-Fernandez2015}. In this work we propose and show the experimental demonstration of a similar design which is however based on a single 3D printed structure. With this method production cost and complexity can in theory be reduced. In addition, it has the advantage of reducing angle misalignment errors since this technique completely circumvents manual assembly.

Waveplates are commonly used for manipulation of the light polarization, which are often designed to work properly only at specific frequencies. Such monochromatic waveplates working in transmission can be easily achieved by using birefringent material of certain thickness. Considering widely employed THz Time-Domain spectroscopy (THz-TDS), waveplates working for a broader frequency range are of interest, since a broad frequency range is generated and probed during a single THz-TDS measurement. Broadband waveplates working in the THz range have already been demonstrated based on combining or stacking several monochromatic waveplates \cite{Wu2020, Masson2006, Ivanov2012, Herrera-Fernandez2015}, for which precise alignment is required. For this intrinsic birefringence of a material (e.g., quartz \cite{Masson2006}, sapphire \cite{Wu2020}) has been employed. In this work we focus on design, production, and characterization of broadband or achromatic waveplates (AWP) for the THz range. However, instead of using materials with intrinsic birefringence, we consider employing 3D printing to produce AWPs based on form birefringence \cite{Scheller2010}. Furthermore, stacking of waveplates can be simplified by relying on precision of the 3D printing process. In addition to 3D printed form birefringent samples, we design AWPs based on 3D printed ceramic samples, which show print related birefringence \cite{Ornik2021}. 

Following this introduction THz-TDS is described in the following chapter, specifically generation and detection of THz radiation. In the third chapter we present the fundamentals of EM wave polarization and propagation in anisotropic media. At the end of the third chapter we also introduce three mathematical representations of polarization. In the fourth chapter the AWP design is described in the context of the Jones calculus. Subsequently, we present the three experimental spectroscopy setups which have been used for the measurements in this work. The characterization of the fabricated waveplate design and a discussion of the results is presented in the sixth chapter and finally in the last chapter a conclusion is given together with an outlook.

%To that extent we introduce THz time domain spectroscopy in the following chapter, specifically generation and detection of THz radiation. In the third chapter the three experimental spectroscopy setups which have been used for the measurements in this work are presented. Following, in the fourth chapter the theoretical fundamentals are described which includes different representations of the polarization state, the underlying optical properties of anisotropic media which are utilized to transform one polarization state into another via waveplates and finally how the Jones calculus can be used to describe the AWPs. The characterization of the fabricated waveplate design and a discussion thereof is presented in the fifth chapter. A conclusion as well as an outlook is given in the final chapter.

%For example magnetic fields in the interstellar medium induce polarization due to the Faraday effect which can be characterized using waveplates \cite{Harper2018}.

%- Motivation (maybe just do this at presentation)
%Polarization control is an important aspect of optics. etc https://en.wikipedia.org/wiki/BB84
%    - usages of waveplates
%    - Folder with examples from litterature

% - Structure