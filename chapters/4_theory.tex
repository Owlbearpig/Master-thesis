\label{ch:theory}

In this chapter we will introduce the basic underlying physical principles of birefringent waveplates. Simply put, the purpose of a waveplate is to modify the polarization state of EM waves in a controlled way as they pass through the waveplate. This is most commonly achieved by making the plates out of a birefringent material with a certain thickness. To this end, we first describe the polarization ellipse in section \ref{sec:polellipse}. Following this the underlying physical principles birefringence and dichroism are explained in more detail in section \ref{sec:wave_prop}. Finally, in section \ref{sec:waveplates} we then show what role these two principles play in regards to waveplates. Any change to a polarization state and the polarization states themselves, can be described mathematically using the Jones calculus which will be shown in section \ref{sec:math_desc}. At the end this leads up to the description of achromatic composite waveplates and the design as it is used in this work. Both are presented in section \ref{sec:jonescalc} of this chapter.

\section{The polarization ellipse}
\label{sec:polellipse}
Simply put, the polarization or polarization state of an EM-wave is the direction in which the wave is oscillating. There are several ways to describe or represent the polarization state of EM-waves. One of them is the polarization ellipse which is a geometrical representation of the state. This remained the only satisfactory way to visualize the different polarization states for almost the entire 19\textsuperscript{th} century, although it is not very practical for carrying out calculations. At the end of the century another representation was proposed by Henri Poincaré, now known as the Poincaré sphere, which could be used for both calculations and visualization of the polarization states \cite{Collett2008a}. Poincaré's representation will be explained in section \ref{sec:math_desc}. The polarization ellipse can be derived either directly from a solution to the EM-wave equation or starting from Maxwell's equations.  It is practical to start from Maxwell's equations, since they are used for derivations later on in this work. In differential form Maxwell's equations are given by:
\par
\noindent\begin{minipage}{.5\linewidth}
\begin{align}
    \bm{\nabla}\cdot\bm{D} &=\rho_{f}\;,\label{eq:maxwell1}
    \vphantom{\frac{\partial\bm{B}}{\partial t}}\\
    \bm{\nabla}\cdot\bm{B} &=0\;,\vphantom{\frac{\partial\bm{B}}{\partial t}}\label{eq:maxwell2}
\end{align}
\end{minipage}%
\begin{minipage}{.5\linewidth}
\begin{align}
    \bm{\nabla}\times\bm{E} &=-\frac{\partial\bm{B}}{\partial t}\;,\label{eq:maxwell3}
    \\
    \bm{\nabla}\times\bm{H} &=\bm{J}_f
    +\frac{\partial\bm{D}}{\partial t}\;,\label{eq:maxwell4}
\end{align}
\end{minipage}
\newline

where $\bm{E}, \bm{H}$ denote the electric and magnetizing fields respectively,  $\bm{D}=\hat{\epsilon} \bm{E}$ is the electric displacement for a linear medium, $t$ is the time and $\hat{\epsilon}$ is the permittivity which depends on the material and is a complex function of the frequency. The materials used in this work are non magnetic, which means that the magnetic and magnetizing fields are proportional: $\bm{B} = \mu_0 \bm{H}$, where $\mu_0$ denotes the vacuum permeability. $\rho_f, \bm{J}_f$ is the free charge and current density respectively. Usually in optics there are no free charge carriers or free currents, which is also the case for all materials in this work, so we set $\rho_f=\bm{J}_f=0$ \cite{Roth2019}. If we then further assume that the materials are homogeneous so that $\hat{\epsilon}$ does not depend on position, then we can reduce equations \ref{eq:maxwell1}-\ref{eq:maxwell4} to the following symmetrical set of equations:

\noindent\begin{minipage}{.5\linewidth}
\begin{align}
    \bm{\nabla}\cdot \bm{E} &=0\;,\label{eq:maxwell1reduced}
    \vphantom{\frac{\partial\bm{B}}{\partial t}}\\
    \bm{\nabla}\cdot\bm{B} &=0\;,\vphantom{\frac{\partial\bm{B}}{\partial t}}\label{eq:maxwell2reduced}
\end{align}
\end{minipage}%
\begin{minipage}{.5\linewidth}
\begin{align}
    \bm{\nabla}\times\bm{E} &=-\frac{\partial\bm{B}}{\partial t}\;,\label{eq:maxwell3reduced}
    \\
    \bm{\nabla}\times\bm{B} &=\mu_0 \hat{\epsilon} \frac{\partial\bm{E}}{\partial t}\;,\label{eq:maxwell4reduced}
\end{align}
\end{minipage}
\newline

Applying the curl to equation \ref{eq:maxwell3reduced} and with equation \ref{eq:maxwell4reduced} we can decouple $\bm{E}$ and $\bm{B}$: 
\begin{equation}
    \bm{\nabla}\times \bm{\nabla}\times \bm{E} = -\mu_0 \hat{\epsilon} \frac{\partial^2\bm{E}}{\partial t^2}
\end{equation}
Then using the vector calculus identity: $\bm{\nabla}\times \bm{\nabla}\times \bm{F} = \bm{\nabla} \left( \bm{\nabla} \cdot \bm{F} \right) - \bm{\nabla}^2 \bm{F}$, for any function $\bm{F}$, together with equation \ref{eq:maxwell1reduced} we get the EM-wave equation: \footnote{We neglect the magnetic field, since we are only concerned with non magnetic materials in this work. Nevertheless, the derivation of the wave equation for the magnetic field looks similar to the derivation for the electric field.}
\begin{equation}
    \label{eq:waveeq}
    \bm{\nabla}^2 \bm{E} = \mu_0 \hat{\epsilon} \frac{\partial^2\bm{E}}{\partial t^2}
\end{equation}
and via a Fourier transform $(\partial_t \xrightarrow{\mathscr{F}} i\omega)$ we readily obtain the wave equation in frequency space also known as the Helmholtz equation for $\bm{E}$:
\begin{equation}
    \label{eq:waveeqfreq}
    \bm{\nabla}^2 \bm{E} = -\mu_0 \hat{\epsilon} \omega^2 \bm{E},
\end{equation}
where $\omega$ is the angular frequency.
A solution to the wave equation is a monochromatic coherent plane wave with a frequency $\omega$ which is given by: 
\begin{equation}
    \label{eq:planewave}
    \bm{E} = \bm{E_0} e^{i(\bm{k}\bm{r} - \omega t + \delta)},
\end{equation}
where $\bm{r}$ and $\bm{k}$ denote the position and wave vector respectively, $\bm{E_0}$ is a complex vector which denotes the maximum amplitude and oscillation direction of the wave, $\delta$ is the phase constant. If we assume the wave propagates along the z-direction in a Cartesian coordinate system, then the amplitude will be orthogonal to the z-direction, i.e a transverse wave. This follows directly from equation \ref{eq:maxwell1reduced}. From the Helmholtz equation we then find the requirement that the wave number, which is the magnitude of the wave vector, and the angular frequency are related by:
\begin{equation}
    \label{eq:wavevector_req}
    k = \sqrt{\mu_0 \hat{\epsilon}} \omega
\end{equation}
The phase velocity of the wave can then be expressed as:
\begin{equation}
    v = \frac{\omega}{k} = \frac{1}{\sqrt{\mu_0 \hat{\epsilon}}} = \frac{c}{\tilde{n}},\:\tilde{n} = \sqrt{\frac{\hat{\epsilon}}{\epsilon_0}},
\end{equation}
where $c$, $\epsilon_0$ and $n$ are the speed of light in vacuum, vacuum permittivity and refractive index respectively, so that from the wave equation we get two decoupled wave equations, each describing an oscillation in the x-z plane and the y-z plane. This is also known as Fresnel's wave theory which was proposed and verified experimentally around 1820 by Augustin-Jean Fresnel and François Arago, more than 40 years before Maxwell's equations were published \cite{Collett2009,Jackson1998}.

This means that equation \ref{eq:planewave} consists of only two perpendicular components. Additionally, if we consider that Maxwell's equations \ref{eq:maxwell1reduced}-\ref{eq:maxwell4reduced} are all linear and real, then it follows that the real part of any particular solution is also a solution to the wave equation. \footnote{It is worth noting that this only works if the equations are real. For example, Schrödinger's equation is linear but complex, therefore even if $\psi$ is a solution $\operatorname{Re}(\psi)$ and $\operatorname{Im}(\psi)$ are not.} We can then write equation \ref{eq:planewave} as two sinusoidal waves:
\begin{equation}
\label{eq:plane_wave_realparts}
\begin{aligned}
    E_x(z, t) = E_{0x}\cos(kz-\omega t + \delta_x) \\
    E_y(z, t) = E_{0y}\cos(kz-\omega t + \delta_y), 
\end{aligned}
\end{equation}
where $\delta_x$ and $\delta_y$ are arbitrary phases of the two components. Depending on the values of $E_{0x}, E_{0y}$ and the relative phase or simply phase, which is $\delta = \delta_y - \delta_x$, we describe different polarization states. Figure \ref{fig:Ex_Ey_planewaves} shows the real parts of the two plane wave components for a wave traveling along the z-direction when $\delta=0$ and $E_{0x}=E_{0y}$.

\begin{figure}[h]
    \centering
    \includestandalone{images/4_chapter04/tikz_e_wave}
    \caption{Two sinusoidal waves given by equation \ref{eq:plane_wave_realparts}, which are the real parts of the plane wave components. Both waves have the same amplitude and phase.}
    \label{fig:Ex_Ey_planewaves}
\end{figure}

With this in place we can visualize the different polarization states, that is the different directions the electrical field, or simply the light, can oscillate in. If the light is confined to a plane along the direction of propagation it is said to be linearly polarized (LP), this plane is known as the polarization plane. An example of this is shown in figure \ref{fig:E_planewave}, where the polarization plane is at an angle of \SI{45}{\degree} relative to the x-axis. 

\begin{figure}[h]
    \centering
    \includestandalone[scale=1.1]{images/4_chapter04/tikz_linear_pol}
    \caption{The black curve shows the superposition of the two sinusoidal waves from equation \ref{eq:plane_wave_realparts}, which is the same as the real part of a LP plane wave at an angle of \SI{45}{\degree} to the x-axis. In addition, the horizontal and the vertical components of the field vector are shown as the red and blue curves respectively.}
    \label{fig:E_planewave}
\end{figure}

Now, if $\delta$ is exactly $\nicefrac{\pi}{2}$ and $E_{0x} = E_{0y}$, then we have the case shown to the right in figure \ref{fig:circ_pol_planewave}. A field with this kind of polarization traces out a circle as time passes for a fixed position in space. On the other hand, if we keep time fixed then the field vector describes a helix along the direction of propagation. This means that the field only changes direction but not its magnitude. This type of polarization is therefore called circular polarization (CP). Additionally, because we can trace out a circle in two different directions (clockwise or counterclockwise) we can therefore distinguish between two different types of circular polarizations. These two types are called left CP (LCP) or right CP (RCP) depending on the direction the circle is traced out, either in a left- or right-handed sense. This is called the handedness of the light, which can be determined using the right hand rule. For that we fix the right hand at a point on the helix and point the thumb away from the receiver, that is against the direction of propagation. Then the light is right-handed if the helix is traced out in the direction of the other fingers as time evolves, otherwise it is left-handed. The light shown to the left of figure \ref{fig:circ_pol_planewave} is therefore left-handed while the example on the right side of figure \ref{fig:circ_pol_planewave} shows right-handed CP light. The reason why we have two types of CP in the first place, is because of the fact that one component can lead the other, as we can see in figure \ref{fig:circ_pol_planewave}. Where for RCP the horizontal component leads the vertical one by a quarter period $\left(\delta = \frac{\pi}{2}\right)$, for LCP it is the other way around and $\delta = -\frac{\pi}{2}$. 


\begin{figure}
\centering
\subcaptionbox{\label{fig:circ_pol_planewave_a}}
    {\includestandalone[scale=0.8]{images/4_chapter04/tikz_circ_pol_lh}}
\subcaptionbox{\label{fig:circ_pol_planewave_b}}
    {\includestandalone[scale=0.8]{images/4_chapter04/tikz_circ_pol_rh}}
\caption{The left figure shows an example of LCP light and the right figure RCP light. Furthermore, handedness originates from one component leading the other. In the case of LCP light the vertical component leads the horizontal component and vice versa for RCP light.}
\label{fig:circ_pol_planewave}
\end{figure}

Finally, the last polarization state we have not mentioned yet, even though it is in fact the most general, is the elliptical polarization state. The polarization states we have dealt with so far have all been limiting cases of this state. For a fixed position this is fairly simple to visualize geometrically; LP light traces out a line while CP light traces out a circle, both of these shapes are degenerate forms of an ellipse. A summary of the different degenerate states and the corresponding amplitude and phase combinations are shown in table \ref{tab:pol_state_summary}, which is divided into three parts with a pair of states in each. The linear horizontal/vertical polarization (LHP/LVP) states are at the top, the linear $\pm\SI{45}{\degree}$ polarization ($L_{\pm45}P$) states are in the middle and the CP states are at the bottom. The last column shows the corresponding degenerate ellipse for each state.

\begin{table}[h]
    \centering
    \includestandalone{images/4_chapter04/pol_summary_table}
    \caption{Summary of the different degenerate polarization states, with corresponding conditions figures. The first four states are linearly polarized at different angles relative to the x-axis; $\SI{0}{\degree}$, $\SI{90}{\degree}$, $\SI{45}{\degree}$ and $\SI{-45}{\degree}$ respectively. The last two are the RCP and LCP states.}
    \label{tab:pol_state_summary}
\end{table}

We can also realize this mathematically by eliminating the time and position dependency of the two components in equation \ref{eq:plane_wave_realparts}, so that we get \footnote{The derivation is given in \ref{sec:deriv_pol_ellipse}}:

\begin{equation}
    \centering
    \label{eq:pol_ellipse}
    \frac{1}{\sin^2 \delta} \left[ \left(\frac{E_x}{E_{0x}}\right)^2+\left(\frac{E_y}{E_{0y}}\right)^2-2\frac{E_x E_y}{E_{0x} E_{0y}}\cos \delta \right]=1,
\end{equation}

which describes an ellipse in its nonstandard form. Because of this the equation is called the polarization ellipse. It is worth noting that even though the time and position dependencies have been explicitly eliminated, the wave components $E_x$ and $E_y$ continue to depend on them. That means $E_{0x}$, $E_{0y}$ and $\delta$ are the polarization ellipse parameters, which define the shape of the ellipse. Therefore, the shape of the ellipse is independent of time and position. There are several ways to parameterize the polarization ellipse or ellipses in general using two parameters. One way is to use the orientation angle $\psi$ and the ellipticity angle $\chi$, which both can be expressed through the polarization ellipse parameters: 

\begin{equation}
    \label{eq:ellipse_orientation}
    \tan 2\psi = \frac{2E_{0x}E_{0y}}{E_{0x}^2 - E_{0y}^2}\cos \delta = 
    \tan 2\alpha \cos \delta
    ,\: 0\leqslant\psi\leqslant\pi,
\end{equation}
\begin{equation}
    \label{eq:ellipse_ellipticity}
    \sin 2\chi = \frac{2E_{0x}E_{0y}}{E_{0x}^2 + E_{0y}^2}\sin \delta = 
    \sin 2\alpha \sin \delta,\: \nicefrac{-\pi}{4}\leqslant\chi\leqslant\nicefrac{\pi}{4},
\end{equation}
where we have defined $\tan \alpha = \frac{E_{0y}}{E_{0x}}$ so that $\alpha$ and $\delta$ constitute another set of parameters associated with a second equivalent parameterization. Figure \ref{fig:pol_ellipse} shows an example of a polarization ellipse together with the orientation and ellipticity angles. We see from the figure that for $\psi \rightarrow 0, \pi$ or $\psi \rightarrow \frac{\pi}{2}$ the major and minor axes $(a$ and $b)$ coincide with the coordinate axes. Likewise for $\chi \rightarrow 0$ the ellipse degenerates to a line and for $\chi \rightarrow \pm\frac{\pi}{4}$ we get a circle \cite{Collett2009}. Actually, in most cases a light beam is a mixture of different polarization states, this is referred to as unpolarized light. It consists of a high number of short wave trains which are emitted from the individual atoms. In most cases these wave trains are radiated from dipoles and are therefore linearly polarized, but since every atom oscillates in a different direction the polarization will also vary for each wave train. Hence, the beam will be a mixture of different states \cite{Roth2019}. 
% TODO $\psi \rightarrow 0, \pi$ or $\psi \rightarrow \frac{\pi}{2}$ ???
\begin{figure}[h]
    \centering
    \includestandalone[scale=1.0]{images/4_chapter04/tikz_pol_ellipse}
    \caption{Example of a polarization ellipse, which shows the ellipticity $\chi$ and orientation angle $\psi$ geometrically. The major and minor axes of the polarization ellipse are denoted by $a$ and $b$ respectively. We see that $\chi$ changes with the shape of the ellipse while $\psi$ is simply the orientation relative to the x-axis.}
    \label{fig:pol_ellipse}
\end{figure}

We finish off this section, as a bit of a side note, by mentioning the conventions used so far. If we go back to the plane wave solution (equation \ref{eq:planewave}) of the wave equation we see that a replacement of $i$ with $-i$ yields another valid linearly independent solution. This leads to two different sets of definitions, depending on the sign of the imaginary unit. The first one, with a positive sign and the one used here, is commonly used in physics texts. While the other one with a negative sign is mainly used in electrical engineering (EE) texts. Mathematically the difference between these two conventions is simply the direction of rotation of the phasor in the complex plane with increasing time. An example of this is shown for a phasor $x$ in figure \ref{fig:phasor}. 

\begin{figure}[h]
    \centering
    \includestandalone[scale=1.3]{images/4_chapter04/tikz_phasor_rotation}
    \caption{Example of a phasor $x$ in the complex plane. The sign of the imaginary unit determines the direction of rotation for increasing times. The rotation is indicated by the blue arrows and is either in the clockwise $(-i)$ or counterclockwise $(+i)$ direction.}
    \label{fig:phasor}
\end{figure}

The choice of sign has other implications. We know from equation \ref{eq:wavevector_req} that the wavevector is in general complex because the permittivity is, so we can write it as $k=\tilde{\beta} + i \tilde{\alpha}$. On the other hand we obtain two expressions for a plane wave propagating in the z-direction, one for each definition:
\begin{align}
    &\bm{E_{0}}e^{i((\tilde{\beta} + i \tilde{\alpha})z-\omega t + \delta)} \\
    &\bm{E_{0}}e^{-i((\tilde{\beta} + i \tilde{\alpha})z-\omega t + \delta)}, 
\end{align}
if we factor out the $\tilde{\alpha}$ dependency we get:
\begin{align}
    e^{-\tilde{\alpha} z}&\bm{E_{0}}e^{i(\tilde{\beta} z-\omega t + \delta)} \\
    e^{\tilde{\alpha} z}&\bm{E_{0}}e^{-i(\tilde{\beta} z-\omega t + \delta)}. 
\end{align}
We see that in case of the EE definition the amplitude would increase as the wave propagates, which is nonsense. Therefore the sign of the imaginary part of the wavevector must be chosen accordingly. So for the EE definition we get:
\begin{equation}
    k=\tilde{\beta} - i \tilde{\alpha}
\end{equation}
and 
\begin{equation}
    k=\tilde{\beta} + i \tilde{\alpha}
\end{equation}
for the physics definition.
As a result the sign of the imaginary part of the refractive index will also depend on the chosen definition. When the EE definition is used:
\begin{equation}
    \tilde{n}= n - i \kappa
\end{equation}
and using the physics definition:  
\begin{equation}
    \tilde{n}= n + i \kappa.
\end{equation}

The physics definition is used for example in the textbooks by John D. Jackson \cite{Jackson1998} and David J. Griffiths \cite{Griffiths2017} and also in \cite{Collett2009}, which is the reference used for most of this section.
There are not only convention differences between engineers and physicists but also even between different branches of physics. The definition of CP handedness relies on the point of view; that is either looking from the source in the direction of propagation or from the receiver against the direction of propagation. Here we use the version which is defined from the point of view of the receiver. This is also the convention commonly used in optics \cite{Bass1995, M.LandiDeglInnocenti2004} as well as by SPIE \cite{Collett2009}.
The other convention is used by radio astronomers \cite{Born1999}, quantum physicists, because it is consistent with the convention of handedness for a particle's spin and it is also used by electrical engineers \cite{Orfanidis2004}.

\newpage

\section{Wave propagation in anisotropic media}
\label{sec:wave_prop}
In the previous section we saw that the electrical field consists of two perpendicular components which trace out an ellipse with the advancement of time for any fixed position. Additionally, the shape and orientation of the ellipse depend on the relative phase of the two components as well as their amplitudes. Naturally this dependency is symmetrical in the two amplitudes and the relative phase, as can be seen directly from equations \ref{eq:ellipse_orientation} and \ref{eq:ellipse_ellipticity}. Therefore in order to change the polarization state, the change in each of the individual phases must be different and or that of the two amplitudes. Otherwise simply the scale of the ellipse changes and not its shape or orientation. Consequently, for this to happen the medium the wave is propagating in must be anisotropic. Now, if the absorption of the material is direction dependent, then the amplitude will vary differently for the two components. Materials with this property are said to be dichroic. Likewise, some materials cause different phase changes between the two components. This material property is called birefringence. In the following section we will explain these two properties phenomenologically as well as their underlying physical principle.
% TODO Removed subsection
\subsection{Dichroism}
\label{sec:dichroism}
In general, the term dichroism refers to the selective absorption of one of the two perpendicular field components of the light. Materials with this property are said to be dichroic. This can be used for example in polarizers, which are extremely anisotropic so that one component is ideally completely extinguished by the polarizer, while it is transparent to the other field component. The simplest polarizer of this sort is a grid of parallel conducting wires as it is shown in figure \ref{fig:wire_grid_polarizer} which is also known as a wire grid polarizer (WGP). The WGP is similar to a conducting diffraction grating with a sub-wavelength period. Now, if we assume that the light initially is unpolarized as it propagates towards the wire grid from the left, then the light will be linearly polarized perpendicular to the wires after the WGP. We can see this by considering the interaction of the two field components with the wires separately. One component can then be set to be parallel to the wires and the other perpendicular. The electrons are free to move in the y-direction, because the wires are conducting, so that a current is generated due to the acceleration of the electrons by the parallel field component. This current results in joule heating of the wires so that a part of the energy of the electric field is turned into heat. Additionally, the accelerated electrons radiate in both the forward and backward directions. The part that is radiated forwards, along the direction of propagation, mostly cancels the transmitted incident wave and the other part appears as the reflected wave. On the other hand, the electrons along the x-direction, that is perpendicular to the wires, are not free to move very far so that the corresponding wave component passes almost freely. So, at the end the component parallel to the wires is mostly extinguished and the light is therefore linearly polarized perpendicular to the wires \cite{Hecht}.

\begin{figure}[h]
    \centering
    \includestandalone[scale=1.30]{images/4_chapter04/tikz_wire_grid_polarizer}
    \caption{A schematic of a wire grid polarizer which is essentially a conducting sub-wavelength grating. From the left unpolarized light interacts with the wire grid so that the field components parallel to the wires are eliminated. After passing the polarizer the light is therefore linearly polarized perpendicular to the wires.}
    \label{fig:wire_grid_polarizer}
\end{figure}

It has been shown that it is possible to polarize even visible light using these wire grids. In 1960 George R. Bird and Maxfield Parrish fabricated a WGP having 2160 wires per mm, making it suitable for Near-Infrared light \cite{Bird1960}. Naturally, WGPs also exist for the lower THz-range. An example of this is an aluminum-based WGP with a frequency range of \SIrange{0.3}{2.5}{\tera \hertz} and an extinction ratio ranging from \SI{45}{\dB} for lower frequencies and \SI{30}{\dB} for the higher ones. The extinction ratio is simply the ratio of the transmitted power coefficients for the field components perpendicular and parallel to the wires in decibel, so it is a measure of how well the light is only transmitted perpendicular to the wires after passing the polarizer \cite{Ferraro2016}. 
In contrast to a WGP some materials are inherently dichroic. For example the mineral tourmaline has a particular axis, which in general is called the optic axis, light propagating parallel to this axis experiences a smaller absorption coefficient compared to any other direction. Additionally, this selective absorption depends on the wavelength of the light. Therefore if the crystal is illuminated with white polarized light, then the absorbed colors will change depending on the polarization and so the crystal will change color depending on the polarization. This is where the name dichroic comes from, meaning two colors. The mechanism that gives rise to dichroism in crystals can be explained by considering its microscopic structure. The electrons bound to the lattice points of the crystal interact with electrons bound to other nuclei, especially the nearest neighbors. So if the crystal lattice is asymmetrical then the binding forces of the electrons will also be asymmetrical resulting in a polarization dependent response. Additionally, if the conductivity depends on the direction so will the amount of joule heating generated due to the induced currents. As a consequence the absorption will be direction dependent. 

To summarize, in this section we discussed two types of dichroism; one originating from a non-zero anisotropic conductivity of the structure on a macroscopic scale explained using the example of a WGP and the other type arising from asymmetries on the atomic scale. Loosely speaking, the effects of dichroism and birefringence in the cases relevant for this work can be explained by a direction and polarization dependent imaginary and real part of the refractive index respectively. This complicates the description of em-wave propagation in anisotropic materials, as we will see in the following section which introduces the most relevant concepts related to birefringence \cite{Hecht}.
% TODO Removed subsection
\subsection{Birefringence}
\label{sec:birefringence}
Birefringence is the material property responsible for the phenomenon of double refraction. This phenomenon was observed for the first time in calcite crystals by the Danish scientist Rasmus Bartholin in 1669 \cite{RasmusBartholin1669, Restaino2015}. The effect of double refraction on a light beam passing through a birefringent material is relatively easy to show. One can simply place a birefringent crystal on top of some text, as it is shown in figure \ref{fig:3birefringence}. Two images of the text become visible through the crystal and each image is shifted by a different amount. These images are not simply reflections at the crystal-air interface, since the images are brighter compared to images formed through reflections, and they are even visible when looking down directly from above. Also if the crystal is rotated then one of the images will be stationary, while the other one will circle around the stationary image. The waves forming the stationary image are known as the ordinary waves (o-rays or o-waves) and combined they form the o-wave, while the ones forming the moving image are called extraordinary waves (e-rays or e-waves) as suggested by Bartholin at the discovery of the phenomena. The same effect can be illustrated by sending an unpolarized laser beam through the crystal. The beam is split in two as it propagates through the crystal. One can show, using a WGP or any other polarizer, that the polarization states of the two beams are orthogonal \cite{Roth2019}.

\begin{figure}[h]
    \centering
    \includegraphics[scale=0.25]{images/4_chapter04/birefringence.png}
    \caption{A calcite crystal placed on top of some text. This demonstrates the effect of birefringence on the original image seen through the crystal \cite{Roth2019}.}
    \label{fig:3birefringence}
\end{figure}

The explanation of the underlying physical principle of birefringence is similar to that of dichroism; it can be explained by considering the structure of the material. We can represent the binding of the electron cloud to the nuclei at each crystal lattice point using a harmonic oscillator model. The model, which is shown in figure \ref{fig:electron_shell}, consists of a shell bound to the nuclei with springs. Furthermore, the springs have different spring constants and the nuclei is fixed in space. So, if an electron is displaced parallel to one of the springs it will oscillate with a different frequency relative to any other direction. How this finally influences the refractive index can be explained by considering the propagation of light through a transparent medium. As the wave propagates through the medium the electrical field excites the electrons which then in turn reradiate because any accelerating charge emits radiation. These reradiated wavelets recombine to form the refracted wave. Additionally, the phase velocity of the refracted wave will be determined by the frequency difference between the natural oscillation of the electrons and the frequency of the electrical field. Since the refractive index is simply the ratio between the speed of light in vacuum and in the medium it will change too depending on this frequency difference. If we go back to figure \ref{fig:electron_shell} this means that for example a wave linearly polarized parallel to the x-axis experiences a different refractive index compared to one polarized parallel to the z-axis. That is also the definition of birefringence; a material with two or more different refractive indices. 

\begin{figure}[h]
    \centering
    \includestandalone[scale=0.9]{images/4_chapter04/tikz_electron_shell}
    \caption{Lorentz oscillator model of an electron cloud, which is represented by a negatively charged spherical shell. It is bound to its positively charged nucleus via springs of different strength. The mass of the nucleus is assumed to be infinite so that it is fixed in space. This model is a simplification of an optical anisotropic material, for an isotropic material the spring constants would all be equal.}
    \label{fig:electron_shell}
\end{figure}

At this point it is worth making a few general remarks about this mechanical spring model in the following part before we continue. It is called the Lorentz oscillator model after Hendrik Antoon Lorentz who first devised this theory in the late nineteenth century. This was before the discovery of quantum mechanics and it was therefore originally purely a classical model. The model was then later adapted to quantum mechanics and is still often used to describe the interaction between atoms and electric fields. Lorentz did not assume that the electron actually was bound to the nucleus by a physical spring but rather that the force could be approximated by a harmonic oscillator. This is a justified assumption since any kind of binding force can be approximated by a harmonic oscillator, if the displacement is small enough so that only the linear terms in the Taylor expansion are significant. However, this classical version of the model fails at explaining a large number of the observed effects, especially when a weak field is interacting with a small number of atoms. For this situation and others, at least a partially quantum mechanical description is required. As an example, in \cite{Rerat2020} the birefringence is calculated using a quantum mechanical treatment \cite{doi:https://doi.org/10.1002/9780470409718.ch3}. For the rest of this section we will explain the phenomena described at the start of the section and then introduce a few more useful concepts to finish off the section. A concept that is helpful in understanding the phenomena is the optic axis of a crystal\footnote{In the context of crystals it is called the optic axis, otherwise it is called the optical axis like the axis perpendicular to the center of a lens \cite{Hecht}.}. That is, if we have the case in which two springs have the same rigidity, for example those in the xy-plane, then the optic axis will be parallel to the z-axis. The special property of the optic axis is that any wave propagating parallel to it will only experience one refractive index, because the material is optically symmetrical under any rotation about this axis. This means that the two components of the wave, which are perpendicular to the direction of propagation, will experience a homogeneous medium and therefore same refractive index. Since this symmetry exists at every lattice point, the optic axis is actually a direction and not just a line through a single point \cite{Hecht}. 

Furthermore, this explains the phenomena described at the beginning of this section. Going back to the calcite crystal shown in figure \ref{fig:3birefringence} we see that the optic axis must be at an angle, not parallel or perpendicular, to the surface normal. Otherwise we would not observe double refraction. Additionally, the field components perpendicular to the optic axis will not be refracted and therefore form the o-waves, which makes the image look like it is seen through an ordinary piece of glass. The other field components will partly be aligned parallel to the optic axis and partly perpendicular to it. This means that the wavefront is distorted and that part of the field is therefore refracted. Those refracted parts then form the e-wave. Finally, because of the spatial separation of the two field components we also have an explanation of why the polarizations states of the o- and e-waves are orthogonal. For clarity a sketch of the field, as it passes through the calcite crystal, is shown in figure \ref{fig:calcite_beams}. A crystal with exactly one symmetry axis or optic axis like calcite is said to be uniaxial. Crystals with two optic axes also exists, i.e. the case shown in figure \ref{fig:electron_shell} where the springs in the x-, y- and z-directions all differ, these crystals are called biaxial but we will not encounter them in this work. 

\begin{figure}[h]
    \centering
    \includestandalone[scale=0.95]{images/4_chapter04/tikz_calcite_beams}
    \caption{The field components are separated as the light passes through the calcite crystal thereby forming the e- and o-wave. The components perpendicular to the optic axis shown as red dots pass through as expected for normal incidence, while the other components represented by the blue arrows are refracted. The component dependent spatial separation also explains why the polarization states of the two waves are orthogonal.}
    \label{fig:calcite_beams}
\end{figure}

Figure \ref{fig:uniaxial_source} shows the propagation of a wave emitted from an unpolarized source placed within a uniaxial crystal. The component of the field perpendicular to the optic axis will propagate with the same phase velocity $v_{\bot}$ in all directions, so that the wavefronts form the o-wave which further expands in the shape of a circle. Meanwhile the field component, which is not everywhere perpendicular to the optic axis, will be traveling with a phase velocity $v_{\parallel}$ in the directions perpendicular to the optic axis. The wavefronts of the e-wave therefore form an ellipse. We therefore end up with different refractive indices with extrema parallel and perpendicular to the optic axis. In this case the highest refractive index $n_o = \nicefrac{c}{v_{\bot}}$ is found in the directions parallel to the optic axis, while it is lowest for the perpendicular directions. As we can see from the figure $v_{\parallel} > v_{\bot}$ and since the refractive index perpendicular to the optic axis is given by $n_e = \nicefrac{c}{v_{\parallel}}$ it follows that $n_e < n_o$. With this it is possible to quantify the birefringent strength of the material via the difference $\Delta n = n_e - n_o$ in the refractive indices of the e- and o-wave. The difference $\Delta n$ is known as the birefringence of the material. Furthermore, if $\Delta n$ is negative then the crystal is said to be negative uniaxial which is the case for calcite and the example shown in figure \ref{fig:uniaxial_source}. Likewise, if $\Delta n$ is positive the material is positive uniaxial and in that case the ellipse formed by the wavefronts of the e-wave are enclosed within the circle formed by the wavefronts of the o-wave. 

\begin{figure}[h]
    \centering
    \includestandalone[scale=0.73]{images/4_chapter04/tikz_uniaxial_source}
    \caption{Cross section perpendicular to the optic axis of an unpolarized light source inside a uniaxial crystal. The red dots represent components which are everywhere perpendicular to the optic axis. They form the o-wave and expand in a circle because they experience the same refractive index everywhere. The components represented by the blue arrows are partly pointing in the direction of the optic axis. Therefore, depending on their orientation to the optic axis they experience a different refractive index so that the wavefront forms an ellipse.}
    \label{fig:uniaxial_source}
\end{figure}

For biaxial crystals birefringence is still defined as the biggest difference between the refractive indices along the different directions. For completeness, it is worth mentioning the case in which the refractive indices along all the directions are equal. The unit cell of these materials is cubic causing them to be isotropic \cite{Hecht}. 
% TODO @JAN Move index ellipsoid+determining optic axis to appendix or remove completely 
Finally, to conclude this section we will give a description of the same situation but from another point of view. To that extend we consider the energy density of an em-wave in an anisotropic medium.
We start by introducing a charge density $\rho$ and then express the work done by the fields moving it a distance $d\bm{l}=\bm{v}dt$ in terms of the fields themselves. Since that gives us an idea of the energy stored in the fields. The force acting on the charge distribution is the Lorentz force density\footnote{The force density has the dimensions of force per volume and is needed so that when we integrate over the volume we get the total force.} $\bm{f}$:
\begin{equation}
    \bm{f}\cdot d\bm{l}=\rho(\bm{E}+\bm{v}\times\bm{B})\cdot\bm{v}dt=\rho\bm{E}\cdot\bm{v}dt
\end{equation}
We get the last equality because $\bm{v}\cdot(\bm{v}\times\bm{B})$ is zero for any $v$ and $\bm{B}$. 
If we then integrate over the volume $V$ enclosing the charge distribution and use the fact that $\bm{J}=\rho\bm{v}$ we get:
\begin{equation}
    \int_V \bm{f}\cdot d\bm{l} dV =\int_V \bm{E}\cdot\bm{J} dVdt
\end{equation}
since $\int_V \bm{f} dV = \bm{F}$ and $dW = \bm{F} d\bm{l}$ we finally end up with:
\begin{equation}
    \label{eq:emf_work}
    \frac{dW}{dt} = \int_V \bm{E}\cdot\bm{J} dV
\end{equation}
Since we want to express this in terms of the fields only we use the Ampère-Maxwell law, that is equation \ref{eq:maxwell4}:
\begin{equation}
    \bm{J}\cdot\bm{E} = \frac{1}{\mu_0}\bm{E}\cdot(\bm{\nabla}\times\bm{B})-\bm{E}\cdot\frac{\partial\bm{D}}{\partial t}
\end{equation}
We can rewrite $\bm{E}\cdot(\bm{\nabla}\times\bm{B})$ as $-\bm{B}\cdot\frac{\partial \bm{B}}{\partial t} - \bm{\nabla}\cdot(\bm{E}\times\bm{B})$ using a product rule from vector calculus and the Maxwell–Faraday law (equation \ref{eq:maxwell3}). With the fact that $\bm{A}\cdot\frac{\partial \bm{A}}{\partial t} = \frac{1}{2} \frac{\partial A^2}{\partial t}$ for any vector $\bm{A}$ we finally end up with:
\begin{equation}
    \bm{J}\cdot\bm{E} = -\frac{1}{2}\frac{\partial}{\partial t}\left(\frac{1}{\mu_0}B^2+\bm{E}\cdot\bm{D}\right)-\frac{1}{\mu_0}\bm{\nabla}\cdot(\bm{E}\times\bm{B})
\end{equation}
which we then substitute back into equation \ref{eq:emf_work} and with the help of the divergence theorem we get:
\begin{equation}
    \label{eq:poyntings_theorem}
    \frac{dW}{dt} = -\frac{\partial}{\partial t} \int_V \frac{1}{2}\left(\frac{1}{\mu_0}B^2+\bm{E}\cdot\bm{D}\right)dV -\frac{1}{\mu_0}\oint_{\partial V} (\bm{E}\times\bm{B}) \cdot d\bm{s}
\end{equation}
This is known as Poynting's theorem in integral form and can be compared to the work-energy theorem of classical mechanics. That is, the work done by an electromagnetic force on a charge distribution equals the decrease of energy stored in the fields. The volume integral of equation \ref{eq:poyntings_theorem} is the total energy stored in the fields while the surface integral describes the energy flux or rate of transport out of the volume by the fields. We can therefore rewrite equation \ref{eq:poyntings_theorem} as:
\begin{equation}
    \frac{dW}{dt} = -\frac{\partial}{\partial t} \int_V u\, dV -\oint_{\partial V} \bm{S} \cdot d\bm{s}
\end{equation}
where we have set
\begin{equation}
    u = \frac{1}{2}\left(\frac{1}{\mu_0}B^2+\bm{E}\cdot\bm{D}\right),\; \bm{S} = \frac{1}{\mu_0}(\bm{E}\times\bm{B}),
\end{equation}
$u$ is the energy density of the fields and $\bm{S}$ is the energy flux density known as the Poyinting vector. Since we have no free charges for the cases considered in this work, that is free space or neutrally charged media, we set $\frac{dW}{dt}=0$. With equation \ref{eq:emf_work} we get the differential form:
\begin{equation}
    \label{eq:poynting_theorem_diff_form}
    \frac{\partial u}{\partial t} = -\bm{\nabla} \cdot \bm{S},
\end{equation}
because the volume integral (equation \ref{eq:emf_work}) is zero for any volume. We see that \ref{eq:poynting_theorem_diff_form} has the same shape as a continuity equation like the heat flow equation or the continuity equation in quantum mechanics related to the conservation of probability, in this case it shows that energy is conserved locally. Finally, if we take only the electric field contribution to the energy density, that is $u_e = \frac{1}{2}\bm{E}\cdot\bm{D}$, and choose a coordinate system in which the dielectric tensor is diagonal, known as the principal coordinate system, then we can write the electrical displacement vectors for which $u_e$ is constant as:
\begin{equation}
    \frac{D_x^2}{\epsilon_x}+\frac{D_y^2}{\epsilon_y}+\frac{D_z^2}{\epsilon_z}= 2u_e
\end{equation}
where $\epsilon_x$, $\epsilon_y$ and $\epsilon_z$ are the dielectric constants on the diagonal of $\hat{\epsilon}$ called the principal dielectric constants. Additionally we scale $\bm{D}$ by $\nicefrac{1}{\sqrt{2u_e}}$, replace it with $\bm{r}$ and set $n_i^2 = \nicefrac{\epsilon_i}{\epsilon_0},\; (i=x,y,z)$. This gives us:
\begin{equation}
    \frac{x^2}{n_x^2}+\frac{y^2}{n_y^2}+\frac{z^2}{n_z^2}=1,
\end{equation}
which is the equation of an ellipsoid and is therefore called the index ellipsoid. In a uniaxial crystal two of the three principal dielectric constants are equal because of the symmetry plane oriented perpendicular to the optic axis. In that case the index ellipsoid reduces to:
\begin{equation}
    \label{eq:negative_index_ellipsod}
    \frac{x^2+y^2}{n_o^2}+\frac{z^2}{n_e^2}=1.
\end{equation}
Going back to figure \ref{fig:uniaxial_source} we see that the optic axis is along the z-axis and the xy-plane is perpendicular to it. According to convention the symmetry axis is chosen as the z-axis so that $n_x=n_y=n_o$ and $n_z=n_e$. The index ellipsoid can also be used to find the propagation directions of the e- and o-wave as well as the optic axis of a biaxial or uniaxial crystal given a wavevector $\bm{k}$. To do that $\bm{k}$ is placed at the center of the index ellipsoid. A plane normal to $\bm{k}$ is then drawn centered at the origin so that the intersection of the plane and the ellipsoid forms an ellipse. The semi-axes then show the directions of the e- and o-wave and their length is equal to $n_e$ and $n_o$. If the intersection forms a circle instead of an ellipse then it indicates that $\bm{k}$ is parallel to an optic axis, which is then the only optic axis in the case of a uniaxial medium \footnote{Why this method works is shown in section \ref{sec:index_ellipse_proof} of the appendix} \cite{Yariv1984, Griffiths2017}. 
In this section we discussed a physical model based on a classical mechanical description. It explains the observed phenomena of double refraction and the related effects. This model surely only works to some extend and a better model of the interaction between the EM-fields and the charged particles requires a quantum mechanical description. We will see that birefringence does not only have its origin at the atomic level. 

Surprisingly it is also possible for a material to appear birefringent even though all sets of springs are equally stiff in the Lorentz oscillator model. A possible realization of such a material is an infinite medium consisting of alternating layers of two different homogeneous and isotropic media. Even though each individual layer is isotropic the structure as a whole acts as an anisotropic medium. It can be shown that field components parallel and perpendicular to the layering propagate at different speeds. This type of birefringence is called form birefringence. Even though it is a different type the ideas and principles of this section still apply, with the exception of the mechanism causing the birefringence. The derivation and explanation of form birefringence will be the focus of the next section which is based on the works by Rytov \cite{Rytov1956}. 

% TODO @JAN Removed subsection titles and numbering since 3 numbers too much
\subsection{Form birefringence}
\label{sec:form_birefringence}
Form birefringence has the interesting property that the birefringence can be controlled or tuned to some extend directly via the geometry of the medium. As mentioned in the previous subsection \ref{sec:wave_prop} in this subsection we will consider the propagation of an em-wave in a medium consisting of alternating layers. The material of each layer is homogeneous and isotropic with no other restrictions on the optical properties, i.e. the permittivity and permeability are scalars instead of tensors. According to Rytov's work, for sufficiently long wavelengths compared to the structural periodicity of the medium, the periodic layering will then appear anisotropic. So that it in the end is birefringent as well as dichroic if we take absorption into consideration. For each layer we set the refractive index of the layer material as $n_i^2 = \epsilon_i \mu_i$. As mentioned earlier the materials used for the applications in this work are actually non-magnetic, however to keep track of the constants we assign an index to the permeabilities of each layer, even though $\mu_1=\mu_2=\mu_0$. Additionally, we let the width of the layer with $\epsilon_1$ be $a$ and $b$ for the other layer. The derivation is fairly tedious but it is interesting since it shows which approximations and assumptions are made as well as how the wave actually propagates in the structure. The idea of the derivation is to calculate the average field strength over the width $a+b\equiv d$. Therefore the calculation only makes sense if the change of the field is small over a period $d$. This condition can be written as:
\begin{equation}
    \label{eq:rytov_cond1}
    kd|n|\ll 1,
\end{equation}
where n is the effective index of refraction. There are only three different directions in which the wave can propagate in the medium because of symmetry. That is any other direction is equivalent in the sense that it is either mirrored or a superposition of two of the three directions and the field can then be decomposed along those two directions. In two of the cases the wave propagates parallel to the layers and then either the magnetic or the electric field is parallel to the layers as well. In the third case the wave propagates perpendicular to the layers. One of the two cases for parallel propagation is shown in \ref{fig:tikz_rytov_derivation}. This is the geometry that is relevant for the applications we discuss in this work, that is one of the electric field components is perpendicular to the layering and one is parallel. 

\begin{figure}[h]
    \centering
    \includestandalone[scale=1.5]{images/4_chapter04/tikz_rytov_derivation}
    \caption{Infinite finely stratified medium. The medium consists of two isotropic and homogeneous material layers with dielectric constants and permeabilities $\epsilon_1, \mu_1$ (width a) and $\epsilon_2, \mu_2$ (width b). Here we consider the case for a wave propagating parallel to the layers along the x-axis so that one component of the electric field is perpendicular to the layering and the other one is parallel.}
    \label{fig:tikz_rytov_derivation}
\end{figure}

We set the z-axis perpendicular to the layers so that $\bm{k}$ and $\hat{\bm{e}}_z$ are perpendicular. For this case the electric field has two components and the magnetic has one: $\bm{E}=\bm{E_{\bot}} + \bm{E_{\parallel}} = E_x \hat{\bm{e}}_x + E_{z} \hat{\bm{e}}_z$ and $\bm{H}=H_y=H \hat{\bm{e}}_y$. That is we set the y- and z-axis along the directions of $\bm{H}$ and $\bm{E_{\bot}}$ respectively. Again, we start from Maxwell's equations. We use Ampère's law (equation \ref{eq:maxwell3}) and Faraday's law of induction (equation \ref{eq:maxwell4}). Again there are no free charges or currents although now we have that $\bm{B} = \hat{\mu} \bm{H}$. The equations transformed into k-space $(\partial_t \xrightarrow{\mathscr{F}} ik)$ are:
\begin{equation}
    \label{eq:rytov_maxwell_initial}
    \bm{\nabla}\times\bm{E} =-ik \hat{\mu} \bm{H},
    \qquad
    \bm{\nabla}\times\bm{H} = ik \hat{\epsilon} \bm{E}.
\end{equation}

We substitute in the fields and use the fact that both materials are isotropic. This gives us three equations for the field components:
\begin{equation}
    \label{eq:rytov_maxwell}
    \partial_z E_x - \partial_x E_z = - ik \mu H,
    \qquad
    \partial_x H = ik \epsilon E_z,
    \qquad
    \partial_z H = -ik \epsilon E_x.
\end{equation}
Necessarily, since the structure of the material is periodic $\epsilon$ and $\mu$ must be periodic as well. Therefore the solution to the equations in \ref{eq:rytov_maxwell} must be periodic too in $z$ with a period $d$ so a good guess are plane waves:
\begin{equation}
    \label{eq:rytov_field_ansatz}
    H = U(z)e^{-iknx},
    \qquad
    E_x = V(z)e^{-iknx},
    \qquad
    E_z = W(z)e^{-iknx},
\end{equation}
which we substitute into equation \ref{eq:rytov_maxwell} to get a set of equations connecting the amplitudes:
\begin{equation}
    \partial_z V + iknW = -ikn \mu U,
    \qquad
    -Un = \epsilon W,
    \qquad
    \partial_z U = ik \epsilon V.
\end{equation}
For this set of equations there exists analytical solutions:
\begin{align}
\begin{split}
    \label{eq:rytov_amplitudes}
    U=A_j\cos \alpha_jz + B_j\cos \alpha_j & z, 
    \qquad
    V=\frac{-\alpha_j}{ik\epsilon_j}(A_j\sin \alpha_j z -B_j\cos \alpha_j z ),
    \\
    &W=\frac{n}{\epsilon_j}U,
\end{split}
\end{align}
where $\alpha_j = k \sqrt{\epsilon_j\mu_j-n^2}$ and $j$ is 1 $\text{if $z \bmod d \in [0, a)$}$ else $2$. The field components parallel to the layers must be continuous at the interfaces. This implies that the components have to be periodic, since e.g. the interface at $0$ is indistinguishable from the one at $a$. Furthermore, the same conditions apply to the amplitudes because of equation \ref{eq:rytov_field_ansatz}, that is they must be equal at the interfaces $\pm 0$ and $a, -b$:
\begin{align}
\begin{split}
    U(+0) = U(-0), 
    \qquad
    &V(+0) = V(-0),
    \\
    U(a-0) = U(-b+0),
    \qquad
    &V(a-0) = V(-b+0),
\end{split}
\end{align}
where the signs indicate the directions of the limits. We substitute the amplitudes (equation \ref{eq:rytov_amplitudes}) in these boundary conditions so that we get four equations determining the coefficients $A_1,A_2,B_1$ and $B_2$:
\begin{align}
\begin{split}
    \label{eq:rytov_waves}
    A_2 = A_1, 
    \quad
    &A_2\cos \alpha_2 b - B_2 \sin \alpha_2 b= A_1 \cos \alpha_1 a + B_1 \sin \alpha_1 a,
    \\
    B_2=\beta B_1,
    \quad
    &A_2 \sin \alpha_2 b + B_2 \cos \alpha_2 b = -\beta (A_1 \sin \alpha_1 a - B_1 \cos \alpha_1 a),
\end{split}
\end{align}
where $\beta = \frac{\epsilon_2 \alpha_1}{\epsilon_1 \alpha_2}$. A homogeneous system of equations has infinitely many solutions if its determinant is zero. This is the interesting case since otherwise we would be left with the trivial solution only. 
\begin{equation}
   0\overset{!}{=}
   \begin{vmatrix} 
   1 & 0 & -1 & 0 \\
   \cos \alpha_1 a & -\sin \alpha_1 a & \cos \alpha_2 b & -\sin \alpha_2 b \\
   0 & -\beta & 0 & 1 \\
   \beta \sin \alpha_1 a & -\beta \cos \alpha_1 a & \sin \alpha_2 b & \cos \alpha_2 b \\
   \end{vmatrix} 
\end{equation}
calculating this determinant we see that:
\begin{equation}
    (1+\beta^2) \sin \alpha_1 a \sin \alpha_2 b + 2\beta (1-\cos \alpha_1 a \cos \alpha_2 b) = 0,
\end{equation}
i.e. a dispersion relation since $\alpha$ is a function of $n$ and $k$. This equation is quadratic in $\beta$ so we end up with two solutions $\beta_1$ and $\beta_2$:
\begin{equation}
    \beta_1 = -\frac{\tan \frac{\alpha_2 b}{2}}{\tan \frac{\alpha_1 a}{2}},
    \qquad
    \beta_2 = -\frac{\tan \frac{\alpha_1a}{2}}{\tan \frac{\alpha_2 b}{2}}.
\end{equation}
The question is which of these two solutions are of physical interest. To answer that we calculate the mean of the field components over the period $d$ and tryout each of the two possible values for $\beta$. To calculate the mean values we have to split the integral over the full width $d$ into two; one where $z$ goes from $0$ to $a$ and one from $0$ to $-b$. Together with equation \ref{eq:rytov_waves} we can then calculate the ratios of the means which simplifies the expressions. Specifically, we compare the mean field of the perpendicular component to the two parallel mean fields:
\begin{equation}
    \label{eq:rytov_mean_ratios}
    \frac{\overline{U}}{\overline{W}} = \frac{\overline{H}}{\overline{E_z}} = \frac{\epsilon_1 \alpha_2^2 - \epsilon_2 \alpha_1^2}{n\left(\alpha_2^2-\alpha_1^2 \right)},
    \qquad
    \frac{\overline{V}}{\overline{W}} = \frac{\overline{E_x}}{\overline{E_z}} = \frac{\epsilon_1^{-1} - \epsilon_2^{-1}}{k^2\left(\alpha_1^{-2} - \alpha_2^{-2} \right)}P,
\end{equation}
with
\begin{equation}
    P = \frac{U(a-0)-U(+0)}{V(a-0)-V(+0)} = -\frac{ik\epsilon_1}{\alpha_1}\frac{1}{\tan \frac{\alpha_2 b}{2}}\frac{\beta \tan \frac{\alpha_1 a}{2} + \tan \frac{\alpha_2 b}{2}}{\beta \tan \frac{\alpha_2 b}{2} + \tan \frac{\alpha_1 a}{2}}.
\end{equation}
We see that depending on which solution we choose the last factor of $P$ is either $0$ or tends to infinity. Furthermore, if $\beta = \beta_1$ is chosen then because of equation \ref{eq:rytov_mean_ratios} $\overline{E_x}$ is $0$. In that case the mean field is a transverse wave. For $\beta = \beta_2$ it is possible to show that $E_z$ and $H$ are odd functions and $E_x$ is even around the middle of each layer. The mean of $E_z$ and $H$ therefore vanishes and only $E_x$ is left. But $E_z$ and $B$ are odd and so the fields are $0$ in the middle of each layer. This situation is similar to a rectangular waveguide in the sense that the parallel component of the electric field and the normal component of the magnetic field is $0$ at the inner walls. This means that waves with frequencies below the cutoff frequency will be exponentially attenuated and will not propagate far. Since we assume that the wavelength must be greater than the period $d$ (equation \ref{eq:rytov_cond1}) it follows that the frequency is below the cutoff frequency. Consequently, waves for which $\beta = \beta_2$ will not propagate far and therefore $\beta = \beta_1$. The mean fields also satisfy equations \ref{eq:rytov_maxwell} but without the field component along the x-axis:
\begin{equation}
    \label{eq:rytov_maxwell_reduced}
    \partial_x \overline{H} = ik \epsilon_e \overline{E_z},
    \qquad
    \partial_x \overline{E_z} = ik \mu_e \overline{H},
\end{equation}
where $\epsilon_e \mu_e =n^2$ is the effective permittivity of the layers. Substituting in the mean fields we get:
\begin{equation}
    \left(\frac{\overline{U}}{\overline{W}}\right)^2 = \frac{\mu_e}{\epsilon_e}, 
\end{equation} so that with equation \ref{eq:rytov_mean_ratios} we obtain an expression for the effective permittivity and permeability:
\begin{equation}
    \epsilon_e = \frac{\epsilon_1 \alpha_2^2 - \epsilon_2 \alpha_1^2}{\alpha_2^2-\alpha_1^2},
    \qquad
    \mu_e = n^2\frac{\alpha_2^2-\alpha_1^2}{\epsilon_1 \alpha_2^2 - \epsilon_2 \alpha_1^2},
\end{equation}
where $n$ is determined by the first root ($\beta_1$) of the dispersion relation:
\begin{equation}
    \label{eq:rytov_n}
    \frac{\alpha_2}{\epsilon_2}\tan \frac{\alpha_2 b}{2} = -\frac{\alpha_1}{\epsilon_1}\tan \frac{\alpha_1 a}{2}.    
\end{equation}
This would be a good place to stop and use the previous result to determine $n$ but that is easier said than done. The problem is that \ref{eq:rytov_n} is a transcendental equation which often are not solvable in closed form, like in this case. That is also why equation \ref{eq:rytov_cond1} cannot be written in a more explicit form by solving equation \ref{eq:rytov_n} for $n$. There are of course numerical methods which can be used to calculate the roots but that approach introduces other practical issues. A way of dealing with this is by approximating the tangent function by its Taylor series around the origin. Since the error of this approximation increases for larger arguments we require that the arguments of the tangent functions are small:
\begin{equation}
    |\alpha_1 a| \ll 1, 
    \qquad
    |\alpha_1 b| \ll 1.
\end{equation}
It follows from the definition of $\alpha$ that the wavelength must be large compared to $a$ and $b$. The first three terms of the Taylor series are: $\tan x \approx x + \frac{1}{3}x^3 + \frac{2}{15}x^5 + ...$ The first order approximation is therefore obtained by simply replacing tangent by its argument. Solving for $n$ and using $n=\sqrt{\epsilon_e \mu_e}$ we end up with the following first order approximation:
\begin{equation}
    \label{eq:rytov_1st_order}
    \epsilon_{e,0} = \frac{\epsilon_1 \epsilon_2 (a+b)}{a\epsilon_2+b\epsilon_1},
    \qquad
    \mu_{e,0} = \frac{a\mu_1+b\mu_2}{a+b}.
\end{equation}
If we include the second order terms in the expression the approximation results in:
\begin{align}
\begin{split}
    \label{eq:rytov_2nd_order}
    \epsilon_{e} &= \epsilon_{e,0} + \frac{k^2a^2b^2}{12d^2} \frac{a\epsilon_1+b\epsilon_2}{(a+b)\epsilon_1^2\epsilon_2^2}\left(n_1^2-n_2^2\right)\left(\epsilon_1-\epsilon_2\right)\epsilon_{e,0}^3,
    \\
    \mu_{e} &= \mu_{e,0} + \frac{k^2a^2b^2}{12d^2}\frac{a\epsilon_1 + b\epsilon_2}{a\epsilon_2 + b\epsilon_1}\left(n_1^2-n_2^2\right)\left(\mu_1-\mu_2\right).
\end{split}
\end{align}
The $x^5$ term becomes too difficult to solve for $n$, so this is a good place to stop the expansion. The expressions as they are given in equation \ref{eq:rytov_2nd_order} are not directly applicable to the geometries and devices discussed later in this work. Like the geometry we just discussed the structure of the present devices are also layered, but the fields in question have field components parallel and perpendicular to the layering. This geometry is shown in figure \ref{fig:stratified_structure} where the electric field has components $E_{\bot}$ and $E_{\parallel}$ which are respectively perpendicular and parallel to the layering. The geometries shown in figure \ref{fig:tikz_rytov_derivation} and \ref{fig:stratified_structure} are equivalent except that the former has been rotated by $\SI{-90}{\degree}$ around the y-axis. 

\begin{figure}[h]
    \centering
    \includestandalone[scale=2]{images/4_chapter04/tikz_stratified_structure}
    \caption{The same structure as is shown in figure \ref{fig:tikz_rytov_derivation} but from a different point of view and in this case the material is non-magnetic. In this figure the structure is rotated by $\SI{-90}{\degree}$ around the y-axis. In general the electric field has two components parallel to the yz-plane at normal incidence. They can be decomposed into a perpendicular and parallel component. This is useful since the effective permittivity is known along those two directions. The magnetic field vector is not shown since we are only interested in the electric field.}
    \label{fig:stratified_structure}
\end{figure}

We can take any field propagating parallel to the x-axis and decompose it into a component parallel and perpendicular to the layering. Now, since Maxwell's equations are symmetric under the interchange of the electric and magnetic field in regions without free charge or currents, we can go back to the start of the derivation and apply the following transformation: $(\bm{H},\bm{E},\hat{\epsilon},\hat{\mu}) \rightarrow (\bm{E},\bm{-H},\hat{\mu},\hat{\epsilon})$. The derivation and the equations are still valid with this replacement. By doing this we quickly get an expression for the effective permittivity in the direction parallel to the layers. Specifically, we simply replace $\mu_i$ with $\epsilon_i$ and vice versa in equations \ref{eq:rytov_1st_order} and \ref{eq:rytov_2nd_order}. This is also the reason why we assigned each layer a permeability different from $\mu_0$. Finally, because Maxwell's equations are linear it follows that a superposition of two valid solutions is again a valid solution. Using this fact while actually setting $\mu_1=\mu_2=\mu_0$ and then carrying out the replacement described above we obtain an expression for the approximate effective permittivities of the parallel and perpendicular field components:
\begin{align}
\begin{split}
    \label{eq:rytov_2nd_order_final}
    \epsilon_{s} &= \tilde{\epsilon}_{s} + \frac{k^2a^2b^2}{12d^2} \frac{a\epsilon_1+b\epsilon_2}{(a+b)\epsilon_1^2\epsilon_2^2}\left(n_1^2-n_2^2\right)\left(\epsilon_1-\epsilon_2\right){\tilde{\epsilon}_{s}}^3,
    \\
    \epsilon_{p} &= \tilde{\epsilon}_{p} + \frac{k^2a^2b^2}{12d^2}\left(n_1^2-n_2^2\right)\left(\epsilon_1-\epsilon_2\right),
\end{split}
\end{align}
with the first order terms:
\begin{equation}
    \label{eq:rytov_1st_order_final}
    \tilde{\epsilon}_{s} = \frac{\epsilon_1 \epsilon_2 (a+b)}{a\epsilon_2+b\epsilon_1},
    \qquad
    \tilde{\epsilon}_{p} = \frac{a\epsilon_1+b\epsilon_2}{a+b},
\end{equation}
where $s$ and $p$ indicate the perpendicular and parallel directions respectively. If not stated otherwise then equations \ref{eq:rytov_2nd_order_final} and \ref{eq:rytov_1st_order_final} are used throughout this work to calculate the birefringence in the case of form birefringent materials. Additionally, we see that Maxwell's equations at the start of the derivation in this section as well as the boundary conditions are invariant under rotations of the geometry around the z-axis. This means that equations \ref{eq:rytov_2nd_order_final} and \ref{eq:rytov_1st_order_final} are invariant under such rotations as well. The optic axis must therefore be along the z-direction. Furthermore, this means that the effective permeability is $\epsilon_{p}$ throughout the whole xy-plane. Therefore the structure has one optic axis which makes it uniaxial. Moreover, we can write the dielectric tensor in the principal coordinate system of the geometry as follows:
\begin{equation}
    \label{eq:rytov_tensor}
    \hat{\epsilon} = 
    \begin{pmatrix}
        \epsilon_{p} & 0 & 0 \\
        0 & \epsilon_{p} & 0 \\
        0 & 0 & \epsilon_{s} \\
    \end{pmatrix}
\end{equation}
The question is though whether the structure is positive or negative uniaxial. To answer this we need to determine if $n_{s} < n_{p}$. Since the permittivities of the individual layers are complex numbers or rather complex functions the effective permittivities are as well and therefore also the refractive indices. Hence, for the inequality to make sense we only compare the real parts of the refractive indices. To this end we therefore set $(n_r+in_i)^2 \overset{!}{=} \epsilon_r + i\epsilon_i$ where the subscripts $i$ and $r$ indicate the real and imaginary parts respectively. Comparing the coefficients then gives us expressions for the real and imaginary parts of the refractive index:
\begin{equation}
    n_r = \sqrt{\frac{|\epsilon|+\epsilon_r}{2}}, 
    \qquad 
    n_i = \sqrt{\frac{|\epsilon|-\epsilon_r}{2}},
\end{equation}
where $|\cdot|$ denotes the magnitude. The birefringence is therefore given by:
\begin{equation}
    \label{eq:form_bf}
    \Delta n = \operatorname{Re}(n_e) - \operatorname{Re}(n_o) = \sqrt{\frac{|\epsilon_s|+\operatorname{Re}(\epsilon_s)}{2}} - \sqrt{\frac{|\epsilon_p|+\operatorname{Re}(\epsilon_p)}{2}}.
\end{equation}
Similarly we can define the dichroism $\Delta \kappa$ as the difference of the imaginary parts:
\begin{equation}
    \Delta \kappa = \operatorname{Im}(n_s) - \operatorname{Im}(n_p) = \sqrt{\frac{|\epsilon_s|-\operatorname{Re}(\epsilon_s)}{2}} - \sqrt{\frac{|\epsilon_p|-\operatorname{Re}(\epsilon_p)}{2}}.
\end{equation}

Furthermore, in the case of $\epsilon_r^2 \gg \epsilon_i^2$ the refractive index is approximately given by $n^2 \approx \epsilon_r$. It is therefore sufficient to determine if $\epsilon_{s} < \epsilon_{p}$. It is fairly straight forward to show that this inequality is true for the first order, even for arbitrary parameters. For the second order we need to make some reasonable restrictions on the values of the parameters to obtain an answer independent of the value domain. Specifically, the layering of the structures with which we are concerned in this work are alternating air or dielectric. The real part of the permittivity of the dielectric is assumed to be lower than four, which is the case for the materials used in this work. With this it is possible to show that $\epsilon_{s} < \epsilon_{p}$ is true for the second order approximation as well. This is shown in section \ref{sec:bf_proof} of the appendix. If we go back to figure \ref{fig:uniaxial_source} we see that the field component polarized in the direction of the optic axis forms the e-wave which is along the z-axis. Therefore, the e-wave experiences an effective permittivity of $\epsilon_{s}(=\epsilon_{e})$. Analogously, the effective permittivity of the o-wave is $\epsilon_{p}(=\epsilon_{o})$. Since $\epsilon_{s} < \epsilon_{p}$ we can finally conclude that the structure is negative uniaxial, i.e. $n_e < n_o$. Additionally, figure \ref{fig:uniaxial_source} shows that along these two directions the birefringence is maximal. 

Finally this means that we can write the equation describing the index ellipsoid as:
\begin{equation}
    \frac{x^2+y^2}{n_p^2}+\frac{z^2}{n_s^2}=1.
\end{equation}
The shape of this ellipsoid is shown in figure \ref{fig:index_ellipse}. Since the wave propagates along the x-direction we draw an ellipse intersecting the ellipsoid in the yz-plane. Again we see that the two indices of refraction of the electric field components in this direction are $n_{p}$ and $n_{s}$. 
\begin{figure}[h]
    \centering
    \includestandalone[scale=1]{images/4_chapter04/tikz_index_ellipse}
    \caption{Index ellipsoid of the stratified structure shown in figure \ref{fig:stratified_structure}. $\bm{k}$ indicates the direction of propagation. The blue ellipse is perpendicular to this direction and can be used to find the two indices of refraction of the wave; in this case $n_{p}$ and $n_{s}$.}
    \label{fig:index_ellipse}
\end{figure}

In summary, we considered the propagation of an em-wave in an evenly layered medium. To that extend we went through the derivation of an equation for the mean field effective refractive index. Since we considered the spatial average of the field for this derivation, the result is only valid in the case of waves which are long relative to the structural period. I.e. the field variation has to be small compared to the period in order to justify using the mean field. The resulting equation does not in general permit an expression of the effective refractive index in terms of the widths and optical parameters of the layers. Following, a Taylor expansion is carried out and approximate expressions are obtained for the effective permittivities of the mean field components. With these expressions the birefringence of the structure can be approximated, where the error of the approximation increases for increasing wavelength. Furthermore, we saw that the stratified structures with which we are concerned appear uniaxial. In total, we considered propagation within the stratified medium only, following the next section will be about what happens at the boundary between two media with different dielectric tensors.

% TODO @JAN You don't have to read this.
%I'm not sure wether to keep it or not. I don't really use it. It's almost like the Fresnel equations anyways. Don't know ... Dump it into the appendix?
% (if removed rewrite last sentence of previous part, move ending of this part) 
\subsection{Interface conditions}
\label{sec:interface_conditions}
The previous section answered the question about how each field component propagates inside the stratified medium in the case of long waves. The question that still remains is what happens with the wave as it enters the medium. There are three cases: Either one medium is isotropic and one is anisotropic or in the second case both media are anisotropic. The third case is less interesting since it is already described by the Fresnel equations. Furthermore, we will only consider the case of normal incidence, since that is given by the geometry of the measurements in this work. We do not have to consider double refraction since we know that the o-wave travels at the same speed in every direction as if the medium was isotropic. The o-wave therefore obeys Snell's Law and will not be refracted at normal incidence. Similarly, since the e-wave is everywhere parallel to the optic axis it will also effectively experience an isotropic medium and the refraction can therefore be described by Snell's Law. In contrast to the calcite crystal example described in section \ref{sec:wave_prop} we will therefore not get two spatially separated waves. However, we will still get a reflected wave. In the case of an interface between two isotropic materials we could use the Fresnel equations but since the material in this case appears anisotropic we have to take this into account in the derivation of the transmission and reflection coefficients. The geometry is shown in figure \ref{fig:interface_derivation} where the wave ($\bm{k_i}$) is normally incident on the interface between two anisotropic media. According to the wave equation we will in general get a reflected ($\bm{k_r}$) and a transmitted wave ($\bm{k_t}$). 

\begin{figure}[h]
    \centering
    \includestandalone[scale=2.7]{images/4_chapter04/tikz_interface_derivation}
    \caption{A wave with wavevector $\bm{k_i}$ which is normally incident on the boundary between two anisotropic media. The wave propagates along the x-direction. The general solution to the wave equation allows a wave propagating in both positive and negative x-direction; a transmitted and reflected wave with wavevectors $\bm{k_t}$ and $\bm{k_r}$ respectively.}
    \label{fig:interface_derivation}
\end{figure}

To begin with, we derive conditions relating the parallel and normal field components at both sides of any neutrally charged non-conducting interface. These conditions will be useful for the calculation of the reflection and transmissions coefficients at the interface between two anisotropic media. Once more we start with Maxwell's equations (\ref{eq:maxwell1reduced}-\ref{eq:maxwell4reduced}) specifically Faraday's law of induction:
\begin{equation}
    \bm{\nabla}\times\bm{E} =-\frac{\partial\bm{B}}{\partial t}.
\end{equation}
Using Stokes' theorem can quickly obtain Faraday's law in integral form:
\begin{equation}
    \label{eq:int_faradays_law}
    \int_A \bm{\nabla}\times\bm{E}\cdot d\bm{A} = \oint_{\delta A} \bm{E}\cdot d \bm{l} = -\int_A \frac{\partial\bm{B}}{\partial t}\cdot d\bm{A},
\end{equation}
where $\delta A$ is the boundary of the area $A$. We set this surface $A$ in the plane of incidence, in this case the xy-plane, so that one half is in medium one and the other in medium two. If we let the sides perpendicular to the interface of area $A$ go to zero then the total area will go to zero and the right hand side of equation \ref{eq:int_faradays_law} will end up as zero. With this we therefore get that:
\begin{equation}
    \int_{l_{1,t}} E_{1,t} dl - \int_{l_{2,t}} E_{2,t} dl = 0,
\end{equation}
where the $t$ subscript denotes the field components and paths tangential to both sides of the interface. Since  we did not choose any specific interface or electric field we can omit the integration. This gives us the continuity condition for the tangential components of the electric field at the interface:
\begin{equation}
    \label{eq:t_boundary_cond_e}
    \bm{n} \times (\bm{E_1} - \bm{E_2}) = 0,
\end{equation}
where $\bm{n}$ is normal to the interface, $\bm{E_1}$ and $\bm{E_2}$ are the electric fields in medium one and two respectively. Furthermore, since we have no free currents and the materials are non-magnetic we can derive a condition for the tangential components of the magnetic field in a similar manner. We use Ampère's circuital law (\ref{eq:maxwell4reduced}) which we write it in integral form. Subsequently we shrink the integration area around the interface. This yields:
\begin{equation}
    \label{eq:t_boundary_cond_b}
    \bm{n} \times (\bm{B_1} - \bm{B_2}) = 0,
\end{equation}
where $\bm{B_1}$ and $\bm{B_2}$ are the magnetic fields in medium one and two respectively.
Again, to obtain interface conditions for the normal components of the fields we apply the same procedure. We start with Gauss's law for magnetism $(\bm{\nabla}\cdot \bm{B} = 0)$. If we integrate over a volume enclosing the interface, and then shrink this volume we get a condition for the normal component of the magnetic field at the boundary:
\begin{equation}
    \label{eq:n_boundary_cond_b}
    \bm{n}\cdot(\bm{B_1}-\bm{B_2}) = 0.
\end{equation}
I.e. the normal component of the magnetic field is continuous across the interface. The final general condition, obtained in a similar manner, is the condition that the normal component of the displacement field $\bm{D}$ is continuous across the interface. Which can be expressed as:
\begin{equation}
    \label{eq:n_boundary_cond_d}
    \bm{n}\cdot(\bm{D_1}-\bm{D_2}) = 0,
\end{equation}
where $\bm{D_1}$ and $\bm{D_2}$ are the displacement fields at each side of the interface \cite{Destriau1949, Griffiths2017}. Specifically, in the context of this work we are interested in the reflection and transmission coefficients of the interface between two periodically stratified media 1 and 2. In the previous section $\ref{sec:wave_prop}$ about form birefringence we derived approximations for the dielectric tensors of such periodic media given the individual layer material and thickness. Additionally, in the case we consider the two anisotropic media on each side of the interface are rotated at some angles $\alpha_1$ and $\alpha_2$ around a common x-axis. This means that in a third system the dielectric tensors of both media given by equation \ref{eq:rytov_tensor} will not be diagonal. We could use the principal coordinate system of one media but since there in general are multiple interfaces we instead use a lab system as the reference system. This system is the one shown in figure \ref{fig:interface_derivation}. The plane of incidence is therefore the xy-plane and the interface lies in the yz-plane, parallel to the layering. In the lab system $\hat{\epsilon}$ is given by:
\begin{equation}
    \label{eq:lab_epsilon}
    \hat{\epsilon_i}(\alpha_i) = \hat{R}_x(\alpha_i) \hat{\epsilon_i}' \hat{R}_x(-\alpha_i),
\end{equation}
where $\hat{\epsilon_i}'$ is the dielectric tensor in its principal coordinate system, $i=1,2$ and $\hat{R}_x(\alpha_i)$ rotates a vector around the x-axis by an angle $\alpha_i$. $\hat{R}_x(\alpha_i)$ can be represented as:
\begin{equation}
    \label{eq:x_rotation_matrix}
    \hat{R}_x(\alpha_i) = 
    \begin{pmatrix}
        1 & 0 & 0 \\
        0 & \cos (\alpha_i) & -\sin (\alpha_i) \\
        0 & \sin (\alpha_i) & \cos (\alpha_i) \\
    \end{pmatrix}
\end{equation}
\cite{J.Muthsam2006}.
For clarification, the dielectric tensors of media 1 and 2 are different not only because of the different angles $\alpha_i$ but also because of different diagonal entries, e.g $\epsilon_{p,1} \neq \epsilon_{p,2}$. With this description of the setup we can proceed with the derivation of the reflection and transmission amplitudes. Respectively, we can represent the incident, reflected and transmitted electric field as follows:
\begin{align}
\begin{split}
    \bm{E_i} = \bm{E}_{0,i}e^{i(\bm{k}_i\bm{r}-\omega_i t)},
    \quad
    \bm{E_r} = \bm{E}_{0,r}e^{i(\bm{k}_r\bm{r}-\omega_r t)},
    \quad
    \bm{E_t} = \bm{E}_{0,t}e^{i(\bm{k}_t\bm{r}-\omega_t t)}.
\end{split}
\end{align}
We get similar expressions for the magnetic field, simply by replacing $E$ with $B$. 
The boundary conditions on the interface that the tangential components of the electric and magnetic field are continuous, i.e. equations \ref{eq:t_boundary_cond_e} and \ref{eq:t_boundary_cond_b}, requires that
\begin{align}
\begin{split}
    \label{eq:field_conditions}
    \bm{k}_i \cdot \bm{r} = \bm{k}_r \cdot& \bm{r} = \bm{k}_t \cdot \bm{r}, \\
    \omega_i = \omega_r& = \omega_t,
\end{split}
\end{align}
\begin{align}
\begin{split}
    \label{eq:interface_amplitudes}
    \bm{n} \times (\bm{E}_{i} + \bm{E}_{r}) = \bm{n} \times \bm{E}_{t}, \\
    \bm{n} \times (\bm{B}_{i} + \bm{B}_{r}) = \bm{n} \times \bm{B}_{t}, \\
\end{split}
\end{align}
because all exponents must be equal for any $\bm{r}$ and $t$. The number of expressions in equation \ref{eq:field_conditions} show that conditions \ref{eq:t_boundary_cond_e} and \ref{eq:t_boundary_cond_b} are quite strong. Furthermore, since the wavevectors are given as:
\begin{equation}
    \bm{k}_i = k_{x,i} \bm{n}, \qquad \bm{k}_r = -k_{x,i} \bm{n} \qquad \bm{k}_t = k_{x,t} \bm{n},
\end{equation}
it follows that:
\begin{align}
\begin{split}
    &k_{x,i} = k_{x,r} = k = \sqrt{\hat{\epsilon}_{p,1}}\frac{\omega}{c}, \\
    &k_{x,t} = k_t = \sqrt{\hat{\epsilon}_{p,2}}\frac{\omega}{c}.
\end{split}
\end{align}
The last equation arises from the fact that the rotation is around the x-axis. Equation \ref{eq:lab_epsilon} also shows this; there is a mismatch between the permittivities of the two media along the x-axis. With Faraday's law we can obtain a system of equations relating the electric and magnetic fields of the incident, reflected and transmitted wave:
\begin{equation}
    \label{eq:some_equation}
    \bm{n} \times \bm{B}_i = -\frac{\omega}{k} \hat{\epsilon}_1 \bm{E}_i, \qquad
    \bm{n} \times \bm{B}_r = \frac{\omega}{k} \hat{\epsilon}_1 \bm{E}_r, \qquad
    \bm{n} \times \bm{B}_t = -\frac{\omega}{k_t} \hat{\epsilon}_2 \bm{E}_t.
\end{equation}
If we add the first two expressions of equation \ref{eq:some_equation} and use the fact that the tangential component of the magnetic field is continuous across the interface (equation \ref{eq:interface_amplitudes}) we get:
\begin{equation}
    \bm{n} \times \bm{B}_t = -\frac{\omega}{k}\hat{\epsilon}_1(\bm{E}_{i}-\bm{E}_{r}).
\end{equation}
With the last expression of equation \ref{eq:some_equation} we can eliminate the magnetic field in the previous expression:
\begin{equation}
    \label{eq:electric_field_eq}
    \frac{\omega}{k_t}\hat{\epsilon}_2 \bm{E}_t = \frac{\omega}{k}\hat{\epsilon}_1(\bm{E}_{i}-\bm{E}_{r}).
\end{equation}
Or if we set $n_{p,i} = \sqrt{\epsilon_{p,i}}$ then we can rewrite equation \ref{eq:electric_field_eq} as:
\begin{equation}
    \label{eq:electric_field_eq_rew}
    \frac{n_{p,1}}{n_{p,2}}\hat{\epsilon}_2 \bm{E}_t = \hat{\epsilon}_1(\bm{E}_{i}-\bm{E}_{r}).
\end{equation}

Finally, if we pair the interface condition for the tangential component of the electric field with equation \ref{eq:electric_field_eq} we obtain a homogeneous linear system of equations in the electric field components. These equations can be reduced to expressions which relate the incident electric field amplitudes to the reflected and transmitted amplitudes. Furthermore, we know from the previous section \ref{sec:wave_prop} that the field components along the direction of propagation are extinguished, i.e. a TEM wave. This means that we can omit the x-components in the expressions. With this we can write the relation between the incident and reflected field as:
\begin{equation}
    \label{eq:reflection_general}
    \bm{E_r} = \hat{T}_r \bm{E_i} =
    \frac{-1}{d}
    \begin{pmatrix}
        b_-b_+ - c_+a_- & c_-b_+ - c_+b_- \\
        b_+a_- - a_+b_- & b_+b_- - a_+c_- \\
    \end{pmatrix}
    \bm{E_i},
\end{equation}
we get a similar relation between the incident and transmitted field:
\begin{equation}
    \label{eq:transmission_general}
    \bm{E_t} = \hat{T}_t \bm{E_i} =
    \frac{2}{d}
    \begin{pmatrix}
        a_1c_+ - b_1b_+ & b_1c_+ - c_1b_+ \\
        b_1a_+ - a_1b_+ & c_1a_+ - b_1b_+ \\
    \end{pmatrix}
    \bm{E_i},
\end{equation}
where $a_i=\cos^2(\alpha_i)\epsilon_{p,i}+\sin^2(\alpha_i)\epsilon_{s,i}$, $b_i=\frac{1}{2}\sin(2\alpha_i)(\epsilon_{p,i}-\epsilon_{s,i})$, $c_i = a_i(\alpha_i + \nicefrac{\pi}{2})$ with $i=1,2$ for medium 1 and 2 respectively, $x_{\pm}=x_1\pm \frac{n_{p,1}}{n_{p,2}} x_2, x=a,b,c$ and finally $d=a_+c_+-b_+b_+$. We can consider special cases to check the plausibility of the two expressions \ref{eq:transmission_general} and \ref{eq:reflection_general}. To that extend we first consider the case in which both media are isotropic dielectrics. We therefore set $\hat{\epsilon}_1$ and $\hat{\epsilon}_2$ equal to the identity matrix times $\epsilon_1$ and $\epsilon_2$ respectively. Furthermore, since in this case both media are isotropic we can set $\alpha_1 = \alpha_2 = 0$. This reduces equations \ref{eq:reflection_general} and \ref{eq:transmission_general} to the following:
\begin{equation}
    \bm{E_r} = \frac{\epsilon_1-\frac{n_1}{n_2}\epsilon_2}{\epsilon_1+\frac{n_1}{n_2}\epsilon_2}\bm{E_i}, \qquad \bm{E_t} = \frac{2\epsilon_1}{\epsilon_1+\frac{n_1}{n_2}\epsilon_2}\bm{E_i},
\end{equation}
which with $\epsilon_i=n_i^2, i=1,2$ are the Fresnel equations for normal incidence. The next case we will consider is the interface between an isotropic medium, e.g. air, and a stratified medium. This means that the dielectric tensor of the first medium is $\epsilon_0$ times the identity matrix and for the second medium it is given by equation \ref{eq:lab_epsilon}. Again we can set the angles to zero. This is possible because the first medium is isotropic so that we can set the lab frame to align with the principal axes of medium 2. With this we can simplify equations \ref{eq:reflection_general} and \ref{eq:transmission_general}, resulting in:
\begin{equation}
    \bm{E_r} =
    \begin{pmatrix}
        \frac{1-n_p}{1+n_p} & 0 \\
        0 & \frac{n_p-n_s^2}{n_p+n_s^2} \\
    \end{pmatrix}
    \bm{E_i}, \qquad
    \bm{E_t} =
    \begin{pmatrix}
        \frac{2}{1+n_p} & 0 \\
        0 & \frac{2n_p}{n_p+n_s^2} \\
    \end{pmatrix}
    \bm{E_i},
\end{equation}
which agrees with the solution to a similar problem given in \cite{Lim1993}. Furthermore, in contrast to the previous case, we see that in this case need to distinguish between the two polarizations states. The final case we will consider is where both media are stratified so that their dielectric tensors are described by equation \ref{eq:lab_epsilon}. To be able to reduce equations \ref{eq:reflection_general} and \ref{eq:transmission_general} we limit ourselves to the case in which the principal axes of the two media in some way align with the axes of the lab system. In other words the angles $\alpha_1$ and $\alpha_2$ are a multiple of $\nicefrac{\pi}{2}$. In this case we get the following for equations \ref{eq:reflection_general} and \ref{eq:transmission_general}:
\begin{equation}
    \bm{E_r} =
    \begin{pmatrix}
        \frac{n_{p,2}n_{s,1}^2-n_{p,1}n_{s,2}^2}{n_{p,2}n_{s,1}^2+n_{p,1}n_{s,2}^2} & 0 \\
        0 & \frac{n_{p,1}-n_{p,2}}{n_{p,1}+n_{p,2}} \\
    \end{pmatrix}
    \bm{E_i},
\end{equation}
\begin{equation}
    \bm{E_t} =
    \begin{pmatrix}
        \frac{2n_{p,2}n_{s,1}^2}{n_{s,1}^2n_{p,2}+n_{p,1}n_{s,2}^2} & 0 \\
        0 & \frac{2n_{p,1}}{n_{p,1}+n_{p,2}} \\
    \end{pmatrix}
    \bm{E_i},
\end{equation}
these two equations would further reduce to the Fresnel equations in the case of two isotropic media since then $\epsilon_{p,i}=\epsilon_{s,i}$. Furthermore, only if the axes misalign do we get off-diagonal entries different from zero and then the two field components $E_y, E_z$ no longer separate. Additionally, the last two cases have shown that only in the case of isotropic media are the reflection and transmissions coefficients independent of the polarization state. In other words, they are equal for both the y- and z-component only when the media are isotropic. To summarize this section, we have seen that the propagating modes in periodically stratified media are transverse electromagnetic modes, since the electric and magnetic component in the direction of propagation is exponentially attenuated. Moreover, these structures show a special type of birefringence called form birefringence. Therefore, if we take into account the absorption of the constituent materials, then such a structure is dichroic as well. In addition to dichroism, we have also seen that an effect with similar consequences appears at the interface of such a stratified structure and any other medium. Specifically, in this case the transmitted and reflected fields depend on the polarization of the incident field. This means that in total we consider three effects; transmission and dichroism which change the amplitude of each field component by a different amount and birefringence which causes the phase of one field component to evolve faster than the other resulting in a relative phase change. All three effects change the polarization state as we have seen in section \ref{sec:polellipse}. In the next section we will describe an optical element called a waveplate or retarder. The waveplates which we are concerned with consist of an anisotropic material with a certain thickness. The purpose of a waveplate is to cause a targeted polarization change of a wave traveling through it. Therefore, in order to obtain the correct final polarization state, these previously described effects must be taken into account. 

% TODO Remove mention of interface effects. (Think I did -> recheck)
\section{Waveplates}
\label{sec:waveplates}
To explain how a waveplate works we assume that the materials are non-absorbing and neglect interface effects. These effects will be taken into consideration later on in the mathematical description. In the case where we only consider birefringence the operation of a waveplate or retarder is fairly simple. One of the two polarization states of the wave is caused to have its phase evolve faster compared to the other state. Therefore, depending on the distance the wave travels in the waveplate, the amount one field component is lagging behind the other can be controlled. Using this principle it is in theory possible to convert any initial polarization state into any other state. This is shown in figure \ref{fig:waveplate_conversions}, where the positive values indicate that $E_y$ leads $E_z$ and similarly the negative values show that $E_y$ lags behind $E_z$ by the indicated amount. The waveplates in the context of this work consists of uniaxial materials i.e. they have a single optic axis. Therefore, if the $E_y$ component aligns with the fast axis, which is the axis with the lowest refractive index of the waveplate, then it is advanced compared to the $E_z$ component. Likewise, if the $E_y$ component is aligned with the slow axis of the waveplate then the $E_z$ component is advanced. Similarly, the slow axis is the axis or direction with the highest refractive index. The phase difference between two adjacent states in figure \ref{fig:waveplate_conversions} is $\frac{\pi}{4}$. Going clockwise this phase shift is subtracted and vice versa in the counterclockwise direction, so that eight steps is a full cycle. For example, if linearly polarized light with both components in phase is sent through a waveplate with its fast axis along the y-axis then the state will be shifted counterclockwise using the convention that $\delta=\delta_y-\delta_x$. This means that a phase shift of $-\frac{\pi}{2}$ causes \SI{45}{\degree} linear polarized light to emerge LCP and RCP for a phase shift of $+\frac{\pi}{2}$. Likewise, a phase shift of $\pm \pi$ causes the light to remain linearly polarized but with its plane of polarization rotated by \SI{90}{\degree}.

\begin{figure}[h]
    \centering
    \includestandalone[scale=1.3]{images/4_chapter04/tikz_waveplate_conversions}
    \caption{The different polarization states when the y-component leads or lags behind the z-component by the indicated phase shifts $\pm \Delta$. Each step in the counterclockwise or clockwise direction subtracts or adds a phase shift of $\frac{\pi}{4}$ respectively.}
    \label{fig:waveplate_conversions}
\end{figure}

If we assume that the incident wave has components parallel and perpendicular to the optic axis of the waveplate, then two separate waves will emerge as we have seen in the previous section, that is the e- and o-wave. The phase speed of the e-wave is higher than that of the o-wave. Furthermore, since we in this case have no double refraction the resulting wave will be a superposition of the e- and o-wave, which after passing through the waveplate will have a phase difference of $\Delta$ known as the retardance of the waveplate. For a waveplate of thickness $d$ the optical path difference is then given by:
\begin{equation}
    \Lambda = d|n_o - n_e|
\end{equation}
with $\Delta = k_0 \Lambda$ we therefore get the following expression for the phase difference:
\begin{equation}
    \label{eq:wp_eq}
    \Delta = \frac{2\pi}{\lambda_0}d|n_o - n_e|,
\end{equation}
where $\lambda_0$ is the wavelength of the wave in free space. There are three special types of waveplates: full, half and quarter waveplates. The full-wave plate causes a phase shift of $2\pi$. That means the retardation is one wavelength or a full cycle in figure \ref{fig:waveplate_conversions}. The initial polarization state is therefore not altered. It is important to emphasize that this is only the case for monochromatic light. Since the materials in general are dispersive and because $\Delta$ directly varies as $\lambda_0^{-1}$. The condition that $\Delta = 2\pi$ or any other value can therefore only be fulfilled for monochromatic waves. This means in general that waveplates are chromatic which is the main problem with which we are concerned in this work. Full waveplates can therefore be used as narrow-wavelength filters if they are placed in between two crossed polarizers. Only the frequency component for which $\Delta$ is $2\pi$ will not have a component parallel to the final polarizer. 

A half waveplate causes a phase shift of $\pi$ between the e- and o-wave. Furthermore, we assume in the following that the initial polarization is linear and the plane of polarization is at some angle $\theta$ relative to the fast axis of the waveplate. As the name suggests, this phase shift causes one component to be delayed by half a wavelength relative to the other component. In the case of a positive material this means that the e-wave is delayed by half a wave and the plane of polarization will be rotated by $2\theta$. An example of a half waveplate is shown in figure \ref{fig:half_waveplate}. In this example the material of the waveplate is positive. Since then $n_o<n_e$ so that the e-wave has a shorter wavelength and is delayed relative to the o-wave. For a given wavelength and material it follows from equation \ref{eq:wp_eq} that $d$ must be equal to $\frac{1}{2}\frac{\lambda_0}{|n_o - n_e|}$. As we saw earlier, a phase shift of $2\pi$ leaves the polarization state of the light unaffected. 

\begin{figure}[h]
    \centering
    \includestandalone[scale=1.63]{images/4_chapter04/tikz_half_waveplate}
    \caption{A half waveplate consisting of a positive material. The e-wave is delayed relative to the o-wave since it propagates slower. Adjusting the thickness $d_{\lambda/2}$ gives a final relative phase shift of $\pi$. This phase shift causes the polarization plane to be rotated by $2\theta$, where $\theta$ is the angle between the fast axis and the polarization plane of the waveplate. The dots along the axis of propagation indicate half a wavelength of the similar colored wave.}
    \label{fig:half_waveplate}
\end{figure}

It therefore follows from equation \ref{eq:wp_eq} that a waveplate with an additional thickness of $d_{\lambda}=\frac{\lambda_0}{|n_o - n_e|}$ still functions as a half waveplate. And with the same reasoning, an additional thickness of $md_{\lambda}$ does not change the final polarization state, where $m$ is a whole number. The thickness of a half waveplate is therefore given by:

\begin{equation}
    \label{eq:thickness_half_waveplate}
    d_{\lambda/2} = \frac{(2m+1)\lambda_0}{2|n_o - n_e|}, \quad m=0,1,2,... 
\end{equation}

Furthermore, the effect that a half waveplate rotates the plane of polarization by $2\theta$ is also true for other elliptical polarization states. This is because a half waveplate always shifts one component by half a wave relative to the other component. Additionally, the handedness of the polarization state is inverted. We see this from figure \ref{fig:waveplate_conversions} where the effect of a half waveplate corresponds to half a rotation in the diagram. In other words, the final state is always on the opposite side of the initial state in the diagram. It is worth noting at this point, that linearly polarized light for which the polarization plane aligns with the slow or fast axis of any waveplate is unaffected by a waveplate, since in that case the light only has one component.

The final of the three special waveplate types is the quarter waveplate. Again, as the name suggests it shifts the relative phase by a quarter period or $\frac{\pi}{2}$. For linearly polarized light at an angle of $\frac{\pi}{4}$ to either the slow or fast axis of a quarter waveplate, the light is converted into circular polarized light. An example of this setting is shown in figure \ref{fig:quarter_waveplate}. Because of the fact that the amplitudes of the components along the fast and slow axis initially are equal it follows that the final polarization state is circular. The same reasoning can be applied to circular light, which a quarter waveplate converts into linearly polarized light. Again, figure \ref{fig:waveplate_conversions} also gives an overview of some of the different possible conversions caused by a quarter waveplate. Specifically, a quarter waveplate shifts the polarization state a quarter cycle in the diagram. Similar to a half waveplate we get that the thickness $d_{\lambda/4}$ of a quarter waveplate is given by:

\begin{equation}
    \label{eq:thickness_quarter_waveplate}
    d_{\lambda/4} = \frac{(4m+1)\lambda_0}{4|n_o - n_e|}, \quad m=0,1,2,... 
\end{equation}

The number $m$ is known as the order of the waveplate. An order of $0$ means that it is the thinnest possible realization of a specific waveplate type. These waveplates are referred to as zero-order waveplates. Likewise, multiple-order waveplates have thicknesses that are a multiple of a full $2\pi$ phase shift plus the type specific phase shift. Evidently, for multiple-order waveplates $m$ is larger than zero. Waveplates are often multiple-order. For example a quartz quarter waveplate with a birefringence of $0.0092$ would have a thickness of only \SI{15}{\micro \meter} for a wavelength of \SI{550}{\nano \meter}. This makes it fragile as well as difficult to produce in contrast to a thicker multiple-order quarter waveplate. The half and quarter waveplates shown in figure \ref{fig:half_waveplate} and \ref{fig:quarter_waveplate} respectively, are examples of zero-order waveplates. The frequencies with which we are concerned in this work are in the higher GHz and the lower THz range. Consequently, already a zero-order quarter or half waveplate designed for these frequencies will have a thickness in the lower millimeter or centimeter range. A part of the problem is therefore to reduce the order of the designed waveplates, i.e. to keep them as thin as possible. This has to be taken into consideration since for higher orders the waveplates quickly become fairly thick which makes them impractical mainly due to high absorption losses \cite{Hecht}.

\begin{figure}[h]
    \centering
    \includestandalone[scale=1.55]{images/4_chapter04/tikz_quarter_waveplate}
    \caption{\SI{45}{\degree} Linearly polarized light incident on a quarter waveplate consisting of a positive material. In this case the slow axis is therefore along the optic axis. The thickness $d_{\lambda/4}$ is chosen so that the final phase difference between the e-wave and the o-wave is $\pi/2$. The final state is LCP since the material is positive, for a negative material it would be RCP. The colored dots indicate the wavelength decrease.}
    \label{fig:quarter_waveplate}
\end{figure}

It is clear that it would quickly become complicated to predict the final polarization state of a polarized wave, after it has passed a series of waveplates and polarizers. Especially the way we have done it so far by considering polarization in terms of the individual field components. In the following section we will therefore introduce a better suited alternative method of describing the polarization and changes thereof. We will see that any polarization state can be described by a vector and each optical element by a matrix. This method reduces most calculations to matrix multiplications, which is especially useful for solving problems consisting of a series of optical elements.

\section{Mathematical description of polarization}
\label{sec:math_desc}
The two most widely used representations of polarized light are Jones vectors and Stokes parameters. Stokes parameters were already introduced in 1852 by G. G. Stokes and are essentially four real numbers derived from measurable quantities. These four numbers are often combined into a vector known as the Stokes vector. The Jones vector representation was invented almost 100 years later in 1941 by R. Clark Jones. It is only applicable to polarized light since the components of the Jones vector are given directly by the electric field components. The advantage of using Stokes vectors to describe the polarization state of light is therefore that they can be used in the case of natural, totally polarized and partially polarized light. Nevertheless, both representations can replace conventional algebraic and trigonometric calculation methods, where the latter is required when the optic axes are aligned at different angles in a series of elements. Together with the matrices describing the individual optical elements the Stokes and Jones vectors form the Mueller and Jones calculus \cite{Shurcliff1962}. 

% TODO Removed the subsection ...
\subsection{Stokes parameters}
\label{sec:stokes_parameters}
As mentioned; one of the advantages of the Stokes vectors are that they can describe any type of light. Since in the context of this work the light is always fully polarized we do not actually need this property and the Jones calculus will be sufficient. Nevertheless, the Stokes vectors will be helpful later on where they simplify the description and introduction of the Poincaré sphere, which is a useful tool for visualizing the effects of polarization altering elements and the states themselves. At first we will show how the Stokes parameters can be obtained directly from observables, then how they relate to the electric field. The Stokes parameters are a set of four numbers, which describe the intensity and polarization state of light. The parameters have the unit of intensity, this intensity is not time dependent but instead averaged over the period $T$ of a measurement. A direct consequence of this is that we lose the phase information, which limits the use of the Stokes parameters for interference effect calculations. Each parameter has a filter or polarizer $F_0$ to $F_3$ associated with it which transmits and blocks half of the light equally. The following configuration of the filters is just one way of defining the parameters and is not unique, that is other equivalent configurations exist as well. In any case the filters associated with the four parameters $s_0$ to $s_3$ are a neutral, two linear and a circular polarizer respectively. Where a neutral filter lets any polarization pass, in other words for any polarization state the initial intensity is equal to the final intensity after passing a neutral filter, it does not do anything. Linear polarizers essentially function in the same way as a wiregrid polarizer, rather a wiregrid polarizer is a linear polarizer. A circular polarizer consists of a linear polarizer and a quarter waveplate where the linear polarizer is oriented at an angle of \SI{45}{\degree} to the principal axes of the waveplate. The light will therefore always emerge circular polarized. The transmission axes of $F_1$ and $F_2$ are oriented horizontally and at \SI{45}{\degree} respectively while $F_3$ is opaque to LCP light. The filters are then placed one at a time in the beam path and the transmitted intensities $I_0$ to $I_3$ are measured for each filter. The Stokes parameters can then be calculated using the following combinations:

\begin{equation}
    \label{eq:stokes_param_int}
    s_0 = 2 I_0, \quad 
    s_1 = 2 I_1 - 2 I_0, \quad 
    s_2 = 2 I_2 - 2 I_0, \quad 
    s_3 = 2 I_3 - 2 I_0.
\end{equation}

We see that $s_0$ is the total intensity $I$ of the light since $F_0$ is opaque to all states, while $s_1$, $s_2$ and $s_3$ actually describe the polarization state of the light. The sign of $s_1$ therefore shows if the horizontal or the vertical linear polarization component is dominant. If $s_1$ is zero and we assume the light is not unpolarized then the light is either \SI{\pm 45}{\degree} elliptical or circular polarized. Likewise, if respectively $s_2>0$ or $s_2<0$ then the \SI{45}{\degree} or the \SI{-45}{\degree} linear polarization component is dominant. Like for $s_1$ and $s_2$, the sign of $s_3$ indicates which of the two circular polarization states is dominant, $s_3>0$ for RCP and $s_3<0$ for LCP. Again, if the respective parameter is zero then there is no preference for that polarization type. As mentioned earlier the Stokes parameters can also be related to the electric field. For that we use the expression for quasimonochromatic light:
\begin{align}
\begin{split}
    \label{eq:quasi_mono_efields}
    &\bm{E}_x(t) = \bm{e_x}E_{0x}(t)\cos [(\Bar{k}z-\Bar{\omega} t) + \delta_x(t)], \\
    &\bm{E}_y(t) = \bm{e_x}E_{0y}(t)\cos [(\Bar{k}z-\Bar{\omega} t) + \delta_y(t)],
\end{split}
\end{align}
where the bar superscript indicates averaging, $\bm{e_x}$ and $\bm{e_y}$ are the unit vectors in the x- and y-directions respectively. On a side note, the quasi prefix is added because waves in reality do not have an infinite extend and so the Fourier transform is not quite a sharp delta peak. The total wave packet can be regarded as a traveling wave of frequency $\Bar{\omega}$, known as the carrier. $\Bar{k}$ and $\Bar{\omega}$ are therefore known as the spatial and temporal carrier frequencies. 
With $\bm{E}(t)=\bm{E}_x(t)+\bm{E}_y(t)$ we can rewrite the Stokes parameters as:
\begin{align}
\begin{split}
    \label{eq:stokes_param_e}
    &s_0 = \langle E_{0x}^2 \rangle_T + \langle E_{0y}^2 \rangle_T, \\
    &s_1 = \langle E_{0x}^2 \rangle_T - \langle E_{0y}^2 \rangle_T, \\
    &s_2 = 2\langle E_{0x}E_{0y}\cos \delta \rangle_T, \\
    &s_3 = 2\langle E_{0x}E_{0y}\sin \delta \rangle_T,
\end{split}
\end{align}
where the angled brackets denote time averaging and $\delta=\delta_y-\delta_x$ as in section \ref{sec:polellipse} about the polarization ellipse. In fact these relations can also be obtained by time averaging equation \ref{eq:pol_ellipse} which describes the polarization ellipse. Additionally, the constant $\epsilon_0 c / 2$ has been left out so that the parameters are not equal but instead proportional to the intensities. In theory, for monochromatic light we could leave out the averaging since then the amplitudes and the phases are time independent. If the light is unpolarized then $\langle E_{0x}^2 \rangle_T = \langle E_{0y}^2 \rangle_T$ and $s_1$ is zero. $s_2$ and $s_3$ are zero as well since $\cos \delta$ and $\sin \delta$ averages to zero in this case. The Stokes parameters are often normalized by dividing through by $s_0$ so that the light has an intensity of one. Natural light then has the Stokes parameters $(s_0,s_1,s_2,s_3)=(1,0,0,0)$. Horizontal and vertical polarized light has the Stokes parameters $(1,1,0,0)$ and $(1,-1,0,0)$ respectively, since then either the horizontal or the vertical component is zero. Furthermore, we can square the last three expressions in equation \ref{eq:stokes_param_e} to get the following expression:
\begin{equation}
    \label{eq:stokes_char_eq}
    s_0^2 = s_1^2 + s_2^2 + s_3^2,
\end{equation}
which is an equality only in the case of completely polarized light. For partially polarized light the so called degree of polarization is given by:
\begin{equation}
    V = \frac{\sqrt{s_1^2 + s_2^2 + s_3^2}}{s_0},
\end{equation}
which in the context of this work is equal to one. In the same sense we can define the degree of linear polarization $V_l$ and the degree of circular polarization $V_c$ as:
\begin{equation}
    V_l = \frac{\sqrt{s_1^2+s_2^2}}{s_0} = V \cos 2\chi \qquad V_c = \frac{s_3}{s_0} = V \sin 2\chi,
\end{equation} % TODO we get -1 for degC check if it is LCP
where the angle $\chi$ is the ellipticity angle in the polarization ellipse representation. $V_l$ and $V_c$ are measures of how close the state is to the pure linear and circular polarization states respectively. Since $\chi$ is in the interval $\left[-\frac{\pi}{4}, +\frac{\pi}{4}\right]$, $\cos 2\chi$ is strictly positive. Meanwhile, in this range the values of $V_c$ lie in the interval $\left[-1, +1\right]$, where negative values indicate that the light is LCP and RCP if $V_c$ is positive. Furthermore, we see that $V_l$, $V_c$ and $V$ are related in the following way:
\begin{equation}
    V = \sqrt{V_l^2 + V_c^2}.
\end{equation}
Additionally, the semiaxes of the corresponding polarization ellipse are given by:
\begin{equation}
    a=\frac{s_0}{2}(V+V_l), \qquad b=\frac{s_0}{2}(V-V_l).
\end{equation}
%with an area given by:
%\begin{equation}
%    A=\frac{\pi}{4}I^2V^2\sin^2 2\chi.
%\end{equation}
Before we continue and introduce the Jones vectors, it is worth mentioning the vector representation of the Stokes parameters since it is widely used. Specifically, the Stokes parameters are often arranged as a 4x1 Stokes vector $\bm{s}$:

\begin{equation}
    \bm{s}=
    \begin{pmatrix}
    s_0 \\
    s_1 \\
    s_2 \\
    s_3
    \end{pmatrix},
\end{equation}
or horizontally as $\bm{s}=(s_0, s_1, s_2, s_3)^T$. Strictly speaking the set of Stokes vectors with addition and scalar multiplication do not form a vector space, since negative intensities do not exist. Nonetheless, for most purposes it still behaves as a vector. The normalized Stokes vectors of the degenerate polarization states are shown in table \ref{tab:pol_statevectors} \cite{Hecht, Shurcliff1962, GilPerez2017}.

% TODO removed subsection again because of numbering yes
\subsection{The Poincaré sphere}
\label{sec:the_poincare_sphere}
The Poincaré sphere which was proposed in 1892 by Henri Poincaré is a useful tool for representing polarized light geometrically and predicting how an optical element will change a given polarization state. It is worth noting that an almost equivalent representation known as the Bloch sphere is used in quantum mechanics to visualize a two-level system or qubit. The Poincaré sphere can be constructed directly with the help of the Stokes vector. For this purpose we define the unit Poincaré vector $\bm{u}$ of $\bm{s}$ as:

\begin{equation}
    \bm{u}=
    \begin{pmatrix}
    u_1 \\
    u_2 \\
    u_3
    \end{pmatrix}
    =
    \begin{pmatrix}
    \cos 2\psi \cos 2\chi \\
    \sin 2\psi \cos 2\chi \\
    \sin 2\chi
    \end{pmatrix},
\end{equation}
which is further composed with the Stokes vector so that $\bm{s}$ can be written in the block form: % TODO looks like spherical unit vector? + mention again that V=1 .. so ... eh (Done, it's not)
\begin{equation}
    \label{eq:poincare_stokes}
    \bm{s}=I
    \begin{pmatrix}
    1 \\
    V \bm{u}
    \end{pmatrix}
    =I
    \begin{pmatrix}
    1 \\
    \bm{p}
    \end{pmatrix},
\end{equation}
where the vector $\bm{p}=V\bm{u}$ is called the polarization vector or Bloch vector in a quantum mechanical context. Again $\bm{s}$ is often normalized by setting $I=1$ in equation \ref{eq:poincare_stokes}. Subsequently, with respect to the coordinate system defined by the axes $s_1s_2s_3$, equation \ref{eq:stokes_char_eq} defines the Poincaré sphere. This is shown in figure \ref{fig:poincare_sphere_intro}. Evidently the Poincaré sphere has a radius of one which means that states lying on its surface are totally polarized. All states we encounter in the context of this work will therefore be contained on its surface since $V=1$.

\begin{figure}[h]
    \centering
    \includestandalone[scale=0.9]{images/4_chapter04/tikz_poincare_sphere_intro}
    \caption{Example of a normalized polarization state $\bm{u}$ on the surface of the Poincaré sphere and its associated orthogonal state $-\bm{u}$. All normalized fully polarized states exists on the surface of the sphere and the position of a specific normalized state is defined by the last three components of its corresponding Stokes vector.}
    \label{fig:poincare_sphere_intro}
\end{figure}

Several properties can be observed directly from this geometrical representation. For example all linear polarized states will exist in the plane with $s_3=0$. Similar, the points $s_{2+}(1,0,0)$, $s_{1-}(-1,0,0)$, $s_{2+}(0,1,0)$ and $s_{2-}(0,-1,0)$ are assigned to the linear polarization states whose planes of polarization are along the horizontal, vertical, \SI{+45}{\degree} and \SI{-45}{\degree} axes respectively. States with a positive ellipticity $\psi$ lie on the northern hemisphere and the RCP state is located at the north pole ($s_{3+}(0,0,1)$). Likewise, all points on the southern hemisphere correspond to states with a negative ellipticity and the point $s_{3-}(0,0,-1)$ at the south pole corresponds to the LCP state. Furthermore, a pair of points on opposing sides of the sphere correspond to a pair of orthogonal states. An example of this is shown in figure \ref{fig:poincare_sphere_intro} as the pair $\bm{u}$ and $-\bm{u}$. Another useful feature of this representation is that when the light propagates in a medium and the polarization state changes it describes a trajectory on the sphere. This makes it possible to visualize the effect of an optical element or even a series of elements on a given initial polarization state.

% TODO Removed subsection
\subsection{Jones vectors}
\label{sec:jones_vectors}
The Jones vector is another commonly used representation of polarized light, which again like the Stokes representation has its own advantages and disadvantages. For example the phase information is conserved in the Jones representation but it can only describe fully polarized light. Unlike the Stokes parameters, the Jones vector is also able to describe coherent light. Another advantage of the Jones vector for calculations is that it is straight forward to obtain its components. Specifically, the Jones vector is given directly by the field components as:
\begin{equation}
    \label{eq:jones_vector1}
    \bm{\mathcal{E}} = 
    \begin{pmatrix}
    E_x(t) \\
    E_y(t)
    \end{pmatrix},
\end{equation}
where $E_x(t)$ and $E_y(t)$ are the scalar field components of the electric field. The Jones vectors actually forms a vector space unlike the Stokes vectors. In this notation it is possible to preserve the phase information of the wave, since it is not averaged out as for the Stokes parameters. If we assume a quasimonochromatic wave like it is given in equation \ref{eq:quasi_mono_efields} and define $u(t)=\Bar{k}z-\Bar{\omega}t$, then we can rewrite the Jones vector in complex form and factor out the phases:
\begin{equation}
    \label{eq:jones_vector2}
    \bm{\mathcal{E}} = e^{iu(t)}
    \begin{pmatrix}
    E_{0x(t)}e^{i\delta_x(t)} \\
    E_{0y(t)}e^{i\delta_y(t)}
    \end{pmatrix}.
\end{equation}
 Furthermore, if we consider the special case where the amplitudes and phases do not vary with time we get the time independent representation of the Jones vector:
 \begin{equation}
    \label{eq:jones_vector3}
    \bm{\mathcal{E}}=
    \begin{pmatrix}
    E_{0x}e^{i\delta_x} \\
    E_{0y}e^{i\delta_y}
    \end{pmatrix},
\end{equation}
where we have left out the global phase factor $e^{iu(t)}$, since it does not change the polarization state as we saw in for example section \ref{sec:polellipse}. The Jones vector is often normalized so that the sum of the squares of the components is equal to one. For example in the case of a field with $E_{0x}=E_{0y}$ and $\delta_x=\delta_y$ we get the normalized Jones vector:
\begin{equation}
    \bm{\mathcal{E}}_{\SI{45}{\degree}}= \frac{1}{\sqrt{2}}
    \begin{pmatrix}
    1 \\
    1
    \end{pmatrix},
\end{equation}
which is a \SI{45}{\degree} linear polarized state. Additional examples of Jones vectors representing the degenerate polarization states are summarized in table \ref{tab:pol_statevectors}. A useful operation is the scalar or inner product of two Jones vectors $\bm{\tilde{u}}$ and $\bm{\tilde{v}}$ which is defined as:
\begin{equation}
    \bm{\tilde{u}}^{\dagger}\bm{\tilde{v}} = \left(u_1^{*}, u_2^{*} \right)
    \begin{pmatrix}
    v_1 \\
    v_2
    \end{pmatrix}
    = u_1^{*}v_1 + u_2^{*}v_2,
\end{equation}
where the superscript $\dagger$ denotes conjugate transpose. In the same sense as for real vectors two Jones vectors are therefore said to be orthogonal if their inner product is zero. Moreover, the inner product can be used to calculate the intensity of a polarization state as:
\begin{equation}
    I=|\bm{\mathcal{E}}|^2=\bm{\mathcal{E}}^{\dagger}\bm{\mathcal{E}}
\end{equation}

With this we can express any Jones vector in the polarization ellipse representation through the intensity $I$, the orientation angle $\psi$ and the ellipticity angle $\chi$ as follows:

\begin{align}
\begin{split}
    \bm{\mathcal{E}} &=
    \sqrt{I}
    \begin{pmatrix}
    \cos \chi \cos \psi - i\sin \chi \sin \psi \\
    \cos \chi \sin \psi + i\sin \chi \cos \psi
    \end{pmatrix}
    \\
    &=
    \sqrt{I}
    \begin{pmatrix}
    \cos \psi & -\sin \psi \\
    \sin \psi & \cos \psi
    \end{pmatrix}
    \begin{pmatrix}
    \cos \chi \\
    i\sin \chi
    \end{pmatrix},
\end{split}
\end{align}
where we see that the rightmost factor is a Jones vector representing elliptical polarized light. The matrix in the middle is simply a rotation matrix that rotates the Jones vector by the angle $\psi$ and the factor to the left is the overall amplitude. This direct correspondence between the polarization ellipse parameters and the Jones vectors shows us that they are able to fully describe any polarization state. Before we move on to the next section it is worth mentioning how the components of the Jones vector transform into other reference frames, since this will be useful in the description of rotated elements. So far the Jones vectors have been defined with respect to a pair of axes XY lying in the reference plane tangential to the wavefront. Evidently the Jones vector in this frame can be written as:
\begin{equation}
    \bm{\mathcal{E}} = E_x \bm{e}_x + E_y \bm{e}_y, \qquad 
    \bm{e}_x = 
    \begin{pmatrix}
    1 \\
    0
    \end{pmatrix},
    \;
    \bm{e}_y = 
    \begin{pmatrix}
    0 \\
    1
    \end{pmatrix},
\end{equation}
where the basis vectors $\bm{e}_x$ and $\bm{e}_y$ represents light linear polarized along the X and Y axes respectively. In fact any complex 2x1 vector can be regarded as a Jones vector. It is therefore clear that the Jones vectors form a vector space and with that other bases can be defined as well. Consequently, the reference frame change from XY to X'Y' can be represented by an orthogonal transformation which can be defined as:
\begin{equation}
    \label{eq:jones_vector_transformation}
    \bm{\mathcal{E}}' = \hat{Q}(\theta) \bm{\mathcal{E}}, \qquad
    \hat{Q}(\theta) =
    \begin{pmatrix}
    \cos \theta & \sin \theta \\
    -\sin \theta & \cos \theta
    \end{pmatrix},
\end{equation}
where $\hat{Q}(\theta)$ corresponds to a counterclockwise rotation about the Z axis by the angle $\theta$ as shown in figure \ref{fig:frame_rotation}. 

\begin{figure}[h]
    \centering
    \includestandalone[scale=1]{images/4_chapter04/tikz_frame_rotation}
    \caption{An example of a coordinate transformation of the Jones vectors from the coordinate system with axes XY into the primed system. Such a transformation can be represented by a counterclockwise rotation by an angle $\theta$ around the Z axis.}
    \label{fig:frame_rotation}
\end{figure}

So far in this section we have described two mutually complementary representations of the polarization states. Specifically, certain calculations are better carried out using the Stokes parameters such as in the case of superimposing two incoherent light beams, while the Jones vector representation is better suited for describing fully polarized coherent light as in the context of this work. Nonetheless, the degrees of polarization are readily obtained using the Stokes parameters, which are useful since they give a measure of the distance between a given polarization state and the degenerate polarization states. As mentioned earlier both sets of vectors have a set of matrices associated with them, where the Jones matrices are a subset of the Mueller matrices. However, we will limit the following description to the Jones calculus, since in the context of this work it provides a sufficient mathematical framework for describing the polarization and changes of the polarization state caused by nondepolarizing media. How these matrices relate to optical elements will be shown in the following sections \cite{Hecht, GilPerez2017}.

\newpage

\begin{table}[H]
    \centering
    \includestandalone{images/4_chapter04/pol_statevectors_table}
    \caption{Summary of all the normalized Jones and Stokes vectors for the degenerate polarization states.}
    \label{tab:pol_statevectors}
\end{table}

% TODO1 add remark that we first consider some general useful properties or features of jones matrices and then look at the individual implementations representing different optical elements (Done)

% TODO2 section title :) (or Mathematical description of nondepolarizing media or mathematical calculus or something) (Done)
\section{Jones calculus}
\label{sec:jonescalc}
The Jones calculus or specifically the Jones matrices can be used to describe the non-depolarizing polarimetric interaction of fully polarized light with matter. For that we first introduce a few general properties of the Jones matrices. In the following section we will then apply these properties to describe and categorize the different types of waveplates. Evidently, the following description of optical elements is only valid for changes with a linear relationship between the input and output amplitudes since it is based on a system of linear equations. Under these assumptions we can then express any transformation of an initial or input polarization state $\bm{\mathcal{E}}$ into another final output state $\bm{\mathcal{E}}'$ as:
\begin{equation}
    \bm{\mathcal{E}}' = \hat{T} \bm{\mathcal{E}},
\end{equation}
where $\hat{T}$ is the 2x2 complex Jones matrix specifying the interaction. With this many of the common operations defined in linear algebra can be reused and given a physical interpretation in the Jones calculus. For example the matrix product of two Jones matrices $\hat{T}_1$ and $\hat{T}_2$ represents a so called series or train of optical elements, where each matrix describes the polarimetric change caused by the respective element. This can be extended to multiple elements so that the final state in a train of $n$ consecutive elements is given by:
\begin{equation}
    \label{eq:jones_series_product}
    \bm{\mathcal{E}}' = \prod_{i=1}^{n} \hat{T}_i \bm{\mathcal{E}}.
\end{equation}
It is clear that since the product of two matrices does in general not commute the order of the factors is important \cite{Jones1941}. Therefore if $\hat{T}_1$ is the first element encountered by the light then the order of the indices has to be reversed. It is worth noting that in general the action of materials on the input state depends on properties of the wave as well as how the interaction takes place. For example reflection off of a surface depends on the incident angle as well as the frequency of the wave. The Jones matrices themselves can therefore depend on a number of parameters reflecting the properties of the interaction, material and input state. Furthermore, since all Jones matrices are 2x2 complex matrices they can be factored into a product of three matrices using the singular value decomposition. This means we can express any Jones matrix $\hat{T}$ in the following form:
\begin{equation}
    \label{eq:jones_singular_value_decomposition}
    \hat{T} = \hat{T}_{R2}\diag{p_1, p_2}\hat{T}_{R1},
\end{equation}
where $\hat{T}_{R1}$ and $\hat{T}_{R2}$ are unitary\footnote{A complex square matrix $\hat{U}$ is unitary if $\hat{U} \hat{U}^\dagger=\hat{I}$ where $\hat{I}$ is the identity matrix} matrices and $\diag{p_1, p_2}$ is a diagonal matrix whose entries are the singular values of $\hat{T}$. The squares of $p_1$ and $p_2$ can be calculated directly using the following expressions:
\begin{equation}
    p_{1,2}^2 = \frac{1}{2}\left(\tr(\hat{T}\hat{T}^{\dagger}) \pm 
    \sqrt{\tr(\hat{T}\hat{T}^{\dagger})^2-4\det(\hat{T}\hat{T}^{\dagger})}\right),
\end{equation}
where $\tr$ is the matrix trace and the sign is respectively positive and negative for $p_1$ and $p_2$. The physical interpretation of these two values becomes evident if we consider the effect of the matrix $\diag{p_1, p_2}$ on an input state. Specifically, it simply scales the amplitudes of the state and with that also the intensity. Therefore, if the light is not amplified by the element in question then we get the following condition:
\begin{equation}
    \label{eq:passitivity_condition}
    p_1^2 \leq 1,
\end{equation}
which is also a defining property of Jones matrices. In other words a complex 2x2 matrix is a Jones matrix if equation \ref{eq:passitivity_condition} is satisfied \cite{Barakat1987}. Similar to Jones vectors the Jones matrices are defined up to an arbitrary phase factor $e^{i\phi}$, which means that two matrices $\hat{T}$ and $e^{i\phi}\hat{T}$ are equivalent in the sense that their respective final output states have equal intensities. A standard convention is therefore to assume that $\det \hat{T}$ is positive. Furthermore, the representation of a Jones matrix in another coordinate system follows directly from the coordinate transformation for Jones vectors given by equation \ref{eq:jones_vector_transformation}. Specifically, the transformation of a Jones matrix $\hat{T}$ in system XY to another representation $\hat{T}'$ in system X'Y' is given by:
\begin{equation}
    \label{eq:jones_matrix_transformation}
    \hat{T}' = \hat{Q}(\theta)\hat{T}\hat{Q}(-\theta).
\end{equation}
It is important to distinguish between frame rotations and rotations of the actual element. In the first case we rotate the axes and the second case can be understood as a rotation of the vector. Therefore, in the case of a rotation of the element around the Z-axis the signs of the rotation matrices in equation \ref{eq:jones_matrix_transformation} are inverted. It is worth noting that we only have to consider rotations since they are the only geometrical transformations that actually preserve the physically invariant quantities \cite{GilPerez2017}. Another interesting concept are singular states of polarization. Specifically, the singular value decomposition readily shows that for any Jones matrix there exists two pairwise orthonormal states $\tilde{\bm{\eta}}_1$ and $\tilde{\bm{\eta}}_2$ for which the orthogonality is preserved, in other words their respective output states $\bm{\eta}_1'$ and $\bm{\eta}_2'$ are mutually orthogonal. Furthermore, from the definition of the singular values the states $\tilde{\bm{\eta}}_1$ and $\tilde{\bm{\eta}}_2$ are orthonormal eigenvectors of the matrix $\sqrt{\hat{T}\hat{T}^{\dagger}}=\hat{T}_{R1}^{\dagger}\diag{p_1, p_2}\hat{T}_{R1}$ and are given explicitly by the columns of $\hat{T}^{\dagger}$. We can therefore express the singular states as:
\begin{equation}
    \bm{\eta}_1 = \hat{T}^{\dagger}_{R1}\begin{pmatrix} 1 \\ 0 \end{pmatrix}, \qquad
    \bm{\eta}_2 = \hat{T}^{\dagger}_{R1}\begin{pmatrix} 0 \\ 1 \end{pmatrix},
\end{equation}
and calculate the output states:
\begin{align}
\begin{split}
    \bm{\eta}'_1 &= \hat{T}\tilde{\bm{\eta}}_1 = 
    p_1\hat{T}_{R2}\hat{T}_{R1}\tilde{\bm{\eta}}_1 =
    p_1\hat{T}_{R2}\begin{pmatrix} 1 \\ 0 \end{pmatrix}, \\
    \bm{\eta}'_2 &= \hat{T}\tilde{\bm{\eta}}_2 = 
    p_2\hat{T}_{R2}\hat{T}_{R1}\tilde{\bm{\eta}}_2 =
    p_2\hat{T}_{R2}\begin{pmatrix} 0 \\ 1 \end{pmatrix}.
\end{split}
\end{align}
These expressions for the final states can then be used to calculate their respective intensities as:
\begin{equation}
    \bm{\eta}'^{\dagger}_1\bm{\eta}'_1=p_1^2, \qquad 
    \bm{\eta}'^{\dagger}_2\bm{\eta}'_2=p_2^2, \qquad 
    \bm{\eta}'^{\dagger}_1\bm{\eta}'_2=0.
\end{equation}
If we further define the transmittance as the ratio between the transmitted intensity and the incident intensity, then we see that the states $\bm{\eta}_1$ and $\bm{\eta}_2$ are exactly the states with the highest and lowest transmittance respectively. In other words, with respect to any other normalized output state, the states $\bm{\eta}'_1$ and $\bm{\eta}'_2$ are the output states with the highest and lowest intensities respectively. The last property we will introduce before applying these to describe waveplates in the Jones calculus is the normality of Jones matrices. As we shall see later this property is important for the calculation of the retardance for a given Jones matrix. Additionally, Jones matrices can be separated into two categories depending on the mutual orthogonality of their eigenvectors $\bm{\mathcal{E}}_q$ and $\bm{\mathcal{E}}_r$. The eigenvectors which are also known as the eigenstates or eigenpolarizations of $\hat{T}$ can be interpreted by considering their defining equations:
\begin{equation}
    \hat{T}\bm{\mathcal{E}}_q = \lambda_q\bm{\mathcal{E}}_q, \qquad 
    \hat{T}\bm{\mathcal{E}}_r = \lambda_r\bm{\mathcal{E}}_r,
\end{equation}
where $\lambda_q = |\lambda_q|e^{i\delta_q}$ and $\lambda_r = |\lambda_r|e^{i\delta_r}$ are the associated complex eigenvalues. In other words, the application of $\hat{T}$ to the eigenstates only scales them by a complex number but leaves the polarization type unchanged. Hence if $\bm{\mathcal{E}}_q$ and $\bm{\mathcal{E}}_r$ are mutually orthogonal then the Jones matrix is said to be normal or homogeneous and nonnormal or inhomogeneous if they are nonorthogonal. We can therefore define the nonnormality or inhomogeneity parameter $\eta$ as follows:
\begin{equation}
    \eta=|\bm{\mathcal{E}}^{\dagger}_q \bm{\mathcal{E}}_r|.
\end{equation}
This means that the inhomogenity is zero if $\hat{T}$ is normal and one if the eigenvectors are parallel, where the latter is also known as the degenerate case. To calculate the inhomogeneity and verify that a given Jones matrix $\hat{T}$ is normal we use another derived expression for $\eta^2$ which is given by:
\begin{equation}
    \eta^2 = \frac{2||\hat{T}||_2^2-|\tr\hat{T}|^2-|(\tr\hat{T})^2-4\det\hat{T}|}{2||\hat{T}||_2^2-|\tr\hat{T}|^2+|(\tr\hat{T})^2-4\det\hat{T}|},
\end{equation}
where $||\hat{T}||_2=\sqrt{\tr \hat{T}^{\dagger} \hat{T}}$ denotes the Frobenius norm of $\hat{T}$ \cite{Lu1994}. With this we can define and categorize waveplates in the context of the Jones calculus which is shown in the following section. 

% TODO no more subsections ...
\subsection{Waveplates in the Jones calculus}
In section \ref{sec:waveplates} we presented a phenomenological description of waveplates for which we assumed them to be fully transparent, i.e. the intensity of the incident light was equal to the intensity of the transmitted light. Under this assumption such ideal elements only exhibit birefringence and we will therefore refer to them as retarders or ideal waveplates. Furthermore, in the Jones calculus these ideal waveplates are represented by unitary matrices. Later in this section we will describe and add diattenuators to the framework. The diattenuator will allow us to quantify absorption by the waveplates and it especially simplifies the description of anisotropic absorption effects such as dichroism, reflection losses and transmission losses. In theory together the retarder and diattenuator then constitute a complete model of the waveplate and even of any non-depolarizing material. In the following we classify retarders according to the polarization type of their eigenstates, consequently there are three different types of retarders; elliptic, circular and linear. As for the different polarization types the most general type of retarder is the elliptic retarder which is characterized by the following eigenstates:
\begin{align}
\begin{split}
    \label{eq:ellip_pol_eigenstates}
    \bm{\mathcal{E}}_1(\alpha, \delta) &= 
    \begin{pmatrix} c_{\alpha} e^{-i\delta /2} \\ s_{\alpha} e^{i\delta /2} \end{pmatrix}
    = 
    \begin{pmatrix} c_{\chi} c_{\psi} - i s_{\chi} s_{\psi} \\ 
    c_{\chi} s_{\psi} + i s_{\chi} c_{\psi} \end{pmatrix},
    \\
    \bm{\mathcal{E}}_2 \left( \frac{\pi}{2} - \alpha, \delta+\pi \right) &= 
    \begin{pmatrix} -i s_{\alpha} e^{-i\delta /2} \\ i c_{\alpha} e^{i\delta /2} \end{pmatrix}
    = 
    \begin{pmatrix} -c_{\chi} s_{\psi} + i s_{\chi} c_{\psi} \\ 
    c_{\chi} c_{\psi} + i s_{\chi} s_{\psi} \end{pmatrix},
\end{split}
\end{align}
where we have set $\sin(x) = s_x$ and $\cos(x) = c_x$ which we use from here on. We see that $\bm{\mathcal{E}}_1$ and $\bm{\mathcal{E}}_2$ are both general elliptical polarization states, where the specific shape of their respective polarization ellipses is determined by the parameters $\alpha$ and $\delta$. It is worth emphasizing that $\delta$ is not the phase difference of the input state but a characterizing parameter of the retarder also known as the circularity. The corresponding Jones matrix $T_R(\alpha, \delta, \Delta)$ representing the elliptical retarder is given by:
\begin{equation}
    \hat{T}_R(\alpha, \delta, \Delta) = 
    \begin{pmatrix} 
    c^2_{\alpha}  e^{i\Delta /2} + s^2_{\alpha} e^{-i\Delta /2} & i s_{2\alpha} s_{\Delta/2} e^{-\delta} \\ 
    i s_{2\alpha} s_{\Delta/2} e^{-\delta} & s^2_{\alpha} e^{i\Delta/2} + c^2_{\alpha} e^{-i\Delta /2}
    \end{pmatrix}, 
\end{equation}
which introduces a phase difference of $\Delta$ between its eigenstates $\bm{\mathcal{E}}_1$ and $\bm{\mathcal{E}}_2$. In other words we can define the retardance $\Delta$ in terms of the arguments or phases of the eigenvalues as follows:
\begin{equation}
    \label{eq:jones_ret_def}
    \Delta = |\delta_r-\delta_q|.
\end{equation}
This is a reasonable definition if we consider the definition of the eigenstates and the polar form of the eigenvalues. That is the global phase change, which is also the only change, of each eigenstate is the phase of the eigenvalue. Especially, in the case of birefringent waveplates for which the linear eigenstates align with the fast and slow axis of the birefringent material we get exactly an additional phase shift of $\Delta$ between field components along said axes. Furthermore, we see that this is in fact the most general representation of a retarder, since the coefficients of any 2x2 unitary matrix can be factorized in this manner. If we change the frame in which we represent $\hat{T}_R(\alpha, \delta, \Delta)$ so that it aligns with the mayor and minor axes of the polarization ellipse of $\bm{\mathcal{E}}_1$, then $\psi=0$ for an orientation of \SI{0}{\degree} and therefore $\delta=\frac{\pi}{2}$, $\alpha=\chi$. With this the Jones matrix reduces to the following:
\begin{equation}
    \hat{T}_R\left(\chi, \frac{\pi}{2}, \Delta\right) = 
    \begin{pmatrix} 
    c^2_{\chi} e^{i\Delta /2} + s^2_{\chi} e^{-i\Delta /2} & s_{2\chi} s_{\Delta/2} \\
    -s_{2\chi} s_{\Delta/2} & s^2_{\chi} e^{i\Delta /2} + c^2_{\chi} e^{-i\Delta /2}
    \end{pmatrix}. 
\end{equation}

In the case of circular polarized eigenstates $\bm{\mathcal{E}}_1$ and $\bm{\mathcal{E}}_2$ the polarization ellipse reduces to a circle and $\chi=\frac{\pi}{4}$ so that $\delta=\frac{\pi}{2}$ and $\alpha=\frac{\pi}{4}$. A retarder with circular eigenstates is known as a circular retarder or rotator since in this present case the Jones matrix takes on the following form:
\begin{equation}
    \hat{T}_{RC}(\Delta)=\hat{T}_R\left(\frac{\pi}{4}, \frac{\pi}{2}, \Delta\right) = 
    \begin{pmatrix} 
    c_{\Delta/2} & s_{\Delta/2} \\
    -s_{\Delta/2} & c_{\Delta/2}
    \end{pmatrix},
\end{equation}
which is also the representation of a rotation matrix $\hat{Q}(\theta)$ with $\theta=\Delta/2$. In other words the effect of $\hat{T}_{RC}(\Delta)$ is to rotate an incident state by $\theta=\Delta/2$. The final type is the linear retarder which is characterized by having linear eigenstates. For this type $\chi=0$, therefore $\delta=0$ and $\alpha=\psi$. With this the Jones matrix of the linear retarder is given by:
\begin{equation}
    \hat{T}_{RL}\left(\psi, \Delta\right) = 
    \begin{pmatrix} 
    c^2_{\psi} e^{i\Delta /2} + s^2_{\psi} e^{-i\Delta /2} & is_{2\psi} s_{\Delta/2} \\
    is_{2\psi} s_{\Delta/2} & s^2_{\psi} e^{i\Delta /2} + c^2_{\psi} e^{-i\Delta /2}
    \end{pmatrix},
\end{equation}
and if we again align the coordinate frame with the mayor and minor axes of its eigenstates then $\psi=0$ and $\hat{T}_{RL}$ adopts the following diagonal form:
\begin{equation}
    \hat{T}_{RL}\left(\psi=0, \Delta\right) \equiv \hat{T}_{RL0}\left(\Delta\right) = 
    \begin{pmatrix} 
    e^{i\Delta /2} & 0 \\
    0 & e^{-i\Delta /2}
    \end{pmatrix},
\end{equation}
which is known as a horizontal linear retarder. In this frame we directly recognize the total phase shift $\Delta$ of the eigenstates. Furthermore, we see that the mayor and minor axis of the polarization ellipse of the eigenstates align with the fast and slow axis of the retarder. If we set $\Delta=\pi$ in the expression for the linear retarder we get the Jones matrix for the half wave retarder or transparent half waveplate as follows:
\begin{equation}
    \hat{T}_{RL}(\psi, \pi)= 
    \begin{pmatrix} 
    c_{2\psi} & s_{2\psi} \\
    s_{2\psi} & -c_{2\psi}
    \end{pmatrix}.
\end{equation}
We see that $\hat{T}_{RL}(\psi, \pi)$ is equivalent to a rotation by $-2\psi$ and an inversion of the y-component where the latter can be represented by $\hat{T}_{RL}(0, \pi)$. Therefore, essentially $\hat{T}_{RL}(\psi, \pi)$ inverts the handedness and then rotates the polarization ellipse of the state by $-2\psi$. These two actions combined are also known as an improper rotation. A circular polarized state therefore remains circular but has its handedness reversed by the action of $\hat{T}_{RL}(\psi, \pi)$. Finally, we can also see this by writing $\hat{T}_{RL}(\psi, \pi)$ to obtain the following representation of the Jones matrix describing the half wave retarder:
\begin{equation}
    \label{eq:l2_wp_effect}
    \hat{T}_{RL}(\psi, \pi)\equiv
    \hat{T}_{1/2}(\psi)=
    \hat{Q}(-2\psi)\hat{T}_{RL}(0, \pi)=\hat{Q}(-2\psi)\diag{1,-1}.
\end{equation}
If the input light is for example linear horizontal polarized and $\psi=\frac{\pi}{4}$ then the inversion leaves the state unaffected but the rotation produces an output state which is linear vertical polarized.
Similarly, if we set $\Delta=\pi/2$ then the expression for the linear retarder reduces to the following:
\begin{align}
\begin{split}
    \hat{T}_{1/4}(\psi)\equiv\hat{T}_{RL}(\psi, \pi/2)
    &=\frac{1}{\sqrt{2}}\left(\hat{I}+i\hat{Q}(-2\psi)\hat{T}_{RL}(\pi, \pi)\right)
    \\
    &=\frac{1}{\sqrt{2}}\left(\hat{I}+i\hat{T}_{1/2}(\psi)\right),
\end{split}
\end{align}
which is the Jones matrix for a quarter wave retarder. This expression shows that $\hat{T}_{RL}(\psi, \pi/2)$ is equivalent to the effect of $\hat{T}_{1/2}(\psi)$ followed by a \SI{90}{\degree} phase shift this new state is then superimposed on the input state. For example a linear horizontal state is first transformed into a linear vertical polarized state and then shifted by a quarter period. This rotated and shifted state is then further superimposed on the horizontal input state so that the output state will be RCP. 

It is in fact possible to represent any linear retarder using a combination of rotation matrices $\hat{Q}$ and horizontal linear retarders $\hat{T}_{RL0}$ as follows:
\begin{equation}
    \hat{T}_{RL}(\psi, \Delta)=\hat{T}_{RC}(-2\psi)\hat{T}_{RL0}(\Delta)\hat{T}_{RC}(2\psi)=
    \hat{Q}(-\psi)\hat{T}_{RL0}(\Delta)\hat{Q}(\psi).
\end{equation}
In other words a linear retarder is equivalent to a horizontal linear retarder $\hat{T}_{RL0}(\Delta)$ placed between a left- and right-handed circular retarder $\hat{T}_{RC}(-2\psi)$ and $\hat{T}_{RC}(2\psi)$ respectively. In general an elliptic retarder $\hat{T}_R(\alpha, \delta, \Delta)$ can be realized by the following combination of elements \cite{GilPerez2017}:
\begin{equation}
    \hat{T}_R(\alpha, \delta, \Delta)=
    \hat{T}_{RL0}(-\delta)\hat{Q}(-\alpha)\hat{T}_{RL0}(\Delta)\hat{Q}(\alpha)\hat{T}_{RL0}(\delta).
\end{equation}
In the following we will introduce the diattenuator in a similar fashion as we did with the retarder. It is used to describe the polarization dependent transmittance of the waveplate. Any element for which the singular values $p_1$ and $p_2$ of its Jones matrix are mutually different can be considered a diattenuator \cite{Savenkov2005}. This means that for example matrices $\hat{T}_r$ and $\hat{T}_t$ describing the transmission and reflection at an interface are examples of Jones matrices representing diattenuators. If we further consider homogeneous diattenuators then we can again use a singular value decomposition as in equation \ref{eq:jones_singular_value_decomposition} to factorize the diattenuator Jones matrix as follows:
\begin{equation}
    \label{eq:diattenuator_singular_value_decomposition}
    \hat{T}_D = \hat{T}_D^{\dagger} = \hat{T}_R\hat{T}_{DL0}\hat{T}_R^{\dagger}, 
    \qquad 
    \hat{T}_{DL0}=\diag{p_1, p_2}.
\end{equation}
We know from the previous section that $p_1^2$ and $p_2^2$ are respectively the maximum and minimum transmittance. We can therefore use $p_1$ and $p_2$ to define the diattenuation in the Jones calculus as follows: 
\begin{equation}
    \label{eq:diattenuation_1}
    D = \frac{p_1^2-p_2^2}{p_1^2+p_2^2}
\end{equation}
Furthermore, as with the retarders the diattenuators are also grouped or categorized based on the polarization type of their associated eigenstates. This means that the most general type of homogeneous diattenuators is the elliptic diattenuator with eigenstates given in equation \ref{eq:ellip_pol_eigenstates}. Additionally, equation \ref{eq:diattenuator_singular_value_decomposition} shows that the diattenuator is represented by a Hermitian\footnote{A complex square matrix $\hat{H}$ is Hermitian if $\hat{H}=\hat{H}^{\dagger}$.} matrix. This means that the Jones matrix of the most common or general diattenuator can be written as:
\begin{equation}
    \hat{T}_D(\alpha, \delta, p_1, p_2) = 
    \begin{pmatrix} 
    p_1 c_{\alpha}^2 + p_2 s_{\alpha}^2 & s_{\alpha}c_{\alpha}(p_1-p_2)e^{-i\delta} \\
    s_{\alpha}c_{\alpha}(p_1-p_2)e^{i\delta} & p_1 s_{\alpha}^2 + p_2 c_{\alpha}^2
    \end{pmatrix},
\end{equation}
for which we assume that $0\leq p_2\leq p_1$. If $p_2=0$ then $\hat{T}_D(\alpha, \delta, p_1, 0)$ represents a total elliptic polarizer which completely absorbs the $\bm{\mathcal{E}}_2$ eigenstate. If we align the axes of the frame in which $\hat{T}_D(\alpha, \delta, p_1, p_2)$ is represented to the axes of the polarization ellipse of $\bm{\mathcal{E}}_1$ as in the case of the horizontal retarders, then we get the so called horizontal elliptic diattenuator. Again $\psi=0$ so that $\delta=\frac{\pi}{2}$, $\alpha=\chi$ which in $\hat{T}_D(\alpha=\chi, \delta=\frac{\pi}{2}, p_1, p_2)$ yields the respective Jones matrix representing the horizontal elliptic diattenuator. Figure \ref{fig:horizontal_diattenuator} shows a horizontal diattenuator and the effect of the diattenuation on its elliptical eigenstates. We see that in general the diattenuator changes the shape of the polarization ellipse depending on $p_1$ and $p_2$, while in contrast the retarders only rotate or change the handedness of the ellipse. This is also underlined by the fact that $\det(\hat{T}_D) = p_1p_2 \leq 1$ while $\det(\hat{T}_{R}) = 1$.

\begin{figure}[h]
    \centering
    \includestandalone[scale=1]{images/4_chapter04/tikz_horizontal_diattenuator}
    \caption{Elliptical diattenuator at an angle of \SI{0}{\degree} to its intrinsic frame also known as a horizontal diattenuator. In this case the axes of the diattenuator align with the mayor and minor axis of its eigenstates so that the two eigenstates $\bm{\mathcal{E}}_1$, $\bm{\mathcal{E}}_2$ directly experience an intensity attenuation of respectively $p_1^2$ and $p_2^2$.}
    \label{fig:horizontal_diattenuator}
\end{figure}

A diattenuator which is characterized by circular eigenstates, so that $\chi=\frac{\pi}{4}$ and therefore $\delta=\frac{\pi}{2}$, $\alpha=\frac{\pi}{4}$, is represented by the following Jones matrix:
\begin{equation}
    \hat{T}_{DC}(p_1, p_2) = \hat{T}_D\left(\frac{\pi}{4}, \frac{\pi}{2}, p_1, p_2\right) =
    \frac{1}{2}
    \begin{pmatrix} 
    p_1 + p_2 & -i(p_1 - p_2) \\
    i(p_1 - p_2) & p_1 + p_2
    \end{pmatrix}.
\end{equation}
It is worth noting that for $p_2=0$ $\hat{T}_{DC}(p_1, p_2=0)$ represents a circular polarizer which completely extinguishes the circular eigenstate $\bm{\mathcal{E}}_2$. The final of the three types is the linear diattenuator. In this particular case the eigenstates are linearly polarized so that $\chi=0$ and therefore $\delta=0$, $\alpha=\psi$. The linear diattenuator can accordingly be represented by the following Jones matrix:
\begin{equation}
    \hat{T}_{DL}(\psi, p_1, p_2) = 
    \begin{pmatrix} 
    p_1c_{\psi}^2 + p_2s_{\psi}^2 & c_{\psi}s_{\psi}(p_1 - p_2) \\
    c_{\psi}s_{\psi}(p_1 - p_2) & p_1s_{\psi}^2 + p_2c_{\psi}^2
    \end{pmatrix},
\end{equation}
which for $p_2=0$ represents the well known linear polarizer realized by for example a wiregrid polarizer. The Jones matrix then takes on the form:
\begin{equation}
    \hat{T}_{DL}(\psi, p_1, 0) = \hat{T}_{D}(\psi, 0, p_1, 0) =
    p_1
    \begin{pmatrix} 
    c_{\psi}^2 & c_{\psi}s_{\psi} \\
    c_{\psi}s_{\psi} & s_{\psi}^2
    \end{pmatrix}.
\end{equation}
It is worth noting that Malus' law can easily be derived from this, if we consider horizontally polarized light with an initial intensity $I_0=|E_x|^2$, as follows:
\begin{equation}
    \bm{\mathcal{E}}_{\SI{0}{\degree}}^{\dagger}
    \hat{T}_{DL}(\psi, p_1=1, 0)\bm{\mathcal{E}}_{\SI{0}{\degree}} = |E_x|^2p_1^2c_{\psi}^2 = I_0 c_{\psi}^2.
\end{equation}

Furthermore, oriented at \SI{0}{\degree} the horizontal diattenuator takes on the diagonal form $\hat{T}_{DL0}(p_1, p_2) = \hat{T}_{DL}(0, p_1, p_2)=\hat{T}_{D}(0,0,p_1,p_2)=\diag{p_1, p_2}$ and $\diag{p_1, 0}$ for a linear polarizer at $\psi = \SI{0}{\degree}$. Similar to the general form of the retarder, the elliptic diattenuator can also be written as a product of rotation matrices, horizontal retarders and diattenuators as follows:
\begin{align}
\begin{split}
    \hat{T}_{D}(\alpha, \delta, p_1, p_2) 
    &= \hat{T}_{RL0}(-\delta)\hat{T}_{DL}(\alpha, p_1, p_2)\hat{T}_{RL0}(\delta) \\
    &= \hat{T}_{RL0}(-\delta)\hat{Q}(-\alpha)\hat{T}_{DL0}(p_1, p_2)\hat{Q}(\alpha)\hat{T}_{RL0}(\delta).
\end{split}
\end{align}
This means that any diattenuator is equivalent to a linear diattenuator placed between two crossed linear retarders. With the introduction of the diattenuator we can finally describe real linear retarders or specifically waveplates which in general show some amount of diattenuating effects \cite{D.CLARKE1971}. For this purpose we can assume that the optic axes of the retarder and diattenuator coincide, this assumption can be shown to be equivalent to the normality condition \cite{Bass1995}. Furthermore, we know from the previous section that states polarized along the fast or slow axis of the birefringent material of the waveplate will not undergo a polarization change as it passes through the waveplate. The eigenstates of a birefringent material are therefore linear and the planes of polarization coincide with the slow and fast axes of the material. Consequently, we can describe the waveplate using a linear retarder and diattenuator. A linear nonideal retarder or diattenuating retarder can be represented by the Jones matrix $\hat{T}_{RDL}$, which depends on four parameters; the orientation angle $\psi$ between the fast axis of the retarder $X_0$ and the axis $X$ of the reference frame, the principal intensities $p_1^2$ and $p_2^2$ and finally on the retardance $\Delta$. This means that $\hat{T}_{RDL}$ can be represented by the following product of matrices:
\begin{equation}
    \hat{T}_{RDL}(\psi, \Delta, p_1, p_2) = \hat{Q}(-\psi)\hat{T}_{RL0}(\Delta)\hat{T}_{DL0}(p_1, p_2)\hat{Q}(\psi),
\end{equation}
we can carry out the multiplication to obtain the following representation of $\hat{T}_{RDL}$:
\begin{equation}
    \hat{T}_{RDL}(\psi, \Delta, p_1, p_2) = 
    \begin{pmatrix} 
    p_1e^{i\Delta/2}c_{\psi}^2+p_2e^{-i\Delta/2}s_{\psi}^2 & c_{\psi}s_{\psi}\left(p_1e^{i\Delta/2}-p_2e^{-i\Delta/2}\right) \\
    c_{\psi}s_{\psi}\left(p_1e^{i\Delta/2}-p_2e^{-i\Delta/2}\right) & 
    p_1e^{i\Delta/2}s_{\psi}^2+p_2e^{-i\Delta/2}c_{\psi}^2
    \end{pmatrix}.
\end{equation}
This Jones matrix describes a single waveplate and a train of waveplates is then simply given by the product of the individual $\hat{T}_{RDL}$ matrices. Obtaining the retardance and diattenuation of the final product can be achieved with a polar decomposition, which states that for any complex 2x2 matrix $\hat{T}$ there exist positive-definite\footnote{A Hermitian matrix $\hat{H}$ is positive-definite if $\bm{\mathcal{E}}^{\dagger}\hat{H}\bm{\mathcal{E}}$ is positive for any $\bm{\mathcal{E}}$} Hermitian matrices $\hat{H}$ and $\hat{H}'$ and a unitary matrix $\hat{U}$ such that $\hat{T}=\hat{U}\hat{H}=\hat{H}'\hat{U}$. This means that an arbitrary Jones matrix describing any non-depolarizing optical element can be written as a product of a retarder and diattenuator as follows \cite{GilPerez2017}:
\begin{equation}
    \hat{T} = \hat{T}_R\hat{T}_D = \hat{T}_D'\hat{T}_R.
    \label{eq:diattenuation_retarder}
\end{equation}
Furthermore, the retardance $\Delta$ defined through the eigenvalues of $\hat{T}_R$ can be calculated using the following expression:
\begin{equation}
    \Delta = 2 \cos^{-1} \frac{\left|\tr\hat{T}+\frac{\det \hat{T}}{\left|\det \hat{T}\right|} \tr \hat{T}^{\dagger}\right|}{2\sqrt{\tr \hat{T}\hat{T}^{\dagger}+2\left|\det \hat{T}\right|}},
\end{equation}
which follows from the calculation of the eigenvalues of $\hat{T}$. We have to assume that the transmittances $p_1$ and $p_2$ are not zero since then $\hat{T}$ represents a polarizer for which the determinant of the corresponding Jones matrix is zero. Additionally, this equation is only valid for homogeneous Jones matrices. In the inhomogeneous case the following expression has to be used instead:
\begin{align}
\begin{split}
    \label{eq:retardation_inhomogeneous}
    \Delta &= 2\cos^{-1}\left\{r\left[\frac{(1-\eta^2)(|\lambda_q|+|\lambda_r|)^2}
    {(|\lambda_q|+|\lambda_r|)^2-\eta^2(2|\lambda_q||\lambda_r|+\lambda_q^*\lambda_r+\lambda_q\lambda_r^*)}\right]^{1/2}\right\}, \\
    r &= \left|\cos \frac{\delta_q-\delta_r}{2}\right|
\end{split}
\end{align}
where $\lambda_{r,q}=|\lambda_{q,r}|e^{i\delta_{q,r}}$ are the eigenvalues of $\hat{T}$.
Similarly, the diattenuation $D$ can be calculated from the Jones matrix as follows:
\begin{equation}
    \label{eq:diattenuation_2}
    D=\sqrt{1-\frac{4|\det \hat{T}|^2}{(\tr \hat{T}\hat{T}^{\dagger})^2}},
\end{equation}
which has the advantage that it does not depend directly on $p_1$ and $p_2$. Again, in the case of an inhomogeneous Jones matrix equation \ref{eq:diattenuation_2} is invalid and the following equation should be used:
\begin{equation}
    \label{eq:diattenuation_inhomogeneous}
    D=\sqrt{1- \frac{4(1-\eta^2)^2|\lambda_q|^2|\lambda_r|^2}
    {\left(|\lambda_q|^2+|\lambda_r|^2-\eta^2(\lambda_q\lambda_r^*+\lambda_r\lambda_q^*)\right)^2}}.
\end{equation}
Expressions \ref{eq:retardation_inhomogeneous} and \ref{eq:diattenuation_inhomogeneous} are valid for $\eta \neq 0$ but evidently less efficient \cite{Lu1994}. With this we can describe a train of waveplates and subsequently characterize the final Jones matrix through the equivalent retardance and diattenuation. In the following section we will consider the wavelength dependency of waveplates and introduce the achromatic waveplate (AWP), which ideally has the special property of being wavelength independent. 

% TODO no more subsections. Maybe section instead and add to intro
\subsection{Achromatic waveplates}
\label{sec:achromatic_waveplates}

As we have seen already waveplates are highly wavelength sensitive due to the inverse wavelength dependency of the phase shift but also due to dispersion of the refractive index and consequently dispersion of the birefringence. In other words the three parameters $\Delta$, $p_1$ and $p_2$ of the four parameters in total which characterize a linear diattenuating retarder or waveplate are wavelength dependent. There are several types of AWPs based on different techniques which can in theory compensate for this wavelength dependency. One possibility is to use the phase shift induced by total internal reflection which can be realized through a Fresnel Rhomb. Using this method it is possible to create an almost perfect AWP. The Fresnel Rhomb works over a large frequency range since the variation of retardance is solely due to dispersion causing the reflectance to become wavelength dependent \cite{Hecht}. They often cause strong spatial shifts of the light, take up more space and are hard to align since the retardation strongly depends on the angle. Another method is to combine waveplates of different materials. The problem with this method is that only the dispersion of the materials is compensated for and so the resulting range is relatively low \cite{Masson2006}. The AWPs presented in this work are based on a similar design principle as the THz achromatic quartz quarter waveplates (TAQ) developed by Masson and Gallot in \cite{Masson2006}. The TAQ is ultimately based on the work by Destriau and Prouteau in 1949 \cite{Destriau1949} where they showed that the serial combination of a half waveplate and a quarter waveplate for which $\psi=\SI{60}{\degree}$ results in a new achromatic quarter waveplate. This new AWP showed a $\frac{\pi}{2}$ phase shift over the whole visible range. Masson and Gallot extended this idea to the THz range by combining six quartz plates with variable thicknesses and again at different azimuths $\psi$. This higher number of plates was necessary due to the increased bandwidth of short THz pulses. The operational principle of this type of AWP is based on the partial cancellation each plate is causing with respect to the frequency. Each quartz waveplate was assumed to have a negligible dichroism. It was reported that at \SI{1}{\tera \hertz} crystalline quartz showed an ordinary and extraordinary refractive index of $n_o=2.108$ and $n_o=2.156$ respectively. Due to this low birefringence the total thickness of the TAQ is relatively large. Assuming isotropic absorption of the plates, the AWP can be described by the following product:
\begin{equation}
    \hat{T}_{M} = \prod_{i=1}^{n=6} \hat{T}_{RL}(\psi_i, \Delta_i) = 
    \begin{pmatrix} 
    A & B \\
    -B^* & A^*
    \end{pmatrix},
    \label{eq:equiv_matrix}
\end{equation}
where according to equation \ref{eq:wp_eq} the $\Delta_i$ are proportional to the frequency and the thickness $d_i$ of the i-th waveplate. From the previous section we know that $\hat{T}_M$ can be written as the product of a diattenuator and a retarder. Therefore, setting $p_1=p_2=1$ and comparing the coefficients of $\hat{T}_{M}$ to a retarder represented by $\hat{T}_{RL}(\psi, \Delta_M)$ they get an expression for the so called resulting retardation dephasing $\Delta_M$:
\begin{equation}
    \label{eq:retardation_dephasing}
    \tan^2 \frac{\Delta_M}{2} = \frac{\operatorname{Im}A^2+\operatorname{Im}B^2}
    {\operatorname{Re}A^2+\operatorname{Re}B^2}.
\end{equation}
It is worth noting that the value $\Delta_M$ is not the same as the retardance defined in equation $\ref{eq:jones_ret_def}$. Nevertheless, it can be understood as the phase shift induced by a linear retarder $\hat{T}_{RL}(\SI{45}{\degree}, \Delta_M)$ on a horizontal state. The problem of determining the parameters $\psi_i$ and $d_i$ to get the appropriate phase shift at each frequency can be formulated as an optimization problem where the objective is to minimize the following error function:
\begin{equation}
    \label{eq:mass_loss}
    L(\psi_i, d_i)_M=\sum_{\nu}\left(\Delta_M(\nu)-\frac{\pi}{2}\right)^2.
\end{equation}
They minimize $L(\psi_i, d_i)_M$ using a simulated annealing algorithm over the frequency range \SIrange{0.2}{2}{\tera \hertz} and their result is shown in table \ref{tab:masson_result}. For this set of parameters it is reported that the deviation of $\Delta_M$ from $\frac{\pi}{2}$ is less than \SI{3}{\percent} over the frequency range \SIrange{0.25}{1.75}{\tera \hertz} and that the total thickness of the TAQ is \SI{31.4}{\milli \meter}. 

\begin{table}[h]
    \centering
    \includestandalone{images/4_chapter04/masson_result_table}
    \caption{Optimization result obtained by Masson and Gallot in \cite{Masson2006}. This particular TAQ consists of six plates of varying thicknesses and orientation angles.}
    \label{tab:masson_result}
\end{table}

A similar AWP is presented in another work by Wu, et al. Here they use sapphire instead of crystalline quartz which reduces the overall thickness of the AWP to \SI{2.453}{\milli \meter} since compared to quarts sapphire has a higher birefringence. They report a value of 3.39 and 3.07 for the ordinary and extraordinary refractive index respectively. Similarly, they optimize the loss function given in equation \ref{eq:mass_loss} and by stacking even up to 20 individual plates they obtain a deviation less than \SI{0.5}{\percent} from the $\pi/2$ target over the frequency range \SIrange{0.2}{2.0}{\tera \hertz}. Table \ref{tab:wu_result} shows their obtained values for the result with an overall thickness of \SI{2.453}{\milli \meter} consisting of six plates. This design was produced and characterized but showed more than a threefold increase in phase deviation compared to the optimization result. This deviation was attributed to angle misalignment and a mismatch between the used values from the actual values of the index of refraction of sapphire \cite{Wu2020}. 

\begin{table}[h]
    \centering
    \includestandalone{images/4_chapter04/wu_result_table}
    \caption{Obtained parameters for $n=6$ published in \cite{Wu2020} with a design similar to that of Masson and Gallot. The higher birefringence of sapphire compared to quartz allows the plates to be thinner compared to the result by Masson and Gallot.}
    \label{tab:wu_result}
\end{table}

A possible reduction of the angle misalignment can in theory be achieved using selective laser-induced etching (SLE) to produce form birefringent waveplates. Simply put SLE is a two step process; first transparent fused silica glass is selectively modified using a laser and subsequently the volume connected to the surface of the structure is removed by wet chemical etching. This means that SLE allows the fabrication of 3D structures with a precision of around \SI{1}{\micro \meter} \cite{Hermans2014}. This technique has the potential of removing the alignment step in the assembly of the AWP by fabricating the structure from a single glass piece and thereby reducing angle alignment errors. Ornik et al. showed in \cite{Ornik2018} that it is possible to fabricate high quality form birefringent $\lambda/2$ THz waveplates using this technique. The waveplates have the same periodically stratified structure as the media discussed at the end of section \ref{sec:wave_prop}. Each plate is manufactured from a rectangular slab of fused silica glass using SLE. The plates consist of a series of glass bars separated by air grooves. The structure is homogeneous along the direction of propagation, in other words the plates are stratified throughout the thickness $d_i$. In total four waveplates were characterized with different air and glass layer thicknesses as well as different plate thicknesses. One of the four plates is shown in figure \ref{fig:SLE_waveplate_JO_} next to a one euro coin.

\begin{figure}
    \centering
    \includegraphics[scale=0.45]{images/4_chapter04/SLE_waveplate_JO_.png}
    \caption{Left figure shows one of the four characterized fused silica glass waveplates next to a one euro coin. Right figure shows a microscope image of the same plate. The bright stripes are the glass bars while the black are air gaps. Source: \cite{Ornik2018}}
    \label{fig:SLE_waveplate_JO_}
\end{figure}

Among other characterizing measurements the transmitted intensity was measured after passing a linear polarizer for different azimuths $\psi$. We know from equation \ref{eq:l2_wp_effect} that the effect of a $\lambda/2$ retarder is to rotate the mayor axes of the polarization ellipse by $-2\psi$. Therefore, if the input state is $\bm{\mathcal{E}}_{\SI{45}{\degree}}$ we expect a minimum in the transmitted intensity since then the output state is perpendicular to the transmission axis of the polarizer. This is of course only true for the frequencies where the retardance of the plate is $\pi$. In fact, if we assume the polar decomposition of the waveplate is a linear diattenuating retarder then we can show\footnote{See section \ref{sec:transmission_min} of the appendix} the following equation for the azimuth of the minimum transmittance:  
\begin{equation}
    \psi_{min}=\arctan\left(\sqrt{\frac{p_2}{p_1}}\right)
\end{equation}
Therefore, only if $p_1=p_2$ do we get the minimum transmittance at an azimuth of \SI{45}{\degree}. Although measurements of the intensity at different azimuths showed that the minimum transmission is not obtained for $\psi=\SI{45}{\degree}$ but rather at \SI{46}{\degree}, \SI{46}{\degree}, \SI{47}{\degree}, \SI{49}{\degree} for increasing thickness of the four waveplates. This means that in the case of form birefringent fused silica glass waveplates and waveplates consisting of anisotropic materials dichroism should be taken into consideration. In turn, this furthermore means that we cannot simply rely on optimizing the retardance to obtain designs which produce pure output polarizations. Building onto the method of Masson and Wu we therefore need to implement new loss or objective functions for the $\lambda/4$ and $\lambda/2$ waveplate types if we want to take dichroism into consideration. At the same time the form birefringent AWP type allows us to directly optimize the birefringence of each individual plate, but it also comes with the cost of adding $2n$ more dimensions to the optimization problem. To summarize, the proposed AWP design consists of $n$ individual form birefringent plates where each plate has a similar structure as the one shown in figure \ref{fig:SLE_waveplate_JO_}. Similar to the TAQ waveplate, each individual plate is oriented at an azimuthal angle $\psi_i$ relative to the mayor axes of the polarization ellipse of the initial input state. Furthermore, each plate has a thickness or path length $d_i$ and widths $a_i$ and $b_i$ of the stratification. If not stated otherwise $a_i$ and $b_i$ denote the width of the material bars and air grooves respectively. The Jones matrix describing this configuration is therefore given by the following product:
\begin{equation}
    \hat{T}_{AWP}(\nu) = \prod_{i=1}^{n} \hat{T}_{RDL}(\psi_i, \Delta_i(\nu), p_{1,i}(\nu), p_{2,i}(\nu)),
\end{equation}
where the $\Delta_i$ are given by equation \ref{eq:wp_eq} which directly depends on $d_i$, the frequency $\nu=\frac{c}{\lambda}$ and the birefringence $\Delta n(\nu) = |n_o(\nu) - n_e(\nu)|$. Since the latter is approximated by equation \ref{eq:form_bf} $\hat{T}_{RDL}$ ultimately depends on $a_i$ and $b_i$ as well. This means that each plate has four free parameters. Since this already yields a fairly complex loss function we simplified the model by only considering dichroism and not first order reflection losses at the interfaces during optimization. The parameters $p_{j,i}$ are therefore simply the amplitude absorption losses along the principal axes of the plates, given by the imaginary parts of the calculated dielectric tensor coefficients. In the following we propose two objective functions which we optimized to obtain the results presented in the next chapter. In case of the objective function associated with the $\lambda/4$ waveplate type we optimize the components of the output state, while for the $\lambda/2$ type we directly optimize the coefficients of the Jones matrix describing the train of waveplates. If the input state is $\bm{\mathcal{E}}_{\SI{0}{\degree}}$ then the $\lambda/4$ waveplate should produce an output state $\bm{\mathcal{E}}_o$ which is circular polarized, the principal axes of the $\lambda/4$ waveplate will then be at an angle of $\frac{\pi}{4}$ to the horizontal input. In general we get the following for a linear horizontal input:
\begin{equation}
    \bm{\mathcal{E}}_o = \hat{T}_{AWP}\bm{\mathcal{E}}_{\SI{0}{\degree}} =
     \begin{pmatrix} T_{1,1} \\ T_{2,1} \end{pmatrix}.
\end{equation}
Circularly polarized states have the property that one component is real and the other imaginary. We use this to obtain the following loss function associated with the $\lambda/4$ waveplate:
\begin{equation}
\label{eq:l4_loss_function}
    L_{\lambda/4}(\bm{\psi}, \bm{d}, \bm{a}, \bm{b})=
    \sum_{\nu}\operatorname{Re}\left(r\right)^2+\left(1-\operatorname{Im}\left(r\right)\right)^2,
\end{equation}
with $r=\frac{T_{1,1}}{T_{2,1}}$ and $\bm{\psi}=\psi_1, ..., \psi_n$, $\bm{d}=d_1, ..., d_n$, etc. In other words we optimize the set of parameters so that $\bm{\mathcal{E}}_o=\bm{\mathcal{E}}_{LCP}$.
In case of the $\lambda/2$ waveplate we use the condition that $\hat{T}_{AWP}$ should act on an input state like $\hat{T}_{1/2}(\psi=\frac{\pi}{4})$ does. In other words, a linear horizontal state should transform into a linear vertical state and in the case of non-linear states the handedness should be inverted. We therefore optimize the parameters to fulfill the condition that $\hat{T}_{AWP}=\hat{T}_{1/2}(\psi=\frac{\pi}{4})$. With this the loss function to be optimized in case of the $\lambda/2$ waveplate is given by the following expression:
\begin{align}
\begin{split}
    L_{\lambda/2}(\bm{\psi}, \bm{d}, \bm{a}, \bm{b})=
    &\sum_{\nu}|T_{1,1}|^2+|T_{2,2}|^2\\
    +&\sum_{\nu}(1 - \operatorname{Im}(T_{1,0}))^2+
    (1 - \operatorname{Im}(T_{1,2}))^2.
\end{split}
\end{align}

In the present chapter four different representations of the polarization state of a wave were introduced; the Poincaré sphere, the polarization ellipse, the Stokes and the Jones vectors, while the first two are useful for visualizing the state the latter two are practical for calculations. We further saw how the polarization dependent interface transmission, dichroism and birefringence all affect the polarization state, while specifically the latter is used in waveplates to cause a predetermined relative phase shift of the wave components. It was shown how a stratified structure can give rise to form birefringence which in turn can be utilized for waveplates using the SLE technique. Additionally, the Jones calculus and specifically the Jones matrices allow a concise quantitative description of the polarization state change caused by waveplates. A composite achromatic waveplate consisting of a number of waveplates was shown to be effective over a broad frequency range in other works. At the end of the chapter we proposed a similar composite achromatic waveplate based on this design and the form birefringent waveplates. To that end we formulated two different objective functions; one for a $\lambda/2$ and one for a $\lambda/4$ form birefringent composite waveplate design. The results from the optimization of these objective functions will be presented in the following chapter.

% TODO note form birefringence allows avoiding stacking and almost arbitrary number of plates limited by cpu power (goes as n squared) ?

% TODO maybe define bandwidth like masson?

% TODO remove r, use j00, j10 instead. (done)

% Describe a little more how we get J... and parameters a,b,d,psi,mu (done)

% TODO add remark that retardation is delay between eigenstates or equivalently the difference of the two eigenvalue phases (done?)

% TODO check T_1/4. yea that part's fuckd (done?? check again)

% TODO Masson polar decomp and compare to elliptical with A,B.. matrix. Check what I wrote.

% TODO for results?. Give explanation why two l4 dont give l2. Problem is l4 is also rotator (is it?) -> different azimuths -> different outputs since l4 is rotator depending on azimuth ... (done)

% only for 100% pure circ is azimuth undef. (never happens)

% TODO write summary + transition to chapter 4, right? depends on final structure

% TODO remark that passivity means that abs(T_ij) <= 1 used in l_2 loss function.
