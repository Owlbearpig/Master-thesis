week 18.01.2021: start p32
week 25.01.2021: start p37
week 01.02.2021: start p42
week 08.02.2021: start p47
week 15.02.2021: start p52
week 22.02.2021: start p57
week 01.03.2021: start p62
week 08.03.2021: start p67
week 15.03.2021: start p72
week 22.03.2021: start p74 (deleted old notes ... + didn't write much)
week 29.03.2021: start p79
week 05.04.2021: start p84 (-fig lst(4p), -dedication(1p)) Restart 5p cnt. 
week 12.04.2021: start p80
week 19.04.2021: start p13 (85) only compile result chapter
week 26.04.2021: start p15 (87) slow progress ...
week 03.05.2021: start p18
week 21.06.2021: start p40
stopped counting

- TODO Need to say somewhere that focus is on lambda / 4 waveplates. In intro probably.
-TODO Add E perp to K and B etc. in beginning of maxwell stuff...

-TODO maybe add small note about why dealing with single frequency makes sense -> fft, superposition principle stuff (griffiths explains it 7.12 or something)

-indexellipsoid (done?)

- Important. Mention jones matrices can depend on several parameters in general, like wavelength or other stuff ... think it's described in Hecht (done)

- differential jones matrices, mby interesting, could be outlook ... contionous changing stripe widths or something like that (nah)

- retardation dephasing (masson, wu) vs retardance (eigenvalues) coincide for psi_i = 0

- The significance of the polarization dependent reflection losses will be discussed in the results
    - Don't forget this or scrap it
    
% \section{Lorentz oscillator model sign} Removed. It's from MIT6_007S11_sign.pdf. Add for the Conventions doc. Jan asked for.
% \label{sec:lorentz_model_sign}

% add remark that entries are less than 1 in loss function since jones matrices are passive in this context. why?


% TODO remark that there are a lot of parameters describing the waveplate and polarization state maybe ? Seems like a lot at least. (not important)
% TODO ! zero order does make sense ... retardance < 2pi (done)

%\begin{itemize}
%    \item back in time we used other loss function. (described in appendix)
%    \item that function gave the following design which we had printed by a third party.
%    same printer as previous samples but with slightly different printing and sintering parameters and in the meantime printer had been moved to new location.
%\end{itemize}

% TODO fix colors and change masson in legend to TAQ (example, fig 5.4) (done)
% TODO Jones paper 1941 hurwitz. unitary matrix can be represented as rotator + retarder
% TODO Add pic like fig 1 in https://doi.org/10.1364/AO.54.009758 (done)
% TODO add no, ne, bf plot of sapphire(?) (Appendix sellmeier stuff)
% TODO fix ylabel (done)
% TODO what about roundness of output state? (alpha=arctan(a/b)) (done)
%\begin{itemize}
%    \item loss function. Frequency dependent loss. (done)
%    \item Circularity. Linearity. (done, res1 + masson)
%    \item Ellipses, linear and circular (r/l?) input, shown output. (done,)
%    \item alpha, azimuth, delay, ellip angle (done)
    %\item Transmission along two directions (lin -> wp -> x-pol int(out), (lin -> wp -> y-pol int(out))). Probably not? Maybe just intensity(freq) is enough (done)
%    \item Retardance, Diattenuation (done)
%    \item See what interface does? (go away please...)
%    \item Eigenstates? (done)
%    \item compare to Result 1 -> we do shorter range because of absorption (thinner plate(s))
%    \item high abs compared to quartz (\cite{DGrischkowsky1990}) (done)
%    \item things that need mention but no plot probably: tot. pol. deg., inhomogenity, normality. (nah)
%    \item poincare sphere (take python ss, cba to do it in latex) (Enough in fbf sec.) (done)
%\end{itemize}


% TODO can conlude that the waveplate is most sensitive to bf variations which we unfortunately got a lot of.
% TODO add single waveplate phase shift to original measurement plot (done)
% TODO change measurement result to 6deg ... theres a reason right. (done)
% TODO mention teralyzer ?? (nah whatever lol)
% TODO Did I mention bowtie antennas (done)
% Big TODO change all horizontally etc. to abbrevations please ... (when read through, done)
% Note: handedness of result 3 is wrong because sign of l4 is wrong J10/J00 instead of J00/J10 (done. Mention at ellipse handedness or circ pol degree (+1)) (done)
% TODO write about periodicity compared to wavelength (done)
% TODO Reduced freq. res. (done)
% TODO Send pdf + page cnt to jan (done)
% magnitude of dichorism for formbf hips. (done) Is dichroism more important to account for here ? 
% TODO theory chapter intermission rewrite, no hermann waveplate. (done, I think)
% TODO maybe mention bandwidth? (done)

%\begin{itemize}
    %\item intro + Material selection (done)
    %\item Design description, maybe with photo. Optimization procedure see if still there somewhere. (done)
    %\item Maybe: expected parameters, dephasing, attenuation etc. poincare sphere when? (done)
    %\item measurement execution. Mes. of BF + WP is fairly simple, no need to add in chapter 3. (done)
    %\item evaluation briefly like masson, fits etc. Skip polarizer offset error -> no polar plots. (done)
    %\item add single waveplate to result plots for comparison like masson (replace csv file) (done)
    %\item why are the curves so smooth? (done, I think)
    %\item Result of eval shows large deviation (done)
    %\item explanation of deviation (done)
    %\item mention even data shows deviation? -> unlikely faulty evaluation (done)
    %\item unlikely due to dimension error through fabrication (maybe)
        %\item Wrong material parameters (done)
        %\item Dimension error from printing (done)
        %\item Theory for determination of bf wrong: -> comparison to CST (done)
    %\item expected birefringence vs measured from gratings. CST calculation + rytov (maybe transfer matrix method? nah -> add remark at presentation) (done)
    %\item indication: we see birefringence strongly influenced by printing direction https://archiv.ub.uni-marburg.de/diss/z2017/0786/ (done)
    %\item conclusion seems to work if we get the birefringence right. Not necessarily bad that it is higher but needs to be predictable. (done)
    %\item Show weirdly printed gratings at presentation (done)
%\end{itemize}

%\section{Fused silica glass} (removed)

% Todo ref? what?

% Habe ich nicht wirklich so genau gemacht. Habe die Proben so platziert dass das "Loch" im mount ungefähr abegedeckt war. Wird wahrscheinlich bis zu 5 mm fehler sein.

%The samples were positioned in such a way that the center of the samples was approximately aligned with the axis of rotation. For the rectangular samples center refers to the crossections of the diagonals. For the round samples with flat sides, this corresponds to the center of the smallest circle, which would still incircle the whole sample. This means that the sample was measured on the same position for all three orientations. Nevertheless, due to a potential slight mispositioning of the sample in the rotational mount, the position at which the sample was measured might slightly differ for the three orientations of the sample. Therefore, a small offset could be induced between the extracted refractive index for the three orientations, considering that there is a possible small inhomogenity in the sample thickness. However, this variation can be considered insignificant for the presented results.