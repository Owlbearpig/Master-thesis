\usepackage[T1]{fontenc}
\usepackage[utf8]{inputenc}
\usepackage[english]{babel}
\usepackage{csquotes}
\usepackage{graphicx}
\usepackage{tipa} % for the \ark{} command
\usepackage{graphics} % for pdf, bitmapped graphics files
\usepackage{times} % assumes new font selection scheme installed
\usepackage[leqno]{amsmath}
\usepackage{amssymb}
\usepackage{nicefrac}
%\usepackage{latexsym}
\usepackage{amscd}% for commutative diagrams
\usepackage{mathrsfs} %this package is for the script font \mathscr
\usepackage{mathtools}
\usepackage{relsize}
%\usepackage{pstricks}
\usepackage{theorem}
\usepackage{changepage}
\usepackage{euscript}
\usepackage{textcomp}
\usepackage{esvect}
\usepackage{parskip}
\usepackage{placeins}
\usepackage{subfigure}
\usepackage{stmaryrd}
\usepackage{fancyhdr}
\usepackage{graphpap}
\usepackage{makeidx}
\usepackage{enumerate}
\usepackage{esint}
\usepackage{datetime}
\usepackage{caption}
\usepackage{smartdiagram}
\usepackage{bm}
\usepackage[ruled,vlined]{algorithm2e}
\usepackage[version=4]{mhchem}
\usesmartdiagramlibrary{additions}

% SIunitx
\usepackage{siunitx}
\DeclareSIUnit{\volpercent}{Vol\%}

% Math commands
\DeclarePairedDelimiter{\diagfences}{(}{)}
\newcommand{\diag}{\operatorname{diag}\diagfences}
\DeclareMathOperator{\tr}{tr}

% Logic + calc
\usepackage{ifthen}
\usepackage{fp}

% For tables
\usepackage{makecell}
\usepackage{delarray}
\usepackage{array}
\usepackage{tabu}

% Set Abstract Page
\usepackage{abstract}
\setlength{\absleftindent}{-5mm}
\setlength{\absrightindent}{-5mm}

% Colour definitions - put before TikZ
\usepackage{color}
\definecolor{igreen}{rgb}{0.0, 0.56, 0.0}
\usepackage{xcolor, colortbl}
\colorlet{gred}{-red!75!green!65!}
\colorlet{mamber}{-red!75!green!15!blue!50!}
\colorlet{grown}{-red!75!blue!20!green}
\colorlet{bled}{-red!85!blue!40!green!45!}
\colorlet{waters}{cyan!25} % Define color for the water
\colorlet{water}{cyan!25!green!20!} % Define color for the water
\definecolor{grin}{HTML}{00F9DE}
\usepackage{rotating}
\providecommand{\keywords}[1]{\textbf{\textit{Keywords---}} #1}

% For faint dotted table line
\usepackage{arydshln}
\setlength{\dashlinedash}{.4pt}
\setlength{\dashlinegap}{.8pt}

% For drawings
\usepackage{booktabs}
\usepackage{graphicx}
\graphicspath{ {images/} }
\usepackage{tikz}
\usepackage{tikz-3dplot}
\usepackage{pgfplots}
\usepackage{pgfplotstable}
\pgfplotsset{compat=1.5}
\usepgfplotslibrary{fillbetween}
\usetikzlibrary{
angles,
arrows,
arrows.meta,
automata,
backgrounds,
calc,
decorations,
decorations.pathmorphing,
decorations.pathreplacing,
decorations.fractals,
decorations.markings,
external,
fit,
matrix,
petri,
positioning,
shadows,
shapes,
shapes.multipart,
topaths,
intersections,
bending,
quotes
}
\usepackage{eso-pic}
\def\ba{\begin{array}}
\def\ea{\end{array}}
\def\beann{\begin{eqnarray*}}
\def\eeann{\end{eqnarray*}}
\def\bea{\begin{eqnarray}}
\def\eea{\end{eqnarray}}
\def\bsy{\boldsymbol}
\def\gray#1{{\color{gray}#1}}
\usepackage[mode=buildnew]{standalone}

% COUNTERS
\setcounter{MaxMatrixCols}{20}
\renewcommand{\thesection}{\arabic{section}}
\renewcommand{\thesection}{\thechapter.\number\numexpr\value{section}}
\renewcommand{\thesubsection}{\thesection.\number\numexpr\value{subsection}}

% For changemargin
\def\quote{\list{}{\rightmargin\leftmargin}\item[]}
\let\endquote=\endlist 
\def\changemargin#1#2{\list{}{\rightmargin#2\leftmargin#1}\item[]}
\let\endchangemargin=\endlist 
\makeatletter
\newlength\qvec@height
\newlength\qvec@depth
\newlength\qvec@width
\newcommand{\qvec}[2][]{
    \settoheight{\qvec@height}{$#2$}
    \settodepth{\qvec@depth}{$#2$}
    \settowidth{\qvec@width}{$#2$}
  \def\qvec@arg{#1}
  \raisebox{.2ex}{\raisebox{\qvec@height}{\rlap{% 
    \kern.05em
    \begin{tikzpicture}[scale=1,shorten >=-3pt,shorten <=-3pt]
    \pgfsetroundcap
    \coordinate (Stx) at (.05em,0) ;
		\coordinate (Arx) at (\qvec@width-.05em,0) ;
    \draw[->](Stx) to[bend left] (Arx);
    \end{tikzpicture}
  }}}
  #2
}
\makeatother
\makeatletter
\newlength\pvec@height
\newlength\pvec@depth
\newlength\pvec@width
\newcommand{\pvec}[2][]{
    \settoheight{\pvec@height}{$#2$}
    \settodepth{\pvec@depth}{$#2$}
    \settowidth{\pvec@width}{$#2$}
  \def\pvec@arg{#1}
  \raisebox{.2ex}{\raisebox{\pvec@height}{\rlap{% 
    \kern.05em
    \begin{tikzpicture}[scale=1,shorten >=-3pt,shorten <=-3pt]
    \pgfsetroundcap
    \coordinate (Stx) at (.05em,0) ;
		\coordinate (Arx) at (\pvec@width-.05em,0) ;
    \draw[->](Stx) to[bend right] (Arx);
    \end{tikzpicture}
  }}}
  #2
}
\makeatother
\makeatletter
\newlength\vvec@height%
\newlength\vvec@depth%
\newlength\vvec@width%
\newcommand{\vvec}[2][]{%
  \ifmmode%
    \settoheight{\vvec@height}{$#2$}%
    \settodepth{\vvec@depth}{$#2$}%
    \settowidth{\vvec@width}{$#2$}%
  \else 
    \settoheight{\vvec@height}{#2}%
    \settodepth{\vvec@depth}{#2}%
    \settowidth{\vvec@width}{#2}%
  \fi%
  \def\vvec@arg{#1}%
  \def\vvec@dd{:}%
  \def\vvec@d{.}%
  \raisebox{.2ex}{\raisebox{\vvec@height}{\rlap{%
    \kern.05em%
    \begin{tikzpicture}[scale=1]
    \pgfsetroundcap
    \draw (.05em,0)--(\vvec@width-.05em,0);
    \draw (\vvec@width-.05em,0)--(\vvec@width-.15em, .075em);
    \draw (\vvec@width-.05em,0)--(\vvec@width-.15em,-.075em);
    \ifx\vvec@arg\vvec@d%
      \fill(\vvec@width*.45,.5ex) circle (.5pt);%
    \else\ifx\vvec@arg\vvec@dd%
      \fill(\vvec@width*.30,.5ex) circle (.5pt);%
      \fill(\vvec@width*.65,.5ex) circle (.5pt);%
    \fi\fi%
    \end{tikzpicture}%
  }}}%
  #2%
}
\makeatother
\def\ba{\begin{array}}
\def\ea{\end{array}}
\def\beann{\begin{eqnarray*}}
\def\eeann{\end{eqnarray*}}
\def\bea{\begin{eqnarray}}
\def\eea{\end{eqnarray}}
\def\bsy{\boldsymbol}
\def\gray#1{{\color{gray}#1}}
\usepackage{titlesec}
\usepackage{multirow}

% To reference within text
\usepackage[colorlinks]{hyperref}
\usepackage{lipsum}
\usepackage{tikz-cd}
\usepackage{float}
\usepackage{titling}
\usepackage{epigraph}
\usepackage[title, titletoc]{appendix}
\setlength\epigraphwidth{8cm}
\setlength\epigraphrule{0pt}

\titleformat{\chapter}{\normalfont\LARGE}{\thechapter\,$\vert$}{20pt}{\LARGE}{\setcounter{chapter}{0}}
\setlength{\headheight}{15pt}
\titlespacing*{\chapter}{0pt}{-70pt}{40pt} %Move chapter titles up

% Title page logos:
\makeatletter
\newcommand\BackgroundPicturea[3]{
	\setlength{\unitlength}{1pt}
	\put(0,\strip@pt\paperheight){
		\parbox[t]{\paperwidth}{
			\vspace{#2}\hspace{#3}
			\mbox{\includegraphics[scale=0.5]{#1}}
}}}
\newcommand\BackgroundPictureb[3]{
	\setlength{\unitlength}{1pt}
	\put(0,\strip@pt\paperheight){
		\parbox[t]{\paperwidth}{
			\vspace{#2}\hspace{#3}
			\mbox{\includegraphics[scale=0.3]{#1}}
}}}
\makeatother

% For my abbreviations
\newcommand{\abbrlabel}[1]{\makebox[3cm][l]{\textbf{#1}\ \dotfill}}
\newenvironment{abbreviations}{\begin{list}{}{\renewcommand{\makelabel}{\abbrlabel}}}{\end{list}}

% Line Spacing
\usepackage{setspace}

% Set of command is for the changemargin environment
\def\quote{\list{}{\rightmargin\leftmargin}\item[]}
\let\endquote=\endlist 
\def\changemargin#1#2{\list{}{\rightmargin#2\leftmargin#1}\item[]}
\let\endchangemargin=\endlist

% Replace Contents to Table of Contents	
\addto\captionsenglish{
	\renewcommand{\contentsname}%
	{Table of Contents}
	\setcounter{tocdepth}{3}% Include \subsubsection in ToC
	\setcounter{secnumdepth}{3}% Number \subsubsection in ToC
	}
\renewcommand{\listfigurename}{List of Figures}
\renewcommand{\listtablename}{List of Tables}

\bibliographystyle{unsrtnat}
\usepackage[
    natbib=true,
    style=numeric,
    sorting=none,
    url = false
]{biblatex}

\DeclareSourcemap{ % Mendeley sucks balls. https://tex.stackexchange.com/questions/428959/get-biblatex-to-show-url-for-webpage-references
  \maps[datatype=bibtex, overwrite=true]{
    \map{
      \step[fieldsource=url, final]
      \step[typesource=misc, typetarget=online]
    }
  }
}

% \addbibresource{references.bib}
\addbibresource{references_new.bib}

% Roman numbering for equations
%\renewcommand{\theequation}{\roman{equation}}    url = false

